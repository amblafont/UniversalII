We work in the setting of extensional type theory, although we occasionally indicate through $e_1,\dots,e_n \#
t$ that $t$ is well-typed thanks to the equalities $e_1$,\dots,$e_n$.
% We are also careful to explain where UIP or functional extensionality that is
% available in extensional type theory are used.

The steps consist of the following:
\begin{enumerate}
\item Definition of untyped syntax (as a family of inductive datatypes) together
  with typing judgments (as inductive relations on the untyped
  syntax), and construction of a model of the theory of IIT
  signatures from well-formed terms, denoted $S$.
\item Construction of a morphism $\rec:S\to M$ for arbitrary $M$, by:
  \begin{enumerate}
  \item defining a relation $\blank \sim \blank$ between the syntax and a given
    model (the idea is that given a syntactic context $\Gamma$ and a semantic
    context $\model{\Gamma}$ of the model $M$, we have $\Gamma \sim
    \model{\Gamma}$ if and only if $\rec\,\Gamma = \model{\Gamma}$, and
    similarly for types, terms, and substitutions);
    % semantic context $\model{\Gamma}$ of
    % the model $M$ if and only if $\Gamma$ relates to $\rec\,{\Gamma}$;
    % enjoyed by the initial morphism from the
    % syntax to a given model;
  \item Showing that this relation is functional;
    \end{enumerate}
  \item Proving of the uniqueness of this morphism by showing that
    any morphism $f:S\to M$ satisfies the relation
    (in the sense, for contexts, that given any syntactic context $\Gamma$, we
    have $\Gamma\sim f\,\Gamma$).
      % TODO donner composantes sur les contextes
\end{enumerate}
The next sections detail each of these steps.
\subsection{Syntactic model}

The goal is to define the syntactic model by packing the untyped syntax
with the typing judgments:
% % \subsection{QIIT model}
% Then, we show that they are stable by weakening and substitution.
% Packing the untyped syntax with well-formed typed judgments yields
% a model $S$ of the QIIT. Let us give the most important components:
\begin{alignat*}{5}
 & \syn\Con && := \sum_\Gamma \Gamma\vdash
 \\
 & \syn\Ty \,(\Gamma,\wf\Gamma)&& := \sum_A \Gamma\vdash A
 \\
 & \syn\Tm \,(\Gamma,\wf\Gamma)(A,\wf A)&& := \sum_t \Gamma\vdash t \in A
 \\
 & \syn\Sub \,(\Gamma,\wf\Gamma)(\Delta,\wf \Delta)&& := \sum_\sigma \Gamma\vdash \sigma \Rightarrow \Delta
\end{alignat*}
The next sections are devoted to the definition of the untyped syntax and the
typing judgments.

\subsubsection{Untyped syntax}
The untyped syntax is defined as the following inductive datatype:

\begin{alignat*}{5}
  & \rlap{(1) Substitution calculus} \\[0.5em]
  & \Conu && : \Set \\
  & \Tyu  && : \Set \\
  & \Subu  && : \Set \\
  % & \Subpre  && : \Set \\
  & \Tmu  && : \Set \\
  & \emptyu && : \Conu \\
    & \epsilonu && : \Subu \\
  & \blank\extu\blank && : \Conu\ra\Tyu\ra\Conu \\
  & \blank\consu\blank && : \Subu\ra\Tmu\ra\Subu \\
  & \varu  && : \N \ra \Tmu \\
  & \rlap{(2) Sorts and operators} \\[0.5em]
  & \Uu && : \Tyu \\
  & \Elu && : \Tmu\ra\Tyu \\
  & \rlap{(3) Parameters} \\[0.5em]
  & \Piu && : \Tmu\ra\Tyu\ra\Tyu \\
  & \blank\appu\blank && : \Tmu\ra\Tmu\ra \Tmu \\
  & \rlap{(4) Metatheoretic and infinitary parameters} \\[0.5em]
  & \Pimu && : (T:\Set)\ra(T\ra\Tyu)\ra\Tyu \\
  & \Piiu && : (T:\Set)\ra(T\ra\Tmu)\ra\Tmu \\
  & \blank\appmu \blank && : \Tmu\ra(\alpha:T) \ra\Tmu\\
  & \rlap{(5) Default value} \\[0.5em]
  & \erru && : \Tmu
\end{alignat*}
Note that we choose to share the constructor for untyped application to infinitary and
meta-theoretic parameters.
Variables are modelled as De Bruijn indices, i.e. as natural numbers pointing
to the relevant position in the context.

% The type of (untyped) substitution is then defined as $\Subu = \List\,\Tmu$.
% Note that $\Conu$ could also be defined as a list of types in a similar fashion.

\subsubsection{Untyped weakening}


% Substitutions of types and terms are defined recursively.


Note that $(\Piu \,A\,B)[\sigma]$ should be defined as $\Piu
\,A[\sigma]\,B[\wk\,\sigma]$, and thus we need to define $\wk$, the weakening substitution.
The basic idea is to increment the de Bruijn indices of all the variables.
Actually this is not so simple because of the $\Piu$ type:
indeed, we want to define $\wk(\Piu\,A\,B)$ as the $\Pi$ type of the weakening
of $A$ and $B$, but here, $B$ must be weakened with respect to the second last
variable of the context, rather than the last one.
For this reason, we need to generalize the weakening as occuring anywhere in the context.
\begin{alignat*}{5}
  & \wk_n && :  \Tyu\ra\Tyu \\
  & \wk_n && :  \Tmu\ra\Tmu \\
  % & \wk && : \N \ra \N\ra\N \\
  & \wkO && : \Subu\ra \Subu
  \end{alignat*}
  The natural number $n$ specifies at which position of the context the
  weakening occurs.
  Here, $\wkO$ weakens with respect to the last variable.

  \subsubsection{Untyped substitution}
  We define by induction the specific substitution of one variable of the context:
  % & \rlap{(1) Substitution calculus} \\[0.5em]
\begin{alignat*}{5}
  & \blank[\blank := \blank] && : \Tyu\ra\N\ra\Tmu\ra\Tyu \\
  & \blank[\blank := \blank] && : \Tmu\ra\N\ra\Tmu\ra\Tmu
  % \\
  % & \blank[\blank := \blank] && : \N\ra\N\ra\Tmu\ra\Tmu
  \end{alignat*}
  This is enough to define the well-typed judgments: indeed, the typing rule of application
  involves only a unary substitution.


  However, to construct the initial model of the QIIT, we need to define
  full substitution calculus:
  % First, we define

  % We then define by mutual recursion the full substitution of types and terms.
\begin{alignat*}{5}
  & \blank[\blank] && : \Tyu\ra\Subu\ra\Tyu \\
  & \blank[\blank] && : \Tmu\ra\Subu\ra\Tmu \\
  & \blank \circ \blank && : \Subu\ra\Subu\ra \Subu
  \end{alignat*}
  These can be defined either by iterating unary substitutions, or
  by recursion on the syntax on the untyped syntax: the two ways yield provably
  equal definitions. In the following, we assume that it is defined by recursion.
 We use the additional default constructor
$\erru:\Tmu$ in case of error (for example, when the substitution is not
long enough).
This is an irrelevant constructor in the end, as the typing
judgments will not mention it.

We also make use of the following definition:
  \begin{alignat*}{5}
  % liftV (S (n + p)) (liftV n q) ≡ liftV n (liftV (n + p) q)
  & \keep && : \Subu \ra \Subu \\
   &&& := \lambda \sigma. \varu\, 0 \,\consu\, \wkO\,\sigma
  \end{alignat*}
  The idea is that if $\sigma$ is a substitution between contexts $\Gamma$ and
  $\Delta$, then $\keep\,\sigma$ is a substitution between contexts $\Gamma, A[\sigma]$
  and $\Delta,A$ for any type $A$. This occurs when defining
  $(\Piu\,A\,B)[\sigma]$ as $\Piu (A[\sigma])(B[\keep \,\sigma])$.




  We can define the identity substitution on a context $\Gamma$ as follows,
  where $\length{\Gamma}$ is the length of the context $\Gamma$, and
  $\keep^{\length{\Gamma}}$ is $\keep$ iterated $\length{\Gamma}$ times:
\begin{alignat*}{5}
  & \length{\Gamma} && : \Conu\ra\N
  \\
  & \idu && : \Conu \ra \Subu \\
  & && := \lambda \Gamma.\keep^{\length{\Gamma}} \epsilonu
  \end{alignat*}


  \subsubsection{Exchange laws: weakening and substitution}
  Many lemmas are shown by recursion on the untyped syntax, for terms and types
  (below, $Z$ denotes either a term or a type):
\begin{alignat*}{5}
  % liftV (S (n + p)) (liftV n q) ≡ liftV n (liftV (n + p) q)
  & \wk\mhyphen\wk && : \wk_{n+p+1} (\wk_n\,Z) = \wk_n(\wk_{n+p}\,Z)\\
  % n-subZ-wkZ : ∀ n A z → (lifZZ n A) [ n ↦ z ]Z  ≡ A
  & \wk_n[n] && : (\wk_{n} Z)[ n := z] = Z \\
  % lifZ-l-subV : ∀ n p u x → lifZZ (n + p) (x [ n ↦ u ]V) ≡ (lifZV (S (n + p)) x) [ n ↦ (lifZZ p u) ]V
  & \wk_+[] && : (\wk_{n+p+1} Z)[ n := \wk_p u] = \wk_{n+p}(Z[n := u]) \\
  % l-subV-lifZV : ∀ Δ u n x → (lifZV n x) [ (S (n + Δ)) ↦ u ]V  ≡ lifZZ n (x [ (n + Δ) ↦ u ]V)
  & \wk[+] && : (\wk_{n} Z)[ n+p+1 :=  u] = \wk_{n}(Z[n+p := u]) \\
  % l-subV-l-subV : ∀ n p z u x →  (x [ n ↦ u ]V) [ (n + p) ↦ z ]Z  ≡ (x [ (S (n + p)) ↦ z ]V) [ n ↦ (u [ p ↦ z ]Z) ]Z
  & [][+] && : Z[n :=u][n+p:=z] = Z[n+p+1:=z][n:= (u[p:=z])]
  \end{alignat*}
  The following lemmas are shown by recursion on the syntax, for types and terms:
\begin{alignat*}{5}
  % liftV (S (n + p)) (liftV n q) ≡ liftV n (liftV (n + p) q)
  % liftV=wkS
  & [\keep^n\mhyphen \wkO] && : Z[\keep^n (\wkO\,\sigma)] = \wk_n (Z[\keep^n\,\sigma])
  \\
  % lifZₛV : ∀ n xp σp Zp → (lifZV n xp [ iZer n keep (Zp ∷ σp) ]V) ≡ (xp [ iZer n keep σp ]V)
  & \wk_n[\keep^n\mhyphen\cons]  & &
  :
  (\wk_n Z) [\keep^n (u \consu \sigma)] = Z [ \keep^n \sigma]
  \\
  % l-sub[]V : ∀ n x z σ
    % ((x [ n := z ]V) [ iZer n keep σ ]Z) = (x [ iZer (S n) keep σ ]V) [ n := (z [ σ ]Z) ]Z
   & [:=][\keep] && : Z [ n:= u][\keep^n \sigma] =  Z[\keep^{n+1}\,\sigma][n := u[\sigma]]
   \\
   & {}{\circ}{}\wkO & & : \sigma \circ (\wkO \tau) = \wkO (\sigma \circ \tau)
   \\
  & \wkO{}{\circ}{}, && : \wkO\,\sigma \circ (t \consu \tau ) = \sigma \circ \tau
  \end{alignat*}
  where $\keep^n$ is $\keep$ iterated $n$ times. As a particular case
  for $n=0$, we get
\begin{alignat*}{5}
  & [\wkO] && : t[\wkO\,\sigma] = \wk_0 (t[\sigma]) \\
  & \wk_0[\cons] && : (\wk_0 Z)[u \consu \sigma] = Z[\sigma] \\
   & [0:=][] && : Z [ 0:= u][ \sigma] =  Z[\keep\,\sigma][0 := u[\sigma]]
  \end{alignat*}

Finally, we show the usual lemmas about substitution:
\begin{alignat*}{5}
  & [][] && : Z[\sigma][\tau] = Z[\sigma\circ\tau]
  \\
  & \ass && : (\sigma\circ \delta)\circ\tau = \sigma\circ (\delta\circ\tau)
  \end{alignat*}
  Other substitution lemmas are deferred after the definition of the typing
  judgments, as the proofs require that some input are well-typed.

  % The two ways yield provably equal definitions, but
  % The advantage of the second
  % one in a type theory with UIP and function extensionality is that we get
  % definitional equalities such as $(t\,\appu\,u)[\sigma]=\app\,t[\sigma]\,u[\sigma]$.

  % \\
  % & \blank[\blank := \blank] && : \N\ra\N\ra\Tmu\ra\Tmu


  \subsubsection{Typing judgments}
  The typing judgments are defined as the following inductive datatype indexed over the
  untyped syntax:
\begin{alignat*}{5}
  & \rlap{(1) Substitution calculus} \\[0.5em]
  & \blank \vdash && : \Conu\ra \Set \\
  & \blank \vdash \blank  && : \Conu\ra \Tyu\ra\Set \\
  & \blank \vdash \blank \in \blank  && : \Conu\ra \Tmu\ra \Tyu\ra \Set \\
  & \blank \vdash \blank \invar \blank  && : \Conu\ra \N \ra \Tyu\ra \Set \\
  & \blank \vdash \blank \Rightarrow \blank  && : \Conu\ra \Subu \ra \Conu\ra \Set \\
  % & \Subpre  && : \Set \\
  & \emptyw && : \emptyu \vdash \\
  & \epsilonw && : \Gamma\vdash \epsilonu \Rightarrow \emptyu \\
  & \blank\extw\blank && : ( \Gamma\vdash )  \ra  (\Gamma\vdash A)  \ra
  \Gamma\extu A \vdash \\
  & \consw && :
    (\Delta \vdash) \ra
    (\Gamma\vdash \sigma \Rightarrow\Delta)\ra
    ( \Delta \vdash A) \ra
    ( \Gamma \vdash t \in A[\sigma]) \ra
    \Gamma \vdash t \consu \sigma \Rightarrow \Delta\extu A
   \\
  & \varw  && : (\Gamma \vdash n \invar A) \ra \Gamma \vdash \varu n \in A \\
  & \vzw  && : (\Gamma\vdash)\ra(\Gamma\vdash A)\ra\Gamma \extu  A \vdash 0 \invar \wku \, A \\
  & \vsw  && : (\Gamma\vdash)\ra(\Gamma\vdash A)\ra(\Gamma \vdash n \invar A) \ra (\Gamma \vdash B) \ra \Gamma \extu  B
  \vdash S\,n \invar \wku \, A \\
  & \rlap{(2) Sorts and operators} \\[0.5em]
  & \Uw && : (\Gamma \vdash)\ra \Gamma \vdash \Uu \\
  & \Elw && : (\Gamma \vdash)\ra (\Gamma \vdash a \in \Uu)\ra\Gamma \vdash \Elu\,a \\
  & \rlap{(3) Parameters} \\[0.5em]
  & \Piw && :
    (\Gamma \vdash)\ra (\Gamma \vdash a \in \Uu)\ra(\Gamma \extu
  \Elu\,a \vdash B)\ra\Gamma \vdash \Piu \, a \, B \\
  & \appw && :
    (\Gamma \vdash)\ra (\Gamma \vdash a \in \Uu)\ra(\Gamma \extu
    \Elu\,a \vdash B)\ra
    \\
    & &&
    (\Gamma \vdash t \in \Piu \, a \, B )
    \ra
    (\Gamma \vdash u \in \Elu\,a)
    \ra
    \Gamma \vdash t \appu u \in  B [0 := u] \\
  & \rlap{(4) Metatheoretic parameters} \\[0.5em]
  & \Pimw && :
    (T:\Set)\ra(A : T\ra \Tyu)\ra(\Gamma \vdash)\ra
    ((t : T)\ra\Gamma\vdash A\,t) \ra
    \Gamma \vdash \Pimu \,T\,A
    \\
    & \appmw  && :
    (T:\Set)\ra(A : T\ra \Tyu)\ra(\Gamma \vdash)\ra
    ((t : T)\ra\Gamma\vdash A\,t) \ra
    \\
    & &&
    (\Gamma \vdash t \in \Pimu \,T\,A)
    \ra (u : T) \ra \Gamma \vdash t\,\appmu\,u \in A\,u
    \\
  & \rlap{(5) Infinitary parameters} \\[0.5em]
  & \Piiw && :
    (T:\Set)\ra(A : T\ra \Tmu)\ra(\Gamma \vdash)\ra
    ((t : T)\ra\Gamma\vdash A\,t\in \Uu)
    \\ &&& \ra
    \Gamma \vdash \Pimu \,T\,A \in \Uu
  \\
  & \appiw && :
    (T:\Set)\ra(A : T\ra \Tmu)\ra(\Gamma \vdash)\ra
    ((t : T)\ra\Gamma\vdash A\,t\in\Uu)
    \\
    & &&
    \ra
    (\Gamma \vdash t \in \Piiu \,T\, A)
    \ra (u : T) \ra \Gamma \vdash t\,\appmu\,u \in A\,u
\end{alignat*}
Note that there is possibility of redundancy in the arguments of
the constructors. Here, we are rather ``{paranoid}'', so that we get more inductive
hypotheses when performing recursion.

% The substitution part could be moved after the mutual definition of terms, types,
% and contexts.

\subsubsection{Typed weakening}
Untyped weakening performs weakening anywhere in the context.
Stating a typing rule for this operation requires to have a way to express
the weakening occuring at the middle of a context.
To this end, we consider pairs of untyped contexts, which should be thought
of as a splitting at some place of a full context.
This full context is recovered by merging the two components:
% (which appears in the typing rule of a weakening occuring in the middle of a context),
% and the length $\length{\Gamma}$ of a
% context $\Gamma$:
\begin{alignat*}{5}
  & \blank \merge \blank && :  \Conu\ra\Conu\ra\Conu \\
  & \Gamma \merge \cdot && :=  \Gamma \\
  & \Gamma \merge (\Delta \extu A) && :=  (\Gamma \merge \Delta)\extu A
\end{alignat*}
We think of the second context as a telescope over the first context.
We need to define the define also the weakening of a telescope in order to
state the typing rule of weakening:
\begin{alignat*}{5}
  & \wkO && : \Conu\ra \Conu
  \\
  & \wkO\,\emptyu && := \emptyu
  \\
  & \wkO\,(\Delta \extu A) && := \wkO\,\Delta \extu \wk_{\length{\Delta}}\,A
  \end{alignat*}
Now we state and prove by mutual recursion that typing judgments are stable under
weakening, for contexts, types, terms, and substitutions:
% wkTelw : ∀ {Γp}{Ap}(Aw : Γp ⊢ Ap)Δp (Δw : (Γp ^^ Δp) ⊢) → ((Γp ▶p Ap) ^^ wkTel Δp) ⊢
% liftTw : ∀ {Γp}{Ap}(Aw : Γp ⊢ Ap)Δp{Bp}(Bw : (Γp ^^ Δp) ⊢ Bp) → ((Γp ▶p Ap) ^^ wkTel Δp) ⊢ (liftT ∣ Δp ∣ Bp)
\begin{alignat*}{5}
  & \wf{\wkO} && : (\Gamma \vdash A)\ra(\Gamma \merge \Delta\vdash) \ra
  \Gamma \extu A \merge \wkO\, \Delta \vdash \\
  % liftV (S (n + p)) (liftV n q) ≡ liftV n (liftV (n + p) q)
  & \wf{\wk} && : (\Gamma \vdash A)\ra(\Gamma \merge \Delta\vdash B) \ra
  \Gamma \extu A \merge \wkO\,\Delta \vdash \wk_{\length{\Delta}}\, B  \\
  & \wf{\wk} && : (\Gamma \vdash A)\ra(\Gamma \merge \Delta\vdash t\in B) \ra
  \Gamma \extu A \merge \wkO\,\Delta \vdash \wk_{\length{\Delta}}\, t \in \wk_{\length{\Delta}}\,B  \\
  & \wf{\wkO} && : (\Gamma \vdash A)\ra(\Gamma \vdash \sigma
  \Rightarrow \Delta) \ra
  \Gamma \extu A \vdash \wkO\, \sigma \Rightarrow \Delta
  \end{alignat*}
\subsubsection{Typed substitution}
  We show that judgments are stable under substitution, by mutual recursion:
\begin{alignat*}{5}
  % Tyw[] : ∀ {Γp}{Ap}(Aw : Γp ⊢ Ap) {Δp}(Δw : Δp ⊢){σp}(σw :  Δp ⊢ σp ⇒ Γp) → Δp ⊢ (Ap [ σp ]T)
  & \wf{[]} && : (\Gamma \vdash)\ra(\Delta \vdash A)\ra
  (\Gamma \vdash \sigma \Rightarrow \Delta) \ra
  \Gamma \vdash A [ \sigma ]
  \\
  & \wf{[]} && : (\Gamma \vdash)\ra(\Delta \vdash t \in A)\ra
  (\Gamma \vdash \sigma \Rightarrow \Delta) \ra
  \Gamma \vdash t [\sigma] \in A [ \sigma ]
  \\
  & \wf{[]} && : (\Delta \vdash x \invar A)\ra
  (\Gamma \vdash \sigma \Rightarrow \Delta) \ra
  \Gamma \vdash x [\sigma] \in A [ \sigma ]
  % ∘w : ∀ {Γ} {Δ σ} (σw :  Δ ⊢ σ ⇒ Γ)
  %  {Y}(Yw : Y ⊢) {δ} (δw :  Y ⊢ δ ⇒ Δ) →
  %  Y ⊢  (σ ∘p δ) ⇒ Γ
  \\
  & \wf{{\circ}} && :
  (\Gamma \vdash) \ra
  (\Gamma \vdash \sigma \Rightarrow \Delta) \ra
  (\Delta \vdash \tau \Rightarrow E) \ra
  \Gamma \vdash \tau \circ \sigma \Rightarrow E
  % keepw : ∀ {Γp}(Γw : Γp ⊢){Δp}(Δw : Δp ⊢){σp}(σw :  Γp ⊢ σp ⇒ Δp) {Ap}(Aw : Δp ⊢ Ap ∈ Up ) → (Γp ▶p (Elp Ap [ σp ]T )) ⊢ (keep σp) ⇒ (Δp ▶p Elp Ap)
  % & \wf{keep-\El} && : (\Gamma \vdash)\ra(\Delta \vdash A)\ra
  % (\Gamma \vdash \sigma \Rightarrow \Delta) \ra
  % \Gamma \vdash A [ \sigma ]
  % \\
  \end{alignat*}
Then, we show that identity substitutions is neutral, for well-typed types,
terms, and substitutions:
\begin{alignat*}{5}
  % [idp]V : ∀ {Γ}{A}{x}(xw : Γ ⊢ x ∈v A) → (x [ idp ∣ Γ ∣ ]V) ≡ V x
  & [\idu] && : (\Gamma \vdash A)\ra A [ \idu\,\Gamma]  \\
  & [\idu] && : (\Gamma \vdash x \invar A)\ra x [ \idu\,\Gamma] = V x \\
  & [\idu] && : (\Gamma \vdash t \in A)\ra t [ \idu\,\Gamma] = t \\
  & \idru && : (\Gamma \vdash \sigma \Rightarrow\Delta)\ra \sigma \circ \idu\,\Gamma = \sigma
  \end{alignat*}
  In a seperate and independant recursion, one shows that:
  \begin{alignat*}{5}
  & \idlu && : (\Gamma \vdash \sigma \Rightarrow\Delta)\ra \idu\,\Delta\circ \sigma = \sigma
  \end{alignat*}
  Then, we show that the identity substitution itself is well typed:
  \begin{alignat*}{5}
  & \idw && : (\Gamma \vdash)\ra \Gamma \vdash \idu\,\Gamma \Rightarrow \Gamma
  \end{alignat*}



\subsubsection{Uniqueness of typing and uniqueness of types}
  We define what it means for a type $A$ to be a proposition as follows:

  \[
    \isaprop\,A := (a : A) \ra (a' : A) \ra a = a'
  \]
  It states that there is at most one inhabitant.
We show by recursion that each of these typing judgments is unique in the
following sense:
\begin{align*}
    \isp{\Conw}
  : \isaprop\, (\Gamma\vdash)
 & &
    \isp{\Tmw}
  : \isaprop\, (\Gamma\vdash t \in A)
    \\
    \isp{\Tyw}
  : \isaprop\, (\Gamma\vdash A)
    &&
    \isp{\wf\Var}
  : \isaprop\, (\Gamma\vdash x \invar A)
  \\
  &
        \isp{\wf\Sub}
  : \isaprop\, (\Gamma\vdash \sigma \in \Delta)
  \end{align*}
Actually, the application and successor cases require to show uniqueness of
types in a given context, before.
Let us consider for instance the application constructor $\appw$: it provides a
type $B$ such that the final type is $C=B[0 := u]$. Even if $C$ is known a
priori, there may be another $B$ for which $B[0 := u] = C$, possibly leading to
many proofs that $t\appu u$ has type $C$. Uniqueness of types
solves this issue, as $B$ is then uniquely determined by the type $\Piu\,A\,B$ of $t$.
More formally, we show by recursion:
\begin{alignat*}{5}
  &
  \Tmw{=}\Ty & &:
  (\Gamma \vdash t \in A) \ra
  (\Gamma \vdash t \in B) \ra A = B
  \\
  &
  \wf\Var{=}\Ty & &:
  (\Gamma \vdash x \invar A) \ra
  (\Gamma \vdash x \invar B) \ra A = B
  \end{alignat*}

\subsubsection{UIP and functional extensionality}
None of these constructions use UIP. Functional extensionality is
 necessary because the metatheoretic or infinitary (untyped) $\Pi$ takes a metatheoretic function
 as an argument. An example of induction step that uses it lies in the proof
 that identity substitutions do not change types (this is an equation satisfied
 by a model of the QIIT), in particular in the case  $(\Pim\,T\,
 A)[\id]=\Pim\,T\,A$. Indeed, the left hand side of this equation is
 equal to $\Pim\,T\,(\lambda t.(A\,t)[\id])$ by definition, whereas the induction hypothesis
 states that $(t:T)\ra (A\,t)[\id]=A\,t$.
 % , which is not applyable without
 % functional extensionality.


\subsection{Functional relation enjoyed by initial morphism}
The previous section builds the syntactic model.
The challenge remains to show that it is initial.

Given a model $M$ of the QIIT, we define the functional relation enjoyed by the initial
morphism $\rec : S\to M$ by recursion on the typing judgments.
% As these judgments are propositions, we will sometimes
If $\Gamma$ is a context in $S$ and $\model{\Gamma}$ is a semantic
context (i.e. a context of the model $M$), we want to define a type $\Gamma\sim \model{\Gamma}$
equivalent to $\rec(\Gamma)=\model{\Gamma}$. Of course, at this
stage, $\rec$ is not available yet since the whole point of defining this
relation is to construct $\rec$ in the end. For a type $A$ in a context $\Gamma$,
it is natural to define the relation with a semantic type $\model{A}$ in a
semantic context $\model{\Gamma}$ such that it is equivalent to  $(\rec\,\Gamma =
\model{\Gamma})\times (\rec\,A = \model{A})$ (note that the first equality should
be used to transport the right hand side of the second one).
The discussion is similar for terms.

The ultimate goal is to prove that
$\sum_{\model{\Gamma}}\Gamma\sim\model{\Gamma}$ and
$\sum_{\model{\Gamma}}({\model{A}:\model{\Ty}\,\model{\Gamma}})\times(A\sim\model{A})$
are contractible. The first step consists in showing that these types are propositions.
% We explain below that this requires UIP.
Actually, we take a different approach for the relation on types
(and terms) which yields a more concise definition of the relation.
The adapted goal, for types, will be to prove that
$\sum_{\model{A}}A\sim\model{A}$ is a proposition, and
is inhabited if if $\Gamma \sim \model{\Gamma}$ is.
 This means that in the definition of the relation, we assume
that contexts are already related, so that we don't need to enforce them to be,
as it is the case in the original approach. We give the two versions of the universe
case below, to explain the differences.


% Many of the terms below are well-typed because the target model satisfies some
% equalities that are reflected as definitional equalities.
% In the formalization, we even postulate rewrite rules for some of the equations that the
% model $M$ should satisfy (otherwise, we are faced with too many hellish transports).
% This is, in some sense, another place where UIP is needed, as it is known that
% extensional type theory can be translated in intensional type theory using UIP
% and functional extensionality. This requirement can be qualified: we
% could say that we only claim to eliminate to models that definitionally
% satisfies the equalities that we enforce, although we haven't checked that this is enough to
% construct any IIT from the universal one (and anyway, UIP and functional
% extensionality are needed to construct any IIT from the universal one).

Here we provide the definition of the relation by recursion on the typing judgments.
In the definitions, we abbreviate $\wf{A} \sim (\model{\Gamma}\vdash \model{A})$ by
$\wf{A}\sim\model{A}$ when $\model{\Gamma}$ is clear, and similarly for terms.
\begin{alignat*}{5}
  & \rlap{(1) Substitution calculus} \\[0.5em]
  & \blank \sim{ \blank} && :  \Gamma \vdash \, \ra \model{\Con} \ra \Set \\
  & \blank \sim (\model{\Gamma}\vdash  \blank) && :  \Gamma \vdash A \, \ra \model{\Ty}\,\model{\Gamma} \ra \Set \\
  & \blank \sim{ (\model{\Gamma}\vdash  \blank \in \model{A})} && :  \Gamma \vdash t \in A \, \ra \model{\Tm}\,\model{\Gamma}\,\model{A} \ra \Set \\
  & \blank \sim{ (\model{\Gamma}\vdash  \blank \in \model{A})} && :  \Gamma \vdash x \invar A \, \ra \model{\Tm}\,\model{\Gamma}\,\model{A} \ra \Set \\
  & \blank \sim{ (\model{\Gamma}\vdash  \blank \Rightarrow \model{\Delta})} && :  \Gamma \vdash \sigma \Rightarrow \Delta \, \ra \model{\Sub}\,\model{\Gamma}\,\model{\Delta} \ra \Set \\
  \\
  & \emptyw\sim \model{\Gamma} && := \model{\Gamma}=\model{\cdot} \\
  & \epsilonw\sim
      ( \model{\Gamma} \vdash \model{\delta}\Rightarrow \model{E})&&
 := (e_E : \model{E} = \model{\cdot})\times (\model{\delta} = e_E \#
 \model{\epsilon})
  \\
  & (\wf{\Gamma}\extw \wf{A})\sim \model{\Delta} &&
   :=
    \sum_{\model{\Gamma}} (\wf{\Gamma}\sim\model{\Gamma}) \times
    \sum_{\model{A}} (\wf{A}\sim\model{A}) \times
    (\model{\Delta} = \model{\Gamma} \model{\ext} \model{A})
    \\
    & (\consw \wf{\Delta}\wf{\sigma}\wf{A}\wf{t})\sim
    \\
    & \qquad
    ( \model{\Gamma} \vdash \model{\delta}\Rightarrow \model{E})&&
   :=
    \sum_{\model{\Delta}} (\wf{\Delta}\sim\model{\Delta}) \times
    \sum_{\model{\sigma}} (\wf{\sigma}\sim\model{\sigma}) \times
    \\
    & && \qquad
    \sum_{\model{A}} (\wf{A}\sim\model{A}) \times
    \sum_{\model{t}} (\wf{t}\sim\model{t}) \times
    \\
    & && \qquad
    (e_E : \model{E} = \model{\Delta} \model{\ext} \model{A}) \times
    (\delta = e_E \# \model\sigma \model{\cons} \model{t} )
    \\
  & \varw\,\wf{x} \sim \model{t}
   && := \wf{x}\sim \model{t} \\
  & \vzw\wf{\Gamma}\wf{A}
  \sim \\
  & \qquad (\model{\Delta}\vdash \model{t}\in \model{B} )
   && :=
    \sum_{\model{\Gamma}} (\wf{\Gamma}\sim\model{\Gamma}) \times
    \sum_{\model{A}} (\wf{A}\sim\model{A}) \times \\
    & &&
    \qquad
  (e_\Delta : \model{\Delta} = \model{\Gamma} \model{\ext} \model{A})
     \times
     (e_B : \model{B} = e_\Delta \# \model{\wk}\,\model{A}) \times
     \\
     & && \qquad \model{t} = e_\Delta,e_B \# \model{\vz}
  \\
  & \vsw \wf{\Gamma} \wf{A} \wf{n} \wf{B} \sim
  \\
  & \qquad
     (\model{\Delta}\vdash \model{t}\in \model{C} )
   && :=
    \sum_{\model{\Gamma}} (\wf{\Gamma}\sim\model{\Gamma}) \times
    \sum_{\model{A}} (\wf{A}\sim\model{A}) \times
    \sum_{\model{B}} (\wf{B}\sim\model{B}) \times
    \sum_{\model{n}} (\wf{n}\sim\model{n}) \times
    \\
    & &&
    \qquad
  (e_\Delta : \model{\Delta} = \model{\Gamma} \model{\ext} \model{B})
     \times
     (e_C : \model{C} = e_\Delta \# \model{\wk}\,\model{A}) \times
     \\
     & && \qquad
     \model{t} = e_\Delta,e_C \# \model{\vs} \, \model{n}
  \\
  & \rlap{(2) Sorts and operators} \\[0.5em]
  & \Uw \wf{\Gamma} \wf{A} \sim \model{A} &&
   := \model{A} = \model{\U}
  \\
  & \Elw \wf{\Gamma} \wf{a} \sim \model{A}
  && :=
  \sum_{\model{a}} (\wf{a}\sim  \model{a}) \times
    (\model{A} = \model{\El}\,\model{a})
  \\
  & \rlap{(3) Parameters} \\[0.5em]
  & \Piw \wf{\Gamma} \wf{a} \wf{B} \sim \model{C} &&
   :=
      \sum_{\model{a}} (\wf{a}\sim\model{a})
      \times
      \sum_{\model{B}} (\wf{B}\sim\model{B})
      \\
      &&& \qquad \times (\model{C}= \model{\Pi}\,\model{a}\,\model{B})
  \\
  & \appw \wf{\Gamma} \wf{a} \wf{B} \wf{t} \wf{u} \sim \\
  & \qquad (\model{\Gamma}\vdash x\in \model{C})  &&
    :=
      \sum_{\model{a}} (\wf{a}\sim\model{a})
      \times
      \sum_{\model{B}} (\wf{B}\sim\model{B})
      \times
      \sum_{\model{t}} (\wf{t}\sim\model{t})
      \times
      \sum_{\model{u}} (\wf{u}\sim\model{u})
      \\
      & &&
     \qquad \times (e_C : \model{C} = \model{B}\model{[0 := \model{u}]})
      \times
      (\model{x} = e_C \# \model{t} \model{\oldapp}\model{u})
    \\
  & \rlap{(4) Metatheoretic parameters} \\[0.5em]
  & \Pimw T\,A\,\wf{\Gamma}\wf{A} \sim \model{B} &&
    :=
      \sum_{\model{A}} ((t : T) \ra \wf{A}\sim\model{A}\, t)
      \times
      (\model{B} = \model{\Pim}\,T\,\model{A})
    \\
    & \appmw T\,A\,\wf{\Gamma}\wf{A} \wf{t} u \sim
    \\
    & \qquad
    (\model{\Gamma}\vdash x\in \model{B})
     &&
     :=
      \sum_{\model{A}} ((t : T) \ra \wf{A}\sim\model{A}\, t)
      \times
      \sum_{\model{t}} (\wf{t}\sim\model{t})
      \times
      \\ & &&
      \qquad
      (e_B : \model{B} = \model{\Pim}\,T\,\model{A})
      \times
      (\model{x} = e_B \# \model{t}\model{\hat{\oldapp}}u)
    \\
  & \rlap{(5) Infinitary parameters} \\[0.5em]
    & \Piiw T\,A\,\wf{\Gamma}\wf{A} \sim \model{B} &&
    :=
      \sum_{\model{A}} ((t : T) \ra \wf{A}\sim\model{A}\, t)
      \times
      (\model{B} = \model{\Pii}\,T\,\model{A})
    \\
    & \appiw T\,A\,\wf{\Gamma}\wf{A} \wf{t} u \sim
    \\
    & \qquad
    (\model{\Gamma}\vdash x\in \model{B})
     &&
     :=
      \sum_{\model{A}} ((t : T) \ra \wf{A}\sim\model{A}\, t)
      \times
      \sum_{\model{t}} (\wf{t}\sim\model{t})
      \times
      \\ & &&
      \qquad
      (e_B : \model{B} = \model{\Pii}\,T\,\model{A})
      \times
      (\model{x} = e_B \# \model{t}\model{\tilde{\oldapp}}u)
\end{alignat*}

As discussed above, when writing the definition of the type components, we assume that the input semantic context is already
related to the typing judgment of the syntactic context.
The first version
of the relation that we suggested at the beginning of the section
would rather enforce them to be related. In the case of $\U$, this would lead
to the definition
  $  \Uw\wf{\Gamma} \sim \model{A} := (\wf{\Gamma}\sim\model{\Gamma}) \times (\model{A} = \model{\U})$,
  instead of
  the current definition
  $  \Uw\wf{\Gamma} \sim \model{A} := (\model{A} = \model{\U})$.
  Our choice makes the definitions more concise, and similarly
 in the definition of the term components we assume that the input semantic
 context and type are already related to their associated well-formedness judgments.
 However, for some fields, we cannot avoid the first approach.
  % Unfortunately, we need sometimes to require some equalities that are actually
  % redundant with this external assumption.
  %, although we know that these equalities are
  % already satisfied by our assumption.
  A typical example is the case of empty substitution $\epsilon$: we require the equality
  $(e_C : \model{E} = \model{\cdot})$, although
  our claimed external assumption that $\model{E}$ is  related to the
  canonical proof $\epsilonw$ of the typing judgment $\epsilon \vdash$ should imply
  it.
  Another example is the first variable case, where we follow the first verbose
  approach:
  the relation gives a semantic context $\model\Gamma$ that must be related
  to the syntactic $\Gamma$, and a semantic
  type $\model{A}$ related to the syntactic $A$. The input semantic context is
  then required to equal $\model{\Gamma}\model{\ext}\model{A}$. These
  relations are needed to show that the relation is right unique as
  $\blank \model{\ext}\blank$ is not guaranteed to be injective.

  It is because of these redundancies that we use UIP to show that the
  types $\sum_{\model{\Gamma}}\Gamma\sim\model{\Gamma}$ and
  $\sum_{\model{A}}A\sim\model{A}$ are propositions.
  One may believe that the use of UIP could be avoided by embracing fully the alternative verbose definition that we suggest above. Then, we expect that
  $\sum_{\model{\Gamma}}\sum_{\model{A}}A\sim\model{A}$, rather than
  $\sum_{\model{A}}A\sim\model{A}$, is a proposition.
  However, there seems
  to be no way of defining $\Elw\wf{\Gamma}\wf{a}\sim \model{A}$ in this fashion so that
  UIP can be avoided to prove this statement.

  \todo[inline]{András: Hugunin's paper has an example (which is
     likely generalizable) that a notion of redundancy in
     well-formedness relations implies that UIP is required.}

  After showing these facts, we prove that the relation is stable under weakening and substitution.
  The last step consists of giving a related semantic counterpart to any
  well-typed context, type or term.
  Everything is done by induction on the typing judgments.
  \subsubsection{Right uniqueness}
  We show by recursion that the relation is right unique in the following sense:
  \begin{alignat*}{5}
    &
    \isp{\Sigma{\sim}}
    && : (\wf\Gamma : \Gamma \vdash ) \ra
    \isaprop\, (\sum_{\model{\Gamma}} \wf{\Gamma} \sim \model{\Gamma})
    \\
    &
    \isp{\Sigma{\sim}}
    && : (\wf A : \Gamma \vdash A) \ra
    \isaprop\, (\sum_{\model{A}} \wf{A} \sim \model{A})
    \\
    &
    \isp{\Sigma{\sim}}
    && : (\wf t : \Gamma \vdash t \in A) \ra
    \isaprop\, (\sum_{\model{t}} \wf{t} \sim \model{t})
    \\
    &
    \isp{\Sigma{\sim}}
    && : (\wf x : \Gamma \vdash x \invar A) \ra
    \isaprop\, (\sum_{\model{x}} \wf{x} \sim \model{x})
    \\
    &
    \isp{\Sigma{\sim}}
    && : (\wf \sigma : \Gamma \vdash \sigma \Rightarrow \Delta) \ra
    \isaprop\, (\sum_{\model{\sigma}} \wf{\sigma} \sim \model{\sigma})
  \end{alignat*}
  \subsubsection{Stability under weakening}
  Let us argue that it is necessary to show stability of the relation under
  weakening before showing that it is stable under substitution.
  Indeed, in the proof of stability under substitution, the $\Pi$ case
  requires to show that $\Pi \,A[\sigma] \,B[\keep\,\sigma]$ is related to
  $\model{\Pi}\,\model{A}\model{[\sigma]}
  \model{A}\model{[\model{\keep}\,\sigma]}$.
  We would like to apply the induction hypothesis, so we need to show that
  $\keep\,\sigma=\varu\,0\consu\wkO\,\sigma$ is related to
  $\model{\keep}\,\model{\sigma}$, knowing that $\sigma$ is
  related to $\model{\sigma}$.
  As $\keep\,\sigma=\varu 0\consu \wkO\,\sigma$, we are left with showing that
  $\wkO\,\sigma=\sigma\circ\wk$ (where $\wk=\wkO \id$)
  relates to its semantic counterpart.

  To do that, we indeed show that $\wkO$ preserves the relation, for types and terms.
  This requires to generalize a bit and show that $\wk_n$ preserves the relation,
  as $\wk_0(\Pi\,A\,B)=\Pi\,(\wk_0\,A)(\wk_1\,B)$.
  But remember that $\wk_n$ performs a weakening in the middle of a context, so
  we first define the semantic counterpart of this:
%   {-

% Suppose that Γ ^^ Δ ⊢ and Γ ⊢ E
% The following function computes both Γ ▶ E ^^ wk_E Δ and a substitution
% from this context to Γ ^^ Δ.
% I don't see how to avoid constructing these two components simultaneously

% -}
% ΣwkTel⇒ᵐ :
%   ∀ {Γ}{Γw :  Γ ⊢}(Γm : ∃ (Con~ Γw) )
%     (Em : M.Ty (₁ Γm))
%         {Δ }{Δw : Γ ^^ Δ ⊢}(Δm : ∃ (Con~ Δw)) →
        % ∃ λ T → M.Sub T (₁ Δm)

\begin{alignat*}{5}
  & \model{\Sigma\wkO{\Rightarrow}}  && :
   % & && \qquad
  (\wf{\Gamma} : \Gamma \vdash) \ra
  (\wf\Gamma \sim \model{\Gamma}) \ra
  \\
  & && \qquad
  (\wf{\Delta} : \Gamma\merge\Delta \vdash) \ra
  (\wf\Delta \sim \model{\Delta}) \ra
  \\ &&& \qquad
  (\model{A} : \model\Ty\model\Gamma)\ra
  (\model{\Delta'} : \model{\Con}) \times (\model{\Sub}\model{\Delta'}{\model{\Delta}})
\end{alignat*}
Here, $\model{\Delta'}$ should be thought of as the context $\model\Delta$ where
the weakening has happened in the middle of the context, by inserting the type
$\model{A}$ after the prefix $\model{\Gamma}$. Indeed, we expect that
$\model{\Gamma}$ is a prefix of $\model{\Delta}$, as $\model{\Gamma}$ relates to
$\Gamma$ and $\model{\Delta}$ to $\Gamma\merge\Delta$.
% $\wf{\Gamma}\sim\model\Gamma$
% and $\wf\Delta\sim\model\Delta$, and $\Gamma$ is a prefix of $\Gamma\merge\Delta$.
The substitution from the weakened context to the original one must
be computed at the same time otherwise the recursion hypothesis is not strong enough.
Then, we seperate the two components under the same (implicit) hypotheses:
\begin{alignat*}{5}
  & \model{\wkO}\,\model{A}\,\model\Delta  && :
  % (\model{A} : \model\Ty\model\Gamma)\ra
  % (\wf{\Gamma} : \Gamma \vdash) \ra
  % (\wf{\Delta} : \Gamma\merge\Delta \vdash) \ra
  % \\
  % & && \qquad
  % (\wf\Gamma \sim \model{\Gamma}) \ra
  % (\wf\Delta \sim \model{\Delta}) \ra
  % \\ &&& \qquad
   \model{\Con}
   \\
  & \model{\wk{\Rightarrow}}\,\model{A}\,\model\Delta  && :
  % (\model{A} : \model\Ty\model\Gamma)\ra
  % (\wf{\Gamma} : \Gamma \vdash) \ra
  % (\wf{\Delta} : \Gamma\merge\Delta \vdash) \ra
  % \\
  % & && \qquad
  % (\wf\Gamma \sim \model{\Gamma}) \ra
  % (\wf\Delta \sim \model{\Delta}) \ra
  % \\ &&& \qquad
   \model{\Sub}(\model{\wkO}\model{A}\,\model\Delta)\model\Delta
\end{alignat*}
Now, we are ready to prove by mutual recursion on well-typed judgments that
weakening preserves typing. The following statements are all under the
hypotheses
  $(\wf{\Gamma} : \Gamma \vdash)$,
  $(\wf\Gamma \sim \model{\Gamma})$,
  $(\wf{\Delta} : \Gamma\merge\Delta \vdash)$,
  $(\wf\Delta \sim \model{\Delta})$,
  $(\wf{A} : \Gamma \vdash A)$,
  and
  $(\wf{A} \sim \model{A})$.
  % , except the last one about weakening substitutions,
  % which do not require that $\model\Delta$ is related to any well-formed context.
\begin{alignat*}{5}
  & \wkO{\sim}  &&
   :
  % (\wf{\Gamma} : \Gamma \vdash) \ra
  % (\wf\Gamma \sim \model{\Gamma}) \ra
  % \\
  %  & && \qquad
  % (\wf{\Delta} : \Gamma\merge\Delta \vdash) \ra
  % (\wf\Delta \sim \model{\Delta}) \ra
  % \\
  % & && \qquad
  % (\wf{A} : \Gamma \vdash A) \ra
  % (\wf{A} \sim \model{A}) \ra
  % \\
  % & && \qquad
  \wf{\wkO}\,\wf{A}\,\wf{\Delta}\sim\model{\wkO}\model{A}\model{\Delta}
  \\
  % wk
  & \wk{\sim} && :
  % (\wf{\Gamma} : \Gamma \vdash) \ra
  % (\wf\Gamma \sim \model{\Gamma}) \ra
  % \\
  %  & && \qquad
  % (\wf{\Delta} : \Gamma\merge\Delta \vdash) \ra
  % (\wf\Delta \sim \model{\Delta}) \ra
  % \\
  % & && \qquad
  % (\wf{A} : \Gamma \vdash A) \ra
  % (\wf{A} \sim \model{A}) \ra
  % \\
  % &&& \qquad
  (\wf{T} : \Gamma\merge\Delta \vdash T) \ra
  (\wf{T} \sim \model{T}) \ra
  % \\ &&& \qquad
  \wf{\wk}\,\wf{A}\,\wf{T}\sim\model{T}\model{[\model{\wkO{\Rightarrow}}\model{A}\model{\Delta}]}
  \\
    % wk
  & \wk{\sim} && :
  % (\wf{\Gamma} : \Gamma \vdash) \ra
  % (\wf\Gamma \sim \model{\Gamma}) \ra
  % \\
  %  & && \qquad
  % (\wf{\Delta} : \Gamma\merge\Delta \vdash) \ra
  % (\wf\Delta \sim \model{\Delta}) \ra
  % \\
  % & && \qquad
  % (\wf{A} : \Gamma \vdash A) \ra
  % (\wf{A} \sim \model{A}) \ra
  % \\
  % &&& \qquad
  (\wf{t} : \Gamma\merge\Delta \vdash t\in T) \ra
  (\wf{t} \sim \model{t}) \ra
  % \\ &&& \qquad
  \wf{\wk}\,\wf{A}\,\wf{t}\sim\model{t}\model{[\model{\wkO{\Rightarrow}}\model{A}\model{\Delta}]}
  \\
    % wk
  & \wk{\sim} && :
  % (\wf{\Gamma} : \Gamma \vdash) \ra
  % (\wf\Gamma \sim \model{\Gamma}) \ra
  % \\
  %  & && \qquad
  % (\wf{\Delta} : \Gamma\merge\Delta \vdash) \ra
  % (\wf\Delta \sim \model{\Delta}) \ra
  % \\
  % & && \qquad
  % (\wf{A} : \Gamma \vdash A) \ra
  % (\wf{A} \sim \model{A}) \ra
  % \\
  % &&& \qquad
  (\wf{x} : \Gamma\merge\Delta \vdash t\invar T) \ra
  (\wf{x} \sim \model{x}) \ra
  % \\ &&& \qquad
  \wf{\wk}\,\wf{A}\,\wf{x}\sim\model{x}\model{[\model{\wkO{\Rightarrow}}\model{A}\model{\Delta}]}
  % \\ &&& \qquad
  % \wf{\wk}\,\wf{A}\,\wf{x}\sim\model{x}\model{[\model{\wkO{\Rightarrow}}\model{A}\model{\Delta}]}
\end{alignat*}
Then we deduce, still by recursion, that weakening of substitution preserves the
relation:
\begin{alignat*}{5}
    % wk
  & \wkO{\sim} && :
  (\wf{\Gamma} : \Gamma \vdash) \ra
  (\wf\Gamma \sim \model{\Gamma}) \ra
  % \\
  %  & && \qquad
  % (\wf{\Delta} : \Gamma\merge\Delta \vdash) \ra
  % (\wf\Delta \sim \model{\Delta}) \ra
  % \\
  % & && \qquad
  (\wf{A} : \Gamma \vdash A) \ra
  (\wf{A} \sim \model{A}) \ra
  \\
  &&& \qquad
  (\wf{\sigma} : \Gamma\vdash \sigma \Rightarrow \Delta) \ra
  (\wf{\sigma} \sim \model{\sigma}) \ra
  % \\
  % &&& \qquad
  (\wf{\wkO}\wf{A}\wf\sigma \sim \model{\sigma}\model{\circ}\model{\wk})
  \end{alignat*}



%   But to give a precise meaning to the statement that $\wk_n$ (which weakens at
%   the middle of context) preserve the
%   relation, we define (inductively) the notion of semantic telescopes.
%   First, we define the notion of being a prefix:
% \begin{alignat*}{5}
%   & \blank\leq\blank  && : \model\Con\ra \model\Con\ra\Set \\
%   & {\leq}\cdot  && : \model{\Gamma} \leq \model{\Gamma} \\
%   & \blank{\leq}{\ext}\blank  && : \model{\Gamma} \leq \model{\Delta} \ra (A :
%   \model{\Ty}\Delta) \ra \model{\Gamma}\leq\model{\Delta}\model{\ext}\model{A}
% \end{alignat*}
% Now, a telescope on a semantic context $\model{\Gamma}$ is a semantic context
% $\model{\Delta}$ such that $\model{\Gamma}\leq \model{\Delta}$.

% \begin{alignat*}{5}
%   & \Tel\,\model{\Gamma}  && := (\model{\Delta} : \model{\Con})\times (\model{\Gamma}\leq \model{\Delta})
% \end{alignat*}
% Similarly to the untyped case, we define the merging of a context and a telescope by recursion on the telescope:
% \begin{alignat*}{5}
%   & \blank\model{\merge}\blank  && := (\model{\Gamma} : \model{\Con}) \ra
%   (\model{\Delta} : \Tel\,\model\Gamma) \ra \model{\Con} \\
%   & \model\Gamma \merge (\model\Gamma,{\leq}\cdot) && :=  \model\Gamma \\
%   & \model\Gamma \merge (\model\Delta\model{\ext}\model{A}, \Delta_{\leq} {\leq}{\ext} \model{A}) && :=  (\model\Gamma \model\merge \model\Delta)\model\ext \model{A}\\
% \end{alignat*}
% We then define a relevant relation between syntactic telescopes and semantic telescopes.
% Untyped syntactic telescopes are modeled by untyped contexts, because they are
% list of types.
% A syntactic telescope $\Delta$ is then well formed in a context $\Gamma$ if
% $\Gamma\merge\Delta$ is well formed.
% The relation is defined by recursion on the untyped syntactic telescope:
% \begin{alignat*}{5}
%   & \blank\sim\blank  && := ( \Gamma \merge \Delta \vdash) \ra (\model\Gamma\leq\model\Delta)
%   \ra \Set \\
%   & (\wf\Gamma : \Gamma \merge \emptyu \vdash)\sim\model\Delta  && :=
%   \model\Delta =  {\leq}\cdot\\
%   & (\wf\Delta \extw \wf{A} : \Gamma \merge (\Delta \extu A) \vdash)\sim\model{E}  && :=
%   \sum_{\model\Delta} (\wf\Delta\sim\model\Delta)
%   \times
%   \sum_{\model A} (\wf{A}\sim\model{A}) \times
%   \\
%   & && \qquad
%   \model{E} = \model\Delta {\leq}{\ext} \model{A}
% \end{alignat*}

\subsubsection{Stability under substitution}
Let us see an example of where stability under substitution is used in
the recursion building a related semantic counterpart to the syntax.
 We want to prove that given any well-typed
substitution from a context $\Gamma$ to a context $\Delta$, any semantic
contexts $\model{\Gamma}$ and $\model{\Delta}$ which are related to $\Gamma$ and
$\Delta$, there exist a semantic substitution which is related to the syntactic one.
In the extension case $\Gamma\vdash \sigma \consu t \Rightarrow \Delta \extu A$, the
induction hypothesis provides $\model{\sigma}$, $\model{\Delta}$, $\model{A}$ related to
their syntactic counterpart. However, the premises of the induction hypothesis
for getting a relevant $\model{t}$
require to show that the type $\model{A}\model{[\model{\sigma}]}$ is
related to the syntactic type $A[\sigma]$.

We first establish preservation of the relation by substitution for variables:
\begin{alignat*}{5}
    % []V~
  & []{\sim} && :
  (\wf{\sigma} : \Gamma \vdash \sigma \Rightarrow \Delta) \ra
  (\wf\sigma \sim \model{\sigma}) \ra
  (\wf{x} : \Delta \vdash x \invar A) \ra
  (\wf{x} \sim \model{x}) \ra
  \\ & && \qquad
  % \\
  % & && \qquad
   \wf{[]}\wf{x}\wf{\sigma} \sim \model{x}\model{[\model{\sigma}]}
\end{alignat*}
Then we show it for terms and types  by mutual recursion
under the common hypotheses
  $(\wf{\sigma} : \Gamma \vdash \sigma \Rightarrow \Delta)$,
  $(\wf\sigma \sim \model{\sigma})$,
  $(\wf{\Gamma} : \Gamma \vdash )$,
  $(\wf{\Gamma} \sim \model{\Gamma})$,
  % \\
  % &&&\qquad
    $(\wf{\Delta} : \Delta \vdash )$,
    and
  $(\wf{\Delta} \sim \model{\Delta})$:
\begin{alignat*}{5}
     & []{\sim} && :
    (\wf{A} : \Delta \vdash A) \ra
  (\wf{A} \sim \model{A}) \ra
  % \\
  % & && \qquad
   \wf{[]}\wf{\Gamma}\wf{A}\wf{\sigma} \sim \model{A}\model{[\model{\sigma}]}
   \\
     & []{\sim} && :
    (\wf{t} : \Delta \vdash t \in A ) \ra
  (\wf{t} \sim \model{t}) \ra
  % \\
  % & && \qquad
   \wf{[]}\wf{\Gamma}\wf{t}\wf{\sigma} \sim \model{t}\model{[\model{\sigma}]}
  \end{alignat*}
  Eventually, we show under the same hypothese the following statement:
  \begin{alignat*}{5}
     & {\circ}{\sim} && :
    (\wf{E} : E \vdash ) \ra
  (\wf{E} \sim \model{E}) \ra
  % \\
  % &&&  \qquad
    (\wf{\delta} : \Delta \vdash \delta \Rightarrow E) \ra
  (\wf{\delta} \sim \model{\delta}) \ra
  \\
  & && \qquad
   \wf{{\circ}}\wf{\Gamma}\wf{\delta}\wf{\sigma} \sim \model{\delta}\model{\circ}\model{\sigma}
  \end{alignat*}
  and the fact that identity preserves the relation:
  \begin{alignat*}{5}
     & {\id}{\sim} && :
    (\wf{\Gamma} : \Gamma \vdash ) \ra
  (\wf{\Gamma} \sim \model{\Gamma}) \ra
   \wf{{\id}}\wf{\Gamma} \sim {\id}_{\model{\Gamma}}
  \end{alignat*}
% $\model{t}\model\oldapp\model{u}$, but it has type
% $\model{B}\model{[0 := \model{u}]}$
% instead of type $\model{T}$. Fortunately, we know by hypothesis that $\model{T}$
% relates to $T=B[0:=u]$ so, by uniqueness of the related semantic counterpart,
% we can deduce that $\model{T}=\m$
% Thus, we need to show that $\model{T}=\model{B}\model{[0 := \model{u}]}$ if we
% know that this last type is related to $B[0 := u]$.


\subsection{Existence and uniqueness of the initial QIIT morphism}
Induction on the typing judgments shows that any QIIT morphism from the
syntactic model $S$  to a semantic model $M$ sends well-typed syntax on a related
semantic counter-part. More formally, we show by recursion:
  \begin{alignat*}{5}
     & \Sigma\Con{\sim} && :
    (\wf{\Gamma} : \Gamma \vdash ) \ra
    \sum_{\model{\Gamma}} \wf{\Gamma} \sim \model{\Gamma}
    \\
     & \Sigma\Ty{\sim} && :
    (\wf{\Gamma} : \Gamma \vdash ) \ra
    (\wf{\Gamma}\sim \model\Gamma) \ra
    (\wf{A} : \Gamma \vdash A ) \ra
    ({\model{A}:\model{\Ty}\model\Gamma})\times ( \wf{A} \sim \model{A})
    \\
     & \Sigma\Tm{\sim} && :
    (\wf{\Gamma} : \Gamma \vdash ) \ra
    (\wf{\Gamma}\sim \model\Gamma) \ra
    (\wf{A} : \Gamma \vdash A ) \ra
    ( \wf{A} \sim \model{A}) \ra
    \\ & && \qquad
    (\wf{t} : \Gamma \vdash t \in A ) \ra
    ({\model{t}:\model{\Tm}\model\Gamma}\model{A})\times ( \wf{t} \sim \model{t})
    \\
     & \Sigma\Var{\sim} && :
    (\wf{\Gamma} : \Gamma \vdash ) \ra
    (\wf{\Gamma}\sim \model\Gamma) \ra
    (\wf{A} : \Gamma \vdash A ) \ra
    ( \wf{A} \sim \model{A}) \ra
    \\ & && \qquad
    (\wf{x} : \Gamma \vdash x \invar A ) \ra
    ({\model{x}:\model{\Tm}\model\Gamma}\model{A})\times ( \wf{x} \sim \model{x})
    \\
     & \Sigma\Sub{\sim} && :
    (\wf{\Gamma} : \Gamma \vdash ) \ra
    (\wf{\Gamma}\sim \model\Gamma) \ra
    (\wf{\Delta} : \Delta \vdash  ) \ra
    ( \wf{\Delta} \sim \model{\Delta}) \ra
    \\ & && \qquad
    (\wf{\sigma} : \Gamma \vdash \sigma \Rightarrow \Delta ) \ra
    ({\model{\sigma}:\model{\Sub}\model\Gamma}\model{\Delta})\times ( \wf{\sigma} \sim \model{\sigma})
  \end{alignat*}
  Right uniqueness of the relation is used in this recursion.

It is then straightforward to show that the first projection of these
constructions yield a model morphism from the syntax to the model, using right
uniqueness of the relation.
Now, to show that it is unique, suppose that we have a morphism from the
syntax to the model, inducing the following maps:
\begin{alignat*}{5}
  &
  \mor{\Con}
  && :
   (\Gamma \vdash) \ra \model\Con
   \\
  &
  \mor{\Ty}
  && :
   (\wf\Gamma:\Gamma \vdash) \ra (\Gamma\vdash A)\ra\model\Ty\,(\mor\Con\wf\Gamma)
   \\
  &
  \mor{\Tm}
  && :
  (\wf\Gamma:\Gamma \vdash) \ra (\wf{A}:\Gamma\vdash A)\ra
  (\Gamma\vdash t \in A)\ra\model\Tm\,(\mor\Con\wf\Gamma)\,
  (\mor\Ty\wf\Gamma\,\wf{A})
   \\
  &
  \mor{\Var}
  && :
  (\wf\Gamma:\Gamma \vdash) \ra (\wf{A}:\Gamma\vdash A)\ra
  (\Gamma\vdash x \invar A)\ra\model\Tm\,(\mor\Con\wf\Gamma)\,
  (\mor\Ty\wf\Gamma\,\wf{A})
   \\
  &
  \mor{\Sub}
  && :
  (\wf\Gamma:\Gamma \vdash) \ra
  (\wf\Delta:\Delta \vdash) \ra
  (\Gamma\vdash \sigma \Rightarrow \Delta)\ra\model\Sub\,(\mor\Con\wf\Gamma)\,
  (\mor\Con\wf\Delta)
\end{alignat*}
Then, we show by recursion on the typing judgments that
the image is related:
\begin{alignat*}{5}
  &
  {\sim}\mor{\Con}
  && :
  (\wf\Gamma : \Gamma \vdash) \ra \wf\Gamma\sim \mor\Con\,\wf\Gamma
  \\
  &
  {\sim}\mor{\Ty}
  && :
  (\wf\Gamma : \Gamma \vdash) \ra
  (\wf{A} : \Gamma \vdash A) \ra
  \wf\Gamma\sim \mor\Ty\,\wf\Gamma\,\wf{A}
  \\
  &
  {\sim}\mor{\Tm}
  && :
  (\wf\Gamma : \Gamma \vdash) \ra
  (\wf{A} : \Gamma \vdash A) \ra
  (\wf{t} : \Gamma \vdash t \in A) \ra
  \wf\Gamma\sim \mor\Tm\,\wf\Gamma\,\wf{A}\,\wf{t}
  \\
  &
  {\sim}\mor{\Var}
  && :
  (\wf\Gamma : \Gamma \vdash) \ra
  (\wf{A} : \Gamma \vdash A) \ra
  (\wf{x} : \Gamma \vdash x \invar A) \ra
  \wf\Gamma\sim \mor\Var\,\wf\Gamma\,\wf{A}\,\wf{x}
  \\
  &
  {\sim}\mor{\Sub}
  && :
  (\wf\Gamma : \Gamma \vdash) \ra
  (\wf{\Delta} : \Delta \vdash) \ra
  (\wf{\sigma} : \Gamma \vdash \sigma \Rightarrow \Delta) \ra
  \wf\Gamma\sim \mor\Sub\,\wf\Gamma\,\wf{\Delta}\,\wf{\sigma}
\end{alignat*}
This justifies the uniqueness of the morphism, by right uniqueness of the relation.




% \begin{description}
%   \item[untyped syntax and well typed judgments as an algebra]
%   \begin{enumerate}
%   \item define untyped syntax as an inductive datatype
%   \item define operations on the syntax, such as substitution, by recursion
%   \item define well-typed judgments as an inductive datatype indexed over the untyped syntax
%   \item show the dependant pair of untyped syntax with well-typed judgments
%     define an algebra
%   \end{enumerate}
%   \item[specification of the initial morphism as a functional relation]
%   \begin{enumerate}
%   \item given an algebra, define the functional relation enjoyed by the recursor
%     from the syntax to the algebra by recursion over the well-typed judgment
%   \item show right-uniqueness of the relation
%   \item show left-totality of the relation
%   \end{enumerate}
%   \item[existence and uniqueness of the morphism from the syntax to a model]
%     \begin{enumerate}
%   \item show uniqueness of such a morphism
%   \item extract an algebra morphism from the relation
%     \end{enumerate}
% \end{description}
