
\documentclass[a4paper,UKenglish,cleveref, autoref]{lipics-v2019}
%This is a template for producing LIPIcs articles.
%See lipics-manual.pdf for further information.
%for A4 paper format use option "a4paper", for US-letter use option "letterpaper"
%for british hyphenation rules use option "UKenglish", for american hyphenation rules use option "USenglish"
%for section-numbered lemmas etc., use "numberwithinsect"
%for enabling cleveref support, use "cleveref"
%for enabling cleveref support, use "autoref"
\usepackage{amssymb}
\usepackage{amsmath}
\usepackage{hyperref}
\usepackage{todonotes}
\presetkeys{todonotes}{inline}{}

%\graphicspath{{./graphics/}}%helpful if your graphic files are in another directory

\bibliographystyle{plainurl}% the mandatory bibstyle

\title{For Induction-Induction, Induction is Enough} %TODO Please add

%% \titlerunning{For Induction-Induction, Induction is Enough}%optional, please use if title is longer than one line

\author{Ambrus Kaposi}{E{\"o}tv{\"o}s Lor{\'a}nd University, Budapest, Hungary}{akaposi@inf.elte.hu}{https://orcid.org/0000-0001-9897-8936}{this author was supported by Thematic Excellence Programme, Industry and Digitization Subprogramme (NRDI Office, 2019) and by the European Union, co-financed by the European Social Fund (EFOP-3.6.2-16-2017-00013, Thematic Fundamental Research Collaborations Grounding Innovation in Informatics and Infocommunication).}%TODO mandatory, please use full name; only 1 author per \author macro; first two parameters are mandatory, other parameters can be empty. Please provide at least the name of the affiliation and the country. The full address is optional
\author{Andr{\'a}s Kov{\'a}cs}{E{\"o}tv{\"o}s Lor{\'a}nd University, Budapest, Hungary}{kovacsandras@inf.elte.hu}{https://orcid.org/0000-0002-6375-9781}{this author was supported by the European Union, co-financed by the European Social Fund (EFOP-3.6.3-VEKOP-16-2017-00002).}
\author{Ambroise Lafont}{IMT Atlantique, Inria, LS2N CNRS, Nantes, France}{ambroise.lafont@inria.fr}{https://orcid.org/0000-0002-9299-641X}{}

\authorrunning{A. Kaposi, A. Kov{\'a}cs and A. Lafont}%TODO mandatory. First: Use abbreviated first/middle names. Second (only in severe cases): Use first author plus 'et al.'

\Copyright{A. Kaposi, A. Kov{\'a}cs and A. Lafont}%TODO mandatory, please use full first names. LIPIcs license is "CC-BY";  http://creativecommons.org/licenses/by/3.0/

\ccsdesc[500]{Theory of computation~Logic~Type theory}%TODO mandatory: Please choose ACM 2012 classifications from https://dl.acm.org/ccs/ccs_flat.cfm

\keywords{type theory, inductive types, inductive-inductive types}%TODO mandatory; please add comma-separated list of keywords

\category{}%optional, e.g. invited paper

\relatedversion{}%optional, e.g. full version hosted on arXiv, HAL, or other respository/website
%\relatedversion{A full version of the paper is available at \url{...}.}

\supplement{}%optional, e.g. related research data, source code, ... hosted on a repository like zenodo, figshare, GitHub, ...

%\funding{(Optional) general funding statement \dots}%optional, to capture a funding statement, which applies to all authors. Please enter author specific funding statements as fifth argument of the \author macro.

\acknowledgements{The authors would like to thank Thorsten Altenkirch, Simon Boulier, Fredrik Nordvall-Forsberg and Jakob von Raumer for discussions on the topics of this paper.}%optional

%\nolinenumbers %uncomment to disable line numbering

%\hideLIPIcs  %uncomment to remove references to LIPIcs series (logo, DOI, ...), e.g. when preparing a pre-final version to be uploaded to arXiv or another public repository

%Editor-only macros:: begin (do not touch as author)%%%%%%%%%%%%%%%%%%%%%%%%%%%%%%%%%%
\EventEditors{John Q. Open and Joan R. Access}
\EventNoEds{2}
\EventLongTitle{42nd Conference on Very Important Topics (CVIT 2016)}
\EventShortTitle{CVIT 2016}
\EventAcronym{CVIT}
\EventYear{2016}
\EventDate{December 24--27, 2016}
\EventLocation{Little Whinging, United Kingdom}
\EventLogo{}
\SeriesVolume{42}
\ArticleNo{23}
%%%%%%%%%%%%%%%%%%%%%%%%%%%%%%%%%%%%%%%%%%%%%%%%%%%%%%

%ppp metatheory
\newcommand{\blank}{\mathord{\hspace{1pt}\text{--}\hspace{1pt}}} %from the book
\newcommand{\ra}{\rightarrow}
\newcommand{\Set}{\mathsf{Set}}
\newcommand{\Prop}{\mathsf{Prop}}

% object theory: universal QIIT
\newcommand{\Con}{\mathsf{Con}}
\newcommand{\Ty}{\mathsf{Ty}}
\newcommand{\Sub}{\mathsf{Sub}}
\newcommand{\Tm}{\mathsf{Tm}}
\newcommand{\Var}{\mathsf{Var}}
\newcommand{\id}{\mathsf{id}}
\newcommand{\ass}{\mathsf{ass}}
\newcommand{\idl}{\mathsf{idl}}
\newcommand{\idr}{\mathsf{idr}}
\newcommand{\ext}{\rhd}
\newcommand{\p}{\mathsf{p}}
\newcommand{\q}{\mathsf{q}}
\newcommand{\U}{\mathsf{U}}
\newcommand{\El}{\mathsf{El}}
\newcommand{\app}{\mathsf{app}}
\newcommand{\oldapp}{\mathop{{\scriptstyle @}}}
\newcommand{\Pim}{\hat{\Pi}}
\newcommand{\appm}{\mathop{\hat{{\scriptstyle @}}}}
\newcommand{\Pii}{\tilde{\Pi}}
\newcommand{\appi}{\mathop{\tilde{{\scriptstyle @}}}}
\newcommand{\Id}{\mathsf{Id}}
\newcommand{\reflect}{\mathsf{reflect}}

% untyped syntax
\newcommand{\untyp}[1]{{#1}^{\mathsf{p}}}
\newcommand{\Conu}{\untyp{\Con}}
\newcommand{\Tyu}{\untyp{\Ty}}
\newcommand{\Subu}{\untyp{\Sub}}
\newcommand{\Tmu}{\untyp{\Tm}}
\newcommand{\idu}{\untyp{\id}}
\newcommand{\emptyu}{\untyp{\cdot}}
\newcommand{\epsilonu}{\untyp{\epsilon}}
\newcommand{\extu}{\mathbin{\untyp{\ext}}}
\newcommand{\consu}{\mathbin{\untyp{\cons}}}
\newcommand{\varu}{\untyp{\mathsf{var}}}
\newcommand{\Uu}{\untyp{\U}}
\newcommand{\Elu}{\untyp{\El}}
\newcommand{\Piu}{\untyp{\Pi}}
\newcommand{\appu}{\mathbin{\untyp{{\scriptstyle @}}}}
\newcommand{\Pipu}{\untyp{\Pi}}
\newcommand{\Pimu}{\untyp{\Pim}}
\newcommand{\Piiu}{\untyp{\Pii}}
\newcommand{\appmu}{\mathop{\hat{\tilde{{\scriptstyle @}}}}}
\newcommand{\erru}{\untyp{\mathsf{err}}}
\newcommand{\wku}{\untyp{\wk}}
\newcommand{\idru}{\untyp{\idr}}
\newcommand{\idlu}{\untyp{\idl}}

% well-typed judgement
\newcommand{\wf}[1]{{#1}^{\mathsf{w}}}
\newcommand{\Conw}{\wf{\Con}}
\newcommand{\Tyw}{\wf{\Ty}}
\newcommand{\Subw}{\wf{\Sub}}
\newcommand{\Tmw}{\wf{\Tm}}
\newcommand{\invar}{\in_\N}
\newcommand{\idw}{\wf{\id}}
\newcommand{\emptyw}{\wf{\cdot}}
\newcommand{\extw}{\mathbin{\wf{\ext}}}
\newcommand{\consw}{\wf{\cons}}
\newcommand{\epsilonw}{\wf{\epsilon}}
\newcommand{\varw}{\wf{\mathsf{var}}}
\newcommand{\vzw}{\wf{\mathsf{0}}}
\newcommand{\vsw}{\wf{\mathsf{S}}}
\newcommand{\Uw}{\wf{\U}}
\newcommand{\Elw}{\wf{\El}}
\newcommand{\Piw}{\wf{\Pi}}
\newcommand{\appw}{\wf{\app}}
\newcommand{\Pipw}{\wf{\Pi}}
\newcommand{\Pimw}{\wf{\Pim}}
\newcommand{\appmw}{\wf{\hat{\app}}}
\newcommand{\Piiw}{\wf{\Pii}}
\newcommand{\appiw}{\wf{\tilde{\app}}}

% model
\newcommand{\model}[1]{{#1}^M}

% morphism of model
\newcommand{\mor}[1]{{#1}^f}

% syntax model
\newcommand{\syn}[1]{{#1}^{\mathsf{S}}}

% functional relation
\newcommand{\rel}[1]{{#1}^r}
\newcommand{\Conr}{\rel{\Con}}
\newcommand{\Tyr}{\rel{\Ty}}
\newcommand{\Subr}{\rel{\Sub}}
\newcommand{\Tmr}{\rel{\Tm}}
\newcommand{\emptyr}{\rel{\cdot}}
\newcommand{\extr}{\rel{\ext}}
\newcommand{\varr}{\rel{\mathsf{var}}}
\newcommand{\vzr}{\rel{0}}
\newcommand{\vsr}{\rel{S}}
\newcommand{\Ur}{\rel{\U}}
\newcommand{\Elr}{\rel{\El}}
\newcommand{\Pir}{\rel{\Pi}}
\newcommand{\appr}{\rel{\app}}
\newcommand{\Pipr}{\rel{\Pi}}
\newcommand{\Pimr}{\rel{\Pim}}
\newcommand{\appmr}{\rel{\hat{\app}}}
\newcommand{\Piir}{\rel{\Pii}}
\newcommand{\appir}{\rel{\tilde{\app}}}

\newcommand{\relT}[3]{#1 \sim_{#2} #3}
\newcommand{\relt}[4]{#1 \sim_{#2\vdash #3} #4}
\newcommand{\relV}[4]{#1 \sim_{#2\vdash #3} #4}
\newcommand{\relS}[4]{#1 \sim_{#2\Rightarrow #3} #4}




\newcommand{\A}{\mathsf{A}}
\newcommand{\F}{\mathsf{F}}
\renewcommand{\S}{\mathsf{S}}
\renewcommand{\O}{\mathsf{O}}
\newcommand{\M}{\mathsf{M}}
\newcommand{\con}{\mathsf{con}}
\newcommand{\elim}{\mathsf{elim}}
\newcommand{\nat}{\mathsf{nat}}
\newcommand{\map}{\mathsf{map}}
\newcommand{\List}{\mathsf{List}}
\newcommand{\lb}{\langle}
\newcommand{\rb}{\rangle}
\newcommand{\wk}{\mathsf{wk}}
\newcommand{\wkO}{\wk_0}
\newcommand{\keep}{\mathsf{keep}}
\newcommand{\cons}{,}
\newcommand{\nil}{\mathsf{nil}}
\renewcommand{\merge}{;}
\newcommand{\length}[1]{\| #1 \|}
\newcommand{\vz}{\mathsf{vz}}
\newcommand{\vs}{\mathsf{vs}}
\newcommand{\Ra}{\Rightarrow}
\renewcommand{\tt}{\mathsf{tt}}
\newcommand{\proj}{\mathsf{proj}}
\newcommand{\refl}{\mathsf{refl}}
\newcommand{\J}{\mathsf{J}}
\newcommand{\tr}{\mathsf{tr}}
\newcommand{\trans}{\mathbin{\raisebox{0.2ex}{$\displaystyle\centerdot$}}}
\newcommand{\ap}{\mathsf{ap}}
\newcommand{\apd}{\mathsf{apd}}
\newcommand{\rec}{\mathsf{rec}}
\newcommand{\C}{\mathsf{C}}
\newcommand{\R}{\mathsf{R}}
\newcommand{\E}{\mathsf{E}}
\newcommand{\transp}{\mathsf{transp}}
\newcommand{\Ram}{\mathbin{\hat{\Ra}}}
\newcommand{\funext}{\mathsf{funext}}
\newcommand{\UIP}{\mathsf{UIP}}
\newcommand{\coe}{\mathsf{coe}}
\newcommand{\LET}{\mathsf{let}}
\newcommand{\IN}{\mathsf{in}}
\newcommand{\N}{\mathbb{N}}
\newcommand{\D}{\mathsf{D}}
\newcommand{\K}{\mathsf{K}}
\newcommand{\Eq}{\mathsf{Eq}}
\newcommand{\mk}{\mathsf{mk}}
\newcommand{\unk}{\mathsf{unk}}
\newcommand{\0}{\mathit{0}}
\newcommand{\1}{\mathit{1}}
\newcommand{\eqreflect}{\mathsf{eqreflect}}

\newcommand{\isaprop}{\mathsf{is\text{-}prop}}
\newcommand{\isp}[1]{{#1}^{\mathsf{p}}}

\mathchardef\mhyphen="2D

\renewcommand{\ll}{\llbracket}
\newcommand{\rr}{\rrbracket}

\newcommand{\SignAlg}{\mathsf{SignAlg}}
\newcommand{\SignMor}{\mathsf{SignMor}}
\newcommand{\Sign}{\mathsf{Sign}}
\newcommand{\ADS}{\mathsf{ADS}}
\newcommand{\Bool}{\mathsf{Bool}}
\newcommand{\I}{\mathsf{I}}
\newcommand{\IE}{\mathsf{IE}}



\allowdisplaybreaks

\begin{document}
\maketitle

\begin{abstract}
  Inductive-inductive types (IITs) are a generalisation of inductive types in
  type theory. They allow the mutual definition of types with multiple sorts
  where later sorts can be indexed by previous ones. An example is the James
  Chapman-style syntax of type theory with conversion relations for each sort
  where e.g.\ the sort of types is indexed by contexts. It follows from previous
  work that all finitary IITs can be constructed from a quotient
  inductive-inductive type (QIIT), namely the theory of IIT signatures. This is
  a small domain-specific type theory where a context is a signature for an
  IIT. In this paper we construct this theory using only inductive types,
  thereby showing a reduction of all finitary IITs to inductive types.  We rely
  on function extensionality and uniqueness of identity proofs. We formalised
  the construction of the theory of IIT signatures in Agda.

\end{abstract}

\section{Introduction}
\label{sec:intro}

\todo{Note by Ambrus: to be beautified, I only wanted to add the line of
  thought.}

\todo{András: overall todos: introduction, section 2 examples, intro to section 3, future work, maybe changing big Sigma notation to (x : A) x B notation (or changing the other way?)n}

\todo{
  András: construction of infinitary IITs doesn't (yet) work. Should we drop
  infinitary functions from signatures?
  }

Most mutual inductive types (e.g.\ isEven-isOdd) can be reduced to indexed
inductive types where the index gives the information which sort we
meant. (Indexed inductive types can be reduced to inductive types using a method
similar to what we use here, preterms and well-formedness predicate.)
Inductive-inductive types (IITs \cite{forsberg-phd}) allow the mutual definition
of a type and a family of types over the first one. The above reduction does not
work anymore.

Examples of IITs: Con-Ty is the minimal interesting one, then we have
James Chapman style \cite{chapman09eatitself} well-typed syntax for
type theory.

TODO: list examples here and explain. Also the elimination principles
which is recursive-recursive. Talk about connection to setoidified
CwFs (I think these are explained in some Dybjer paper) and maybe the
initiality conjecture.

Reduction of IITs to inductive types is an open problem. Forsberg
showed that IITs can be reduced to inductive types in the setting of
ETT, but he only constructed the nondependent (or simple? was this the
non-recursive-recursive?) elimination principle. Hugunin \cite{jasper}
reduced an example IIT to inductive types without UIP, but also only
constructing the non-recursive-recursive eliminator. Streicher
\cite{streichersemantics}. Brunerie \cite{} constructed the initial
category with extra structure (TODO: fix this) using UIP.

Kaposi and Kov{\'a}cs defined a type theory of signatures for higher
inductive-inductive types (HIITs)
\cite{kaposi_et_al:LIPIcs:2018:9190}. This is a small type theory in
which every context encodes a valid IIT signature. For example, the
signature for natural numbers is simply the context
$(Nat:\U,zero:Nat,suc:Nat\ra Nat)$ of length three ($Nat$, $zero$ and
$suc$ are simply variable names). Later they showed
\cite{Kaposi:2019:CQI:3302515.3290315} that if a subset of this theory
of signatures exists, then any finitary QIIT can be defined from
it. In this paper we further restrict the subset by not allowing
equality constructors (thus only covering IITs) but at the same time
we extend it to allow infinitary constructors. To derive the existence
of all IITs from the theory of IIT signatures we use a variant of the
CwF model \cite{Kaposi:2019:CQI:3302515.3290315} where types are
modelled not by displayed CwFs but by CwF isofibrations. We need this
extension to support infinitary constructors. Furthermore, using the
method introduced by Streicher \cite{streichersemantics} we show the
existence of the theory of IIT signatures. As opposed to Streicher, we
don't use quotients in this construction as the theory of IIT
signatures can be given only using neutral terms.

Our metatheory is ETT. Using Hofmann \cite{}, Theo \cite{} translation
we can translate our arguments to ITT+UIP+funext. Section
\ref{sec:ambroise} was formalised in Agda using UIP, rewrite rules and
funext.

This is the paper version of our TYPES 2019 talk.

\subsection{Related Work}

Reduction of classes of inductive types: reduce indexed W-types to
W-types. IR types to indexed types. IITs were introduced by
Forsberg. What he did.  QIITs by Forsberg-lots of people. Our QIIT
paper. Jakob TYPES 2019 abstract.  Jasper Huguin: he didn't do
recursion-recursion. Without UIP.

Similar techique for other type theories: Streicher. Initiality
project, Brunerie. These all use quotients. Closest: Lafont TYPES
abstract 2017. Does not use quotients.

Constructing intrinsic syntax but different techniques: Chapman,
Danielsson, Nuo. These use induction-induction which is our goal.

\subsection{Structure of the Paper}

We first define a \emph{theory of IIT signatures} in which every context is a
valid IIT signature (Section \ref{sec:theory_of_signatures}). We present this
type theory as an algebraic \cite{ttintt} theory, specifically as a quotient
inductive-inductive type (QIIT). In Section \ref{sec:ambroise} we show that this
QIIT can be constructed only using inductive types.

\subsection{Formalisation}

\todo{UIP, funext, rewrite rules, version of Agda.}

\section{The Theory of IIT Signatures}
\label{sec:theory_of_signatures}

We use the naming convention: sorts, operators and equations. Sorts
and operators might have parameters.
\begin{alignat*}{5}
  & \rlap{(1) Substitution calculus} \\[0.5em]
  & \Con && : \Set \\
  & \Ty  && : \Con\ra\Set \\
  & \Sub  && : \Con\ra\Con\ra\Set \\
  & \Tm  && : (\Gamma:\Con)\ra\Ty\,\Gamma\ra\Set \\
  & \id && : \Sub\,\Gamma\,\Gamma \\
  & \blank\circ\blank && : \Sub\,\Theta\,\Delta\ra\Sub\,\Gamma\,\Theta\ra\Sub\,\Gamma\,\Delta \\
  & \ass && : (\sigma \circ \delta) \circ \nu = \sigma \circ (\delta \circ \nu) \\
  & \idl && : \id\circ\sigma = \sigma \\
  & \idr && : \sigma\circ\id = \sigma \\
  & \blank[\blank] && : \Ty\,\Delta\ra\Sub\,\Gamma\,\Delta\ra\Ty\,\Gamma \\
  & \blank[\blank] && : \Tm\,\Delta\,A\ra(\sigma:\Sub\,\Gamma\,\Delta)\ra\Tm\,\Gamma\,(A[\sigma]) \\
  & [\id] && : A[\id] = A \\
  & [\circ] && : A[\sigma \circ \delta] = A[\sigma][\delta] \\
  & [\id] && : t[\id] = t \\
  & [\circ] && : t[\sigma \circ \delta] = t[\sigma][\delta] \\
  & \cdot && : \Con \\
  & \epsilon && : \Sub\,\Gamma\,\cdot \\
  & {\cdot\eta} && : \{\sigma : \Sub\,\Gamma\,\cdot\} \ra \sigma = \epsilon \\
  & \blank\ext\blank && : (\Gamma:\Con)\ra\Ty\,\Gamma\ra\Con \\
  & \blank,\blank && : (\sigma:\Sub\,\Gamma\,\Delta)\ra\Tm\,\Gamma\,(A[\sigma])\ra\Sub\,\Gamma\,(\Delta\ext A) \\
  & \p && : \Sub\,(\Gamma\ext A)\,\Gamma \\
  & \q && : \Tm\,(\Gamma\ext A)\,(A[\p]) \\
  & {\ext\beta_1} && : \p\circ(\sigma, t) = \sigma \\
  & {\ext\beta_2} && : \q[\sigma, t] = t \\
  & {\ext\eta} && : (\p, \q) = \id \\
  & {,\circ} && : (\sigma, t)\circ\delta = (\sigma\circ\delta, t[\delta]) \\[0.5em]
  & \rlap{(2) Sorts and operators} \\[0.5em]
  & \U && : \Ty\,\Gamma \\
  & \El && : \Tm\,\Gamma\,\U\ra\Ty\,\Gamma \\
  & {\U[]} && : \U[\sigma] = \U \\
  & {\El[]} && : (\El\,a)[\sigma] = \El\,(a[\sigma]) \\[0.5em]
  & \rlap{(3) Parameters} \\[0.5em]
  & \Pi && : (a:\Tm\,\Gamma\,\U)\ra\Ty\,(\Gamma\ext\El\,a)\ra\Ty\,\Gamma \\
  & \app && : \Tm\,\Gamma\,(\Pi\,a\,B)\ra\Tm\,(\Gamma\ext \El\,a)\,B \\
  & {\Pi[]} && : (\Pi\,a\,B)[\sigma] = \Pi\,(a[\sigma])\,(B[\sigma^\uparrow]) \\
  & {\app[]} && : (\app\,t)[\sigma^\uparrow] = \app\,(t[\sigma]) \\[0.5em]
  & \rlap{(4) Metatheoretic parameters} \\[0.5em]
  & \Pim && : (T:\Set)\ra(T\ra\Ty\,\Gamma)\ra\Ty\,\Gamma \\
  & \blank\appm \blank && : \Tm\,\Gamma\,(\Pim\,T\,B)\ra(\alpha:T) \ra\Tm\,\Gamma\,(B\,\alpha) \\
  & {\Pim[]} && : (\Pim\,T\,B)[\sigma] = \Pim\,T\,(\lambda \alpha.(B\,\alpha)[\sigma]) \\
  & {\appm[]} && : (t\appm\alpha)[\sigma] = (t[\sigma])\mathop{\appm}\alpha \\[0.5em]
\end{alignat*}
\todo{We may need more abbrevs here}
Abbreviation:
\[
  \sigma^\uparrow = (\sigma\circ\p,\q)
\]

\todo{add examples where all the different constructors are used}


\section{Constructing the Theory of IIT Signatures}
\label{sec:ambroise}

\todo{Write this section}

Show that the theory of IIT signatures as described in Section
\ref{sec:theory_of_signatures} exists, that is, it can be defined only
using indexed inductive types and a $\Prop$ sort.

Ambroise has done this in Agda with funext and UIP using an
extrinsic--intrinsic translation.

Ambroise formalisation: \url{https://github.com/amblafont/UniversalII/tree/cwf-syntax/Cwf}.

In this formalization, there is a discrepancy for the typing rule of application
compared to the one of Section~\ref{sec:theory_of_signatures}: it is not
\[
   \app  : \Tm\,\Gamma\,(\Pi\,a\,B)\ra\Tm\,(\Gamma\ext \El\,a)\,B
\]
but
\[
   \app  : \Tm\,\Gamma\,(\Pi\,a\,B)\ra(u : \Tm\,\Gamma\,(\El\,a))\ra\,\Tm\,\Gamma\,B[<u>].
\]
Actually, I started with the first version, but at some point I
changed for the second one because it made some stuff easier (although
I don't remember exactly what it was).

I think this should be doable with only UIP (or a $\Prop$-valued
equality type?) and not funext. This would be important for a
computational justification of IITs.

Unfortunately, it seems that funext is needed (see the use of funext in
Syntax.agda) for the metatheoretic/infinitary
parameters part (4-5), for example in the case of the metatheoretic
$ \Pim  : (T:\Set)\ra(T\ra\Ty\,\Gamma)\ra\Ty\,\Gamma $. This needs to be
investigated more carefully.

\paragraph*{Construction}


% We are also careful to explain where UIP or functional extensionality that is
% available in extensional type theory are used.

The construction consists of the following steps:
\begin{enumerate}
\item Definition of untyped syntax (as a family of inductive datatypes) together
  with typing judgments (as inductive relations on the untyped
  syntax), and construction of a model of the theory of IIT
  signatures from well-formed terms, denoted $S$.
\item Construction of a morphism $\rec:S\to M$ for arbitrary $M$, by:
  \begin{enumerate}
  \item defining a relation $\blank \sim \blank$ between the (well-typed) syntax and a given
    model. The idea is that given a syntactic context $\Gamma$ and a semantic
    context $\model{\Gamma}$ of the model $M$, we have $\Gamma \sim
    \model{\Gamma}$ if and only if $\rec\,\,\Gamma = \model{\Gamma}$, and
    similarly for types, terms, and substitutions;
    % semantic context $\model{\Gamma}$ of
    % the model $M$ if and only if $\Gamma$ relates to $\rec\,{\Gamma}$;
    % enjoyed by the initial morphism from the
    % syntax to a given model;
  \item showing that this relation is functional.
    \end{enumerate}
  \item Proving the uniqueness of this morphism by showing that any morphism
    $f:S\to M$ satisfies the relation. For example, for any syntactic context
    $\Gamma$ we have $\Gamma\sim f\,\Gamma$.
      % TODO donner composantes sur les contextes
\end{enumerate}
The next sections detail each of these steps.
\subsection{Syntactic Model}

The goal is to define the syntactic model where contexts are pairs of a
precontext together with a well-formedness proof, and similarly for types, terms
and substitutions.

Crucially, we do not have conversion relations for typed syntax, nor do we need
to use quotients when giving the syntactic model. This is possible because there
are no $\beta$-rules in the theory of signatures. Hence, we consider only normal
terms in the untyped syntax, and define weakening and substitution by recursion.
Avoiding quotients is important for two reasons. First, it greatly simplifies
formalisation. Second, we aim to reduce IITs to a minimal feature set, and we
get a stronger result if we do not use quotients.

The next sections present the definition of the untyped syntax and the
associated typing judgments.

\subsubsection{Untyped syntax}
The untyped syntax is defined as the following inductive datatype. Variables are
modeled as de Bruijn indices, i.e.\ as natural numbers pointing to a position in
the context. We use the additional default constructor $\erru:\Tmu$ in case of
error (ill-scoped substitution). The typing judgments will not mention $\erru$.
\todo{András: prettify below maybe}
\begin{alignat*}{5}
  & \rlap{(1) Substitution calculus} \\[0.5em]
  & \Conu && : \Set \\
  & \Tyu  && : \Set \\
  & \Subu  && : \Set \\
  % & \Subpre  && : \Set \\
  & \Tmu  && : \Set \\
  & \emptyu && : \Conu \\
    & \epsilonu && : \Subu \\
  & \blank\extu\blank && : \Conu\ra\Tyu\ra\Conu \\
  & \blank\consu\blank && : \Subu\ra\Tmu\ra\Subu \\
  & \varu  && : \N \ra \Tmu \\
  & \rlap{(2) Sorts and operators} \\[0.5em]
  & \Uu && : \Tyu \\
  & \Elu && : \Tmu\ra\Tyu \\
  & \rlap{(3) Inductive parameters} \\[0.5em]
  & \Piu && : \Tmu\ra\Tyu\ra\Tyu \\
  & \blank\appu\blank && : \Tmu\ra\Tmu\ra \Tmu \\
  & \rlap{(4-5) Metatheoretic parameters} \\[0.5em]
  & \Pimu && : (T:\Set)\ra(T\ra\Tyu)\ra\Tyu \\
  & \Piiu && : (T:\Set)\ra(T\ra\Tmu)\ra\Tmu \\
  & \blank\appmu \blank && : \Tmu\ra(\alpha:T) \ra\Tmu\\
  & \rlap{(6) Default value} \\[0.5em]
  & \erru && : \Tmu
\end{alignat*}


% The type of (untyped) substitution is then defined as $\Subu = \List\,\Tmu$.
% Note that $\Conu$ could also be defined as a list of types in a similar fashion.

\subsubsection{Untyped weakening}


% Substitutions of types and terms are defined recursively.


Note that $(\Piu \,A\,B)[\sigma]$ should be defined as $\Piu
\,A[\sigma]\,B[\varu\, 0 \,\consu\, \wkO\,\sigma]$, and thus we need to define
$\wk0$, the weakening of substitutions.  The basic idea is to increment the de
Bruijn indices of all the variables.  Actually this is not so simple because of
the $\Piu$ type: indeed, we want to define $\wk(\Piu\,A\,B)$ as the $\Pi$ type
of the weakening of $A$ and $B$, but here, $B$ must be weakened with respect to
the second last variable of the context, rather than the last one.  For this
reason, we need to generalize the weakening as occuring anywhere in the context.
\begin{alignat*}{5}
  & \wk_n && :  \Tyu\ra\Tyu \\
  & \wk_n && :  \Tmu\ra\Tmu \\
  % & \wk && : \N \ra \N\ra\N \\
  & \wkO && : \Subu\ra \Subu
  \end{alignat*}
  The natural number $n$ specifies at which position of the context the
  weakening occurs.
  Here, $\wkO$ weakens with respect to the last variable.

Later, in Section~\ref{ss:weaken-typing}, we show that weakening preserves
typing. Stating a typing rule for this operation requires to have a way to
express the weakening occuring at the middle of a context. We consider pairs of
untyped contexts, which should be thought of as a splitting of a context at some
position. The full context is recovered by merging the two components:
% (which appears in the typing rule of a weakening occuring in the middle of a context),
% and the length $\length{\Gamma}$ of a
% context $\Gamma$:
\begin{alignat*}{5}
  & \blank \merge \blank && :  \Conu\ra\Conu\ra\Conu \\
  & \Gamma \merge \cdot && :=  \Gamma \\
  & \Gamma \merge (\Delta \extu A) && :=  (\Gamma \merge \Delta)\extu A
\end{alignat*}
We think of the second context as a telescope over the first one. We also define
weakening for telescopes, which will be used to give typing rules for telescopes
in Section \ref{sec:typing_judgments}:
\begin{alignat*}{5}
  & \wkO && : \Conu\ra \Conu
  \\
  & \wkO\,\emptyu && := \emptyu
  \\
  & \wkO\,(\Delta\,\extu A) && := \wkO\,\Delta\,\extu \wk_{\length{\Delta}}\,A
  \end{alignat*}
  \subsubsection{Untyped substitution}
  We define single substitution by recursion on presyntax:
  % & \rlap{(1) Substitution calculus} \\[0.5em]
\begin{alignat*}{5}
  & \blank[\blank := \blank] && : \Tyu\ra\N\ra\Tmu\ra\Tyu \\
  & \blank[\blank := \blank] && : \Tmu\ra\N\ra\Tmu\ra\Tmu
  % \\
  % & \blank[\blank := \blank] && : \N\ra\N\ra\Tmu\ra\Tmu
  \end{alignat*}
  This is enough to define the typing judgments: indeed, the typing rule for application
  involves only a unary substitution.

  However, to construct the initial model of the QIIT, we need to define
  the full substitution calculus:
\begin{alignat*}{5}
  & \blank[\blank] && : \Tyu\ra\Subu\ra\Tyu \\
  & \blank[\blank] && : \Tmu\ra\Subu\ra\Tmu \\
  & \blank \circ \blank && : \Subu\ra\Subu\ra \Subu
  \end{alignat*}
These can be defined either by iterating unary substitutions, or by recursion
on untyped syntax: the two ways yield provably equal definitions. In the
following, we assume that it is defined by recursion. We also make use of the
following definition:
  \begin{alignat*}{5}
  % liftV (S (n + p)) (liftV n q) ≡ liftV n (liftV (n + p) q)
  & \keep && : \Subu \ra \Subu \\
   &&& := \lambda \sigma. \varu\, 0 \,\consu\, \wkO\,\sigma
  \end{alignat*}
  The idea is that if $\sigma$ is a substitution between contexts $\Gamma$ and
  $\Delta$, then $\keep\,\sigma$ is a substitution between contexts $\Gamma, A[\sigma]$
  and $\Delta,A$ for any type $A$. This occurs when defining
  $(\Piu\,A\,B)[\sigma]$ as $\Piu (A[\sigma])(B[\keep \,\sigma])$.




  We can define the identity substitution on a context $\Gamma$ as follows,
  where $\length{\Gamma}$ is the length of the context $\Gamma$, and
  $\keep^{\length{\Gamma}}$ is $\keep$ iterated $\length{\Gamma}$ times:
\begin{alignat*}{5}
  & \length{\Gamma} && : \Conu\ra\N
  \\
  & \idu && : \Conu \ra \Subu \\
  & && := \lambda \Gamma.\keep^{\length{\Gamma}} \epsilonu
  \end{alignat*}


  \subsubsection{Exchange laws: weakening and substitution}
  Many lemmas for types and terms are shown by induction on the untyped syntax.
  Below, $Z$ denotes either a term or a type.
\begin{alignat*}{5}
  % liftV (S (n + p)) (liftV n q) ≡ liftV n (liftV (n + p) q)
  & \wk\mhyphen\wk && : \wk_{n+p+1} (\wk_n\,Z) = \wk_n(\wk_{n+p}\,Z)\\
  % n-subZ-wkZ : ∀ n A z → (lifZZ n A) [ n ↦ z ]Z  ≡ A
  & \wk_n[n] && : (\wk_{n} Z)[ n := z] = Z \\
  % lifZ-l-subV : ∀ n p u x → lifZZ (n + p) (x [ n ↦ u ]V) ≡ (lifZV (S (n + p)) x) [ n ↦ (lifZZ p u) ]V
  & \wk_+[] && : (\wk_{n+p+1} Z)[ n := \wk_p u] = \wk_{n+p}(Z[n := u]) \\
  % l-subV-lifZV : ∀ Δ u n x → (lifZV n x) [ (S (n + Δ)) ↦ u ]V  ≡ lifZZ n (x [ (n + Δ) ↦ u ]V)
  & \wk[+] && : (\wk_{n} Z)[ n+p+1 :=  u] = \wk_{n}(Z[n+p := u]) \\
  % l-subV-l-subV : ∀ n p z u x →  (x [ n ↦ u ]V) [ (n + p) ↦ z ]Z  ≡ (x [ (S (n + p)) ↦ z ]V) [ n ↦ (u [ p ↦ z ]Z) ]Z
  & [][+] && : Z[n :=u][n+p:=z] = Z[n+p+1:=z][n:= (u[p:=z])]
  \end{alignat*}
  Below, $\keep^n$ denotes $\keep$ iterated $n$ times.
\begin{alignat*}{5}
  % liftV (S (n + p)) (liftV n q) ≡ liftV n (liftV (n + p) q)
  % liftV=wkS
  & [\keep^n\mhyphen \wkO] && : Z[\keep^n (\wkO\,\sigma)] = \wk_n (Z[\keep^n\,\sigma])
  \\
  % lifZₛV : ∀ n xp σp Zp → (lifZV n xp [ iZer n keep (Zp ∷ σp) ]V) ≡ (xp [ iZer n keep σp ]V)
  & \wk_n[\keep^n\mhyphen\cons]  & &
  :
  (\wk_n Z) [\keep^n (u \consu \sigma)] = Z [ \keep^n \sigma]
  \\
  % l-sub[]V : ∀ n x z σ
    % ((x [ n := z ]V) [ iZer n keep σ ]Z) = (x [ iZer (S n) keep σ ]V) [ n := (z [ σ ]Z) ]Z
   & [:=][\keep] && : Z [ n:= u][\keep^n \sigma] =  Z[\keep^{n+1}\,\sigma][n := u[\sigma]]
  \end{alignat*}
As particular cases for $n=0$, we get
\begin{alignat*}{5}
  & {}{\circ}{}\wkO & & : \sigma \circ (\wkO \tau) = \wkO (\sigma \circ \tau) \\
  & \wkO{}{\circ}{}, && : \wkO\,\sigma \circ (t \consu \tau ) = \sigma \circ \tau \\
  & [\wkO] && : t[\wkO\,\sigma] = \wk_0 (t[\sigma]) \\
  & \wk_0[\cons] && : (\wk_0 Z)[u \consu \sigma] = Z[\sigma] \\
  & [0:=][] && : Z [ 0:= u][ \sigma] =  Z[\keep\,\sigma][0 := u[\sigma]]
\end{alignat*}
Finally, we show the following:
\begin{alignat*}{5}
  & [][] && : Z[\sigma][\tau] = Z[\sigma\circ\tau]
  \\
  & \ass && : (\sigma\circ \delta)\circ\tau = \sigma\circ (\delta\circ\tau)
  \end{alignat*}
We defer laws for identity substitutions after the definition of the typing
judgments, as the proofs require that some inputs are well-typed.

  % The two ways yield provably equal definitions, but
  % The advantage of the second
  % one in a type theory with UIP and function extensionality is that we get
  % definitional equalities such as $(t\,\appu\,u)[\sigma]=\app\,t[\sigma]\,u[\sigma]$.

  % \\
  % & \blank[\blank := \blank] && : \N\ra\N\ra\Tmu\ra\Tmu


  \subsubsection{Typing judgments}
  \label{sec:typing_judgments}
  The typing judgments are defined as the following inductive datatype indexed over the
  untyped syntax:
\begin{alignat*}{5}
  & \rlap{(1) Substitution calculus} \\[0.5em]
  & \blank \vdash && : \Conu\ra \Set \\
  & \blank \vdash \blank  && : \Conu\ra \Tyu\ra\Set \\
  & \blank \vdash \blank \in \blank  && : \Conu\ra \Tmu\ra \Tyu\ra \Set \\
  & \blank \vdash \blank \invar \blank  && : \Conu\ra \N \ra \Tyu\ra \Set \\
  & \blank \vdash \blank \Rightarrow \blank  && : \Conu\ra \Subu \ra \Conu\ra \Set \\
  % & \Subpre  && : \Set \\
  & \emptyw && : \emptyu \vdash \\
  & \epsilonw && : \Gamma\vdash \epsilonu \Rightarrow \emptyu \\
  & \blank\extw\blank && : ( \Gamma\vdash )  \ra  (\Gamma\vdash A)  \ra
  \Gamma\extu A \vdash \\
  & \consw && :
    (\Delta \vdash) \ra
    (\Gamma\vdash \sigma \Rightarrow\Delta)\ra
    ( \Delta \vdash A) \ra
    ( \Gamma \vdash t \in A[\sigma]) \ra
    \Gamma \vdash t \consu \sigma \Rightarrow \Delta\extu A
   \\
  & \varw  && : (\Gamma \vdash n \invar A) \ra \Gamma \vdash \varu n \in A \\
  & \vzw  && : (\Gamma\vdash)\ra(\Gamma\vdash A)\ra\Gamma \extu  A \vdash 0 \invar \wku \, A \\
  & \vsw  && : (\Gamma\vdash)\ra(\Gamma\vdash A)\ra(\Gamma \vdash n \invar A) \ra (\Gamma \vdash B) \ra \Gamma \extu  B
  \vdash S\,n \invar \wku \, A \\
  & \rlap{(2) Sorts and operators} \\[0.5em]
  & \Uw && : (\Gamma \vdash)\ra \Gamma \vdash \Uu \\
  & \Elw && : (\Gamma \vdash)\ra (\Gamma \vdash a \in \Uu)\ra\Gamma \vdash \Elu\,a \\
  & \rlap{(3) Inudctive parameters} \\[0.5em]
  & \Piw && :
    (\Gamma \vdash)\ra (\Gamma \vdash a \in \Uu)\ra(\Gamma \extu
  \Elu\,a \vdash B)\ra\Gamma \vdash \Piu \, a \, B \\
  & \appw && :
    (\Gamma \vdash)\ra (\Gamma \vdash a \in \Uu)\ra(\Gamma \extu
    \Elu\,a \vdash B)
    \\
    & &&
    \ra(\Gamma \vdash t \in \Piu \, a \, B )
    \ra
    (\Gamma \vdash u \in \Elu\,a)
    \ra
    \Gamma \vdash t \appu u \in  B [0 := u] \\
  & \rlap{(4) Inductive parameters} \\[0.5em]
  & \Pimw && :
    (T:\Set)\ra(A : T\ra \Tyu)\ra(\Gamma \vdash)\ra
    ((t : T)\ra\Gamma\vdash A\,t) \ra
    \Gamma \vdash \Pimu \,T\,A
    \\
    & \appmw  && :
    (T:\Set)\ra(A : T\ra \Tyu)\ra(\Gamma \vdash)\ra
    ((t : T)\ra\Gamma\vdash A\,t)
    \\
    & &&
    \ra(\Gamma \vdash t \in \Pimu \,T\,A)
    \ra (u : T) \ra \Gamma \vdash t\,\appmu\,u \in A\,u
\end{alignat*}
There is possibility of redundancy in the arguments of the constructors. Here,
we are ``{paranoid}'', so that we get more inductive hypotheses when performing
recursion.

% The substitution part could be moved after the mutual definition of terms, types,
% and contexts.

\subsubsection{Weakening preserves typing}
\label{ss:weaken-typing}
We prove by mutual induction that typing judgments are stable under weakening,
for contexts, types, terms, and substitutions:
% wkTelw : ∀ {Γp}{Ap}(Aw : Γp ⊢ Ap)Δp (Δw : (Γp ^^ Δp) ⊢) → ((Γp ▶p Ap) ^^ wkTel Δp) ⊢
% liftTw : ∀ {Γp}{Ap}(Aw : Γp ⊢ Ap)Δp{Bp}(Bw : (Γp ^^ Δp) ⊢ Bp) → ((Γp ▶p Ap) ^^ wkTel Δp) ⊢ (liftT ∣ Δp ∣ Bp)
\begin{alignat*}{5}
  & \wf{\wkO} && : (\Gamma \vdash A)\ra(\Gamma \merge \Delta\vdash) \ra
  \Gamma \extu A \merge \wkO\, \Delta \vdash \\
  % liftV (S (n + p)) (liftV n q) ≡ liftV n (liftV (n + p) q)
  & \wf{\wk} && : (\Gamma \vdash A)\ra(\Gamma \merge \Delta\vdash B) \ra
  \Gamma \extu A \merge \wkO\,\Delta \vdash \wk_{\length{\Delta}}\, B  \\
  & \wf{\wk} && : (\Gamma \vdash A)\ra(\Gamma \merge \Delta\vdash t\in B) \ra
  \Gamma \extu A \merge \wkO\,\Delta \vdash \wk_{\length{\Delta}}\, t \in \wk_{\length{\Delta}}\,B  \\
  & \wf{\wkO} && : (\Gamma \vdash A)\ra(\Gamma \vdash \sigma
  \Rightarrow \Delta) \ra
  \Gamma \extu A \vdash \wkO\, \sigma \Rightarrow \Delta
  \end{alignat*}
\subsubsection{Substitution preserves typing}
  First, we show that judgments are stable under substitution.
\begin{alignat*}{5}
  % Tyw[] : ∀ {Γp}{Ap}(Aw : Γp ⊢ Ap) {Δp}(Δw : Δp ⊢){σp}(σw :  Δp ⊢ σp ⇒ Γp) → Δp ⊢ (Ap [ σp ]T)
  & \wf{[]} && : (\Gamma \vdash)\ra(\Delta \vdash A)\ra
  (\Gamma \vdash \sigma \Rightarrow \Delta) \ra
  \Gamma \vdash A [ \sigma ]
  \\
  & \wf{[]} && : (\Gamma \vdash)\ra(\Delta \vdash t \in A)\ra
  (\Gamma \vdash \sigma \Rightarrow \Delta) \ra
  \Gamma \vdash t [\sigma] \in A [ \sigma ]
  \\
  & \wf{[]} && : (\Delta \vdash x \invar A)\ra
  (\Gamma \vdash \sigma \Rightarrow \Delta) \ra
  \Gamma \vdash x [\sigma] \in A [ \sigma ]
  % ∘w : ∀ {Γ} {Δ σ} (σw :  Δ ⊢ σ ⇒ Γ)
  %  {Y}(Yw : Y ⊢) {δ} (δw :  Y ⊢ δ ⇒ Δ) →
  %  Y ⊢  (σ ∘p δ) ⇒ Γ
  \\
  & \wf{{\circ}} && :
  (\Gamma \vdash) \ra
  (\Gamma \vdash \sigma \Rightarrow \Delta) \ra
  (\Delta \vdash \tau \Rightarrow E) \ra
  \Gamma \vdash \tau \circ \sigma \Rightarrow E
  % keepw : ∀ {Γp}(Γw : Γp ⊢){Δp}(Δw : Δp ⊢){σp}(σw :  Γp ⊢ σp ⇒ Δp) {Ap}(Aw : Δp ⊢ Ap ∈ Up ) → (Γp ▶p (Elp Ap [ σp ]T )) ⊢ (keep σp) ⇒ (Δp ▶p Elp Ap)
  % & \wf{keep-\El} && : (\Gamma \vdash)\ra(\Delta \vdash A)\ra
  % (\Gamma \vdash \sigma \Rightarrow \Delta) \ra
  % \Gamma \vdash A [ \sigma ]
  % \\
  \end{alignat*}
\subsubsection{Laws for identity substitutions}
We show category and functor laws involving identity substitution, for
well-formed types, terms and substitutions.
\begin{alignat*}{5}
  % [idp]V : ∀ {Γ}{A}{x}(xw : Γ ⊢ x ∈v A) → (x [ idp ∣ Γ ∣ ]V) ≡ V x
  & [\idu] && : (\Gamma \vdash A)\ra A [ \idu\,\Gamma]  \\
  & [\idu] && : (\Gamma \vdash x \invar A)\ra x [ \idu\,\Gamma] = V x \\
  & [\idu] && : (\Gamma \vdash t \in A)\ra t [ \idu\,\Gamma] = t \\
  & \idru && : (\Gamma \vdash \sigma \Rightarrow\Delta)\ra \sigma \circ \idu\,\Gamma = \sigma \\
  & \idlu && : (\Gamma \vdash \sigma \Rightarrow\Delta)\ra \idu\,\Delta\circ \sigma = \sigma
  \end{alignat*}
Finally, we show that the identity substitution itself is well-typed:
  \begin{alignat*}{5}
  & \idw && : (\Gamma \vdash)\ra \Gamma \vdash \idu\,\Gamma \Rightarrow \Gamma
  \end{alignat*}



\subsubsection{Proof irrelevance and unicity of typing}
\label{ss:uniq-typing-types}
A type is a proposition, or proof-irrelevant, if it has at most one inhabitant.
  \[
    \isaprop\,T := (a : T) \ra (a' : T) \ra a = a'
  \]
We show that each of the typing judgments is unique in the
following sense:

\todo{Prettify table.}
\begin{align*}
    \isp{\Conw}
  : \isaprop\, (\Gamma\vdash)
 & &
    \isp{\Tmw}
  : \isaprop\, (\Gamma\vdash t \in A)
    \\
    \isp{\Tyw}
  : \isaprop\, (\Gamma\vdash A)
    &&
    \isp{\wf\Var}
  : \isaprop\, (\Gamma\vdash x \invar A)
  \\
  &
        \isp{\wf\Sub}
  : \isaprop\, (\Gamma\vdash \sigma \Rightarrow \Delta)
  \end{align*}
We also show unicity of typing:
\begin{alignat*}{5}
  &
  \Tmw{=}\Ty & &:
  (\Gamma \vdash t \in A) \ra
  (\Gamma \vdash t \in B) \ra A = B
  \\
  &
  \wf\Var{=}\Ty & &:
  (\Gamma \vdash x \invar A) \ra
  (\Gamma \vdash x \invar B) \ra A = B
  \end{alignat*}
Let us consider for instance the application constructor $\appw$: for a codomain
type $B$ it yields an overall type $C=B[0 := u]$ for an application. Even
if $C$ is known a priori, there may be another $B$ for which $B[0 := u] = C$,
possibly leading to many proofs that $t\appu u$ has type $C$. Unicity of typing
solves this issue, as $B$ is then uniquely determined by the type $\Piu\,A\,B$
of $t$.

%% Actually, the application and successor cases require to show uniqueness of
%% types in a given context, before.
%% Let us consider for instance the application constructor $\appw$: it provides a
%% type $B$ such that the final type is $C=B[0 := u]$. Even if $C$ is known a
%% priori, there may be another $B$ for which $B[0 := u] = C$, possibly leading to
%% many proofs that $t\appu u$ has type $C$. Uniqueness of types
%% solves this issue, as $B$ is then uniquely determined by the type $\Piu\,A\,B$ of $t$.


% \subsubsection{UIP and functional extensionality}
 % , which is not applyable without
 % functional extensionality.

 \subsubsection{Syntactic model}
 The syntactic model is obtained by packing the untyped syntax with the typing
 judgments:
% % \subsection{QIIT model}
% Then, we show that they are stable by weakening and substitution.
% Packing the untyped syntax with well-formed typed judgments yields
% a model $S$ of the QIIT. Let us give the most important components:
\begin{alignat*}{5}
 & \syn\Con && := \sum_\Gamma \Gamma\vdash
 \\
 & \syn\Ty \,(\Gamma,\wf\Gamma)&& := \sum_A \Gamma\vdash A
 \\
 & \syn\Tm \,(\Gamma,\wf\Gamma)(A,\wf A)&& := \sum_t \Gamma\vdash t \in A
 \\
 & \syn\Sub \,(\Gamma,\wf\Gamma)(\Delta,\wf \Delta)&& := \sum_\sigma \Gamma\vdash \sigma \Rightarrow \Delta
\end{alignat*}
The other fields are given straightforwardly.  Regarding the equations, it is
enough to prove them only for the untyped syntactic part: as we argued in
Section~\ref{ss:uniq-typing-types}, the proofs of typing judgments are
automatically equal.

None of the constructions in this section use UIP. Functional extensionality is
necessary because the untyped metatheoretic $\Pi$ takes a metatheoretic function
as an argument. An example induction step that uses it is in the type
preservation proof for identity substitutions (this is an equation satisfied by
a model of the QIIT), in particular in the case $(\Pim\,T\,
A)[\id]=\Pim\,T\,A$. Indeed, the left hand side of this equation is equal to
$\Pim\,T\,(\lambda t.(A\,t)[\id])$ by definition, whereas the induction
hypothesis states that $(t:T)\ra (A\,t)[\id]=A\,t$.

\subsection{Relating the Syntax to a Model}
It remains to show that the constructed syntactic model is initial. To achieve
this, we first define a relation between the syntactic model and an arbitrary
model, then show that the relation is functional, which lets us extract a
homomorphism from the relation.

This approach is an alternative version of Streicher's method for interpreting
preterms in an arbitrary model\cite{streichersemantics}. Streicher first defines
a family of partial maps from the presyntax to a model, then shows that the maps
are total on well-formed input. We have found that our approach is significantly
easier to formalise. Too see why, note that the right notion of partial map in
type theory, which does not presume decidable definedness, is fairly
heavyweight:
\[
  \mathsf{PartialMap}\,A\,B := A\ra (P : \mathsf{hProp})\times(P \ra B)
\]
Here $\mathsf{hProp}$ denotes a type which is propositional. In the above
definition, we notice an opportunity for converting a fibered definition of a
type family into an indexed one; if we drop the propositionality for $P$ for the
time being, we may equivalently return a family indexed over $B$, which is
exactly just a relation $A\ra B\ra\Set$. Then, in our approach, we recover
uniqueness of $P$ through the functionality requirement on the $A\ra B\ra\Set$
relation, and totality by already assuming well-formedness of $A$. In type
theory, using indexed families instead of display maps is a common convenience,
since the former are natively supported in type theory, while the latter require
carrying around auxiliary propositional equalities.

\subsubsection{The functional relation}

Given a model $M$ of the QIIT, we define the functional relation satisfied by
the initial morphism $\rec : S\to M$ by recursion on the typing judgments.
% As these judgments are propositions, we will sometimes
If $\Gamma$ is a context in $S$ and $\model{\Gamma}$ is a semantic context
(i.e.\ a context of the model $M$), we want to define a type
$\Gamma\sim \model{\Gamma}$ equivalent to $\rec\,\,\Gamma=\model{\Gamma}$. Of
course, at this stage, $\rec$ is not available yet since the point of
defining this relation is to construct $\rec$ in the end.

For a type $A$ in a context $\Gamma$, we want to define a relation
$A\sim \model{A}$ that is equivalent to $\rec\,\,A=\model{A}$.
% where $\model{A}$ is a
% semantic type in a semantic context $\model{\Gamma}$.
For this equality to make sense, the semantic type $\model{A}$ must live
in the semantic context $\rec\,\,\Gamma$. But again, $\rec$ is not
yet available at this stage. Exploiting the expected equivalence between
$\Gamma\sim\model\Gamma$ and $\rec\,\,\Gamma=\model\Gamma$, we may consider defining $A\sim\model{A}$ under the
hypotheses that $\model{A}$ lies in a semantics context $\model\Gamma$ which is
related to $\Gamma$. Then, the type of the relation for types is
\[
  (\Gamma : \syn\Con)\ra (A : \syn\Ty\syn\Gamma) \ra
  (\model\Gamma : \model\Con)
  \ra
  (\Gamma \sim \model\Gamma)
  \ra
  (\model{A}:\model\Ty\model\Gamma)
  \ra
  \Set
\]
Note that the relation on contexts
must be defined mutually with the relation on types (see for example the case of
context extension), but here, the relation on contexts appears as the type of an
argument of the relation on types.
% We leave this possibility aside,
Our metatheory does not support this kind of recursive-recursive
definitions\footnote{And to our knowledge there is no established semantics for
it in existing literature.}, so we instead just remove the hypothesis
$\Gamma\sim\model\Gamma$ from the list of arguments.  We proceed similarly for
terms and substitutions. Actually, this removal is not without harm. For
example, consider relating the empty substitution
$\Gamma\vdash \epsilonu \Rightarrow \emptyu $ to a semantic substitution
$\model\sigma : \model\Sub\,\model\Gamma\,\model\Delta$. We would like to assert
that $\model\sigma$ equals the empty semantic substitution $\model\epsilon$, but
this is not possible because typechecking requires that $\model\Delta$ is the
empty semantic context. This is precisely what ensured the removed hypothesis
$\syn\cdot \sim \model\Delta$.  Our way out here is to state that $\model\sigma$
is related to the empty substitution if the target semantic context $\model\Delta$
is empty, and, acknowledging this equality, if $\model\sigma$ is the empty
substitution.
% The idea is that we will first prove that there is atmost one semantic type
% which relates to the syntactic one, and then later we will provide such a
% semantic type under the hypothesis that the contexts are related.

Let us mention another possible solution for avoiding recursion-recursion:
defining $A\sim\model{A}$ so that it is equivalent to $(e:\rec\,\Gamma
= \model{\Gamma})\times (\rec\,A = e\#\model{A})$.
% whereas the right hand side is of type $\model\Ty\model{\Gamma}$.
% For a substitution $\sigma$ between contexts $\Gamma$ and $\Delta$,
% a first idea consists in defining a relation $\sigma \sim_{\blank} \model{\sigma} : \Gamma
% \sim \model\Gamma \to \Set $ where $\model{A}$ is a semantic type in the
% semantic context $\model{\Gamma}$. We define the relation on types under the hypothesis
% that the contexts are already related.
% it is natural to define the relation with a semantic type $\model{A}$ in a
% semantic context $\model{\Gamma}$ such that it is equivalent to  $(\rec\,\Gamma = \model{\Gamma})\times (\rec\,A = \model{A})$ (note that the first equality is
% required for the second one to be well-typed).
% The discussion is similar for terms.
% Then, the ultimate goal is to prove that
% $\sum_{\model{\Gamma}}\Gamma\sim\model{\Gamma}$ and
% $\sum_{\model{\Gamma}}({\model{A}:\model{\Ty}\,\model{\Gamma}})\times(A\sim\model{A})$
% are contractible, i.e., have a unique inhabitant.
%The first step consists in showing that these types are propositions.
In contrast to this, our approach
% Actually, we take a different approach for the relation on types
% (and terms) which
yields a more concise definition of the relation.
For example, in the case of the universe, this would lead
to the definition
  $  \Uw\wf{\Gamma} \sim \model{A} := (\wf{\Gamma}\sim\model{\Gamma}) \times (\model{A} = \model{\U})$,
  instead of
  our definition
  $  \Uw\wf{\Gamma} \sim \model{A} := (\model{A} = \model{\U})$.
% Recall that our adapted goal, for types, is to prove that
% $\sum_{\model{A}}A\sim\model{A}$ is a proposition, and
% is inhabited if $\Gamma \sim \model{\Gamma}$ is.
%  This means that in the definition of the relation, we assume
% that contexts are already related, so that we don't need to enforce them to be,
% as it is the case in the original approach.
% We give the two versions of the universe
% case below, to explain the differences.


% Many of the terms below are well-typed because the target model satisfies some
% equalities that are reflected as definitional equalities.
% In the formalization, we even postulate rewrite rules for some of the equations that the
% model $M$ should satisfy (otherwise, we are faced with too many hellish transports).
% This is, in some sense, another place where UIP is needed, as it is known that
% extensional type theory can be translated in intensional type theory using UIP
% and functional extensionality. This requirement can be qualified: we
% could say that we only claim to eliminate to models that definitionally
% satisfies the equalities that we enforce, although we haven't checked that this is enough to
% construct any IIT from the universal one (and anyway, UIP and functional
% extensionality are needed to construct any IIT from the universal one).

Here we provide the definition of the relation by recursion on the typing judgments.
In the definitions, we abbreviate $\relT{\wf{A}}{\model\Gamma}{\model{A}}$ by
$\wf{A}\sim\model{A}$ when $\model{\Gamma}$ can be inferred, and similarly for
terms and substitutions.
\begin{alignat*}{5}
  & \rlap{(1) Substitution calculus} \\[0.5em]
  & \blank \sim{ \blank} && :  \Gamma \vdash \, \ra \model{\Con} \ra \Set \\
  & \relT{\blank}{\model{\Gamma}}{\blank} && :  \Gamma \vdash A \, \ra \model{\Ty}\,\model{\Gamma} \ra \Set \\
  & \relt{\blank}{\model{\Gamma}}{\model{A}}{\blank} && :  \Gamma \vdash t \in A \, \ra \model{\Tm}\,\model{\Gamma}\,\model{A} \ra \Set \\
  & \relV{\blank}{\model{\Gamma}}{\model{A}}{\blank}&& :  \Gamma \vdash x \invar A \, \ra \model{\Tm}\,\model{\Gamma}\,\model{A} \ra \Set \\
  & \relS{\blank}{\model{\Gamma}}{\model{\Delta}}{ \blank} && :  \Gamma \vdash \sigma \Rightarrow \Delta \, \ra \model{\Sub}\,\model{\Gamma}\,\model{\Delta} \ra \Set \\
  \\
  & \emptyw\sim \model{\Gamma} && := \model{\Gamma}=\model{\cdot} \\
  & \relS{\epsilonw}{\model{\Gamma}}{\model{E}}{\model{\delta}}
  % (  \vdash \model{\delta}\Rightarrow \model{E})
  &&
 := (e_E : \model{E} = \model{\cdot})\times (\model{\delta} = e_E \#
 \model{\epsilon})
  \\
  & (\wf{\Gamma}\extw \wf{A})\sim \model{\Delta} &&
   :=
    \sum_{\model{\Gamma}} (\wf{\Gamma}\sim\model{\Gamma}) \times
    \sum_{\model{A}} (\wf{A}\sim\model{A}) \times
    \\
    & && \qquad
    (\model{\Delta} = \model{\Gamma} \model{\ext} \model{A})
    \\
    &
    \relS{(\consw \wf{\Delta}\wf{\sigma}\wf{A}\wf{t})}{\model{\Gamma}}{\model{E}}{\model{\delta}}
    % (\consw \wf{\Delta}\wf{\sigma}\wf{A}\wf{t})\sim
    % \\
    % & \qquad
    % ( \model{\Gamma} \vdash \model{\delta}\Rightarrow \model{E})
    &&
   :=
    \sum_{\model{\Delta}} (\wf{\Delta}\sim\model{\Delta}) \times
    \sum_{\model{\sigma}} (\wf{\sigma}\sim\model{\sigma}) \times
    \\
    & && \qquad
    \sum_{\model{A}} (\wf{A}\sim\model{A}) \times
    \sum_{\model{t}} (\wf{t}\sim\model{t}) \times
    \\
    & && \qquad
    (e_E : \model{E} = \model{\Delta} \model{\ext} \model{A}) \times
    \\
    & && \qquad
    (\delta = e_E \# \model\sigma \model{\cons} \model{t} )
    \\
  & \varw\,\wf{x} \sim \model{t}
   && := \wf{x}\sim \model{t} \\
   & \relV{\vzw\wf{\Gamma}\wf{A}}{\model{\Delta}}{\model{B}}{\model{t}}
  %  \vzw\wf{\Gamma}\wf{A}
  % \sim \\
  % & \qquad (\model{\Delta}\vdash \model{t}\in \model{B} )
   && :=
    \sum_{\model{\Gamma}} (\wf{\Gamma}\sim\model{\Gamma}) \times
    \sum_{\model{A}} (\wf{A}\sim\model{A}) \times \\
    & &&
    \qquad
  (e_\Delta : \model{\Delta} = \model{\Gamma} \model{\ext} \model{A})
     \times
    \\
    & && \qquad
     (e_B : \model{B} = e_\Delta \# \model{\wk}\,\model{A}) \times
     % \\
     % & && \qquad
     (\model{t} = e_\Delta,e_B \# \model{\vz})
  \\
  &
   \relV{\vsw \wf{\Gamma} \wf{A} \wf{n} \wf{B}}{\model\Delta}{\model{C}}{\model{t}}
  % \vsw \wf{\Gamma} \wf{A} \wf{n} \wf{B} \sim
  % \\
  % & \qquad
  %    (\model{\Delta}\vdash \model{t}\in \model{C} )
   && :=
    \sum_{\model{\Gamma}} (\wf{\Gamma}\sim\model{\Gamma}) \times
    \sum_{\model{A}} (\wf{A}\sim\model{A}) \times
    \\
    & &&\qquad
    \sum_{\model{B}} (\wf{B}\sim\model{B}) \times
    \sum_{\model{n}} (\wf{n}\sim\model{n}) \times
    \\
    & &&
    \qquad
  (e_\Delta : \model{\Delta} = \model{\Gamma} \model{\ext} \model{B})
     \times
    \\
    & && \qquad
     (e_C : \model{C} = e_\Delta \# \model{\wk}\,\model{A}) \times
     % \\
     % & && \qquad
     (\model{t} = e_\Delta,e_C \# \model{\vs} \, \model{n})
  \\
  & \rlap{(2) Sorts and operators} \\[0.5em]
  & \Uw \wf{\Gamma} \wf{A} \sim \model{A} &&
   := \model{A} = \model{\U}
  \\
  & \Elw \wf{\Gamma} \wf{a} \sim \model{A}
  && :=
  \sum_{\model{a}} (\wf{a}\sim  \model{a}) \times
    (\model{A} = \model{\El}\,\model{a})
  \\
  & \rlap{(3) Parameters} \\[0.5em]
  & \Piw \wf{\Gamma} \wf{a} \wf{B} \sim \model{C} &&
   :=
      \sum_{\model{a}} (\wf{a}\sim\model{a})
      \times
      \sum_{\model{B}} (\wf{B}\sim\model{B})
      \\
      &&& \qquad \times (\model{C}= \model{\Pi}\,\model{a}\,\model{B})
  \\
  &
  \relt{\appw \wf{\Gamma} \wf{a} \wf{B} \wf{t} \wf{u}}{\model{\Gamma}}{\model{C}}{\model{x}}  &&
  % & \appw \wf{\Gamma} \wf{a} \wf{B} \wf{t} \wf{u} \sim \\
  % & \qquad (\model{\Gamma}\vdash x\in \model{C})  &&
    :=
      \sum_{\model{a}} (\wf{a}\sim\model{a})
      \times
      \sum_{\model{B}} (\wf{B}\sim\model{B})
      \times
      \\
      & && \qquad
      \sum_{\model{t}} (\wf{t}\sim\model{t})
      \times
      \sum_{\model{u}} (\wf{u}\sim\model{u})
      \times
      \\
      & &&
     \qquad  (e_C : \model{C} = \model{B}\model{[0 := \model{u}]})
      \times
      \\
      & &&
      \qquad
      (\model{x} = e_C \# \model{t} \model{\oldapp}\model{u})
    \\
  & \rlap{(4) Metatheoretic parameters} \\[0.5em]
  & \Pimw T\,A\,\wf{\Gamma}\wf{A} \sim \model{B} &&
    :=
      \sum_{\model{A}} ((t : T) \ra \wf{A}\sim\model{A}\, t)
      \times
      (\model{B} = \model{\Pim}\,T\,\model{A})
    \\
    & \relt{\appmw T\,A\,\wf{\Gamma}\wf{A} \wf{t} u}{\model{\Gamma}}{\model{B}}{\model{x}}
    % & \appmw T\,A\,\wf{\Gamma}\wf{A} \wf{t} u \sim
    % \\
    % & \qquad
    % (\model{\Gamma}\vdash x\in \model{B})
     &&
     :=
      \sum_{\model{A}} ((t : T) \ra \wf{A}\sim\model{A}\, t)
      \times
      \sum_{\model{t}} (\wf{t}\sim\model{t})
      \times
      \\ & &&
      \qquad
      (e_B : \model{B} = \model{\Pim}\,T\,\model{A})
      \times
      (\model{x} = e_B \# \model{t}\model{\hat{\oldapp}}u)
\end{alignat*}

% As discussed above, when writing the definition of the type components, we assume that the input semantic context is already
% related to the typing judgment of the syntactic context.
% The first version
% of the relation that we suggested at the beginning of the section
% would rather enforce them to be related.
% In the case of $\U$, this would lead
% to the definition
%   $  \Uw\wf{\Gamma} \sim \model{A} := (\wf{\Gamma}\sim\model{\Gamma}) \times (\model{A} = \model{\U})$,
%   instead of
%   the current definition
%   $  \Uw\wf{\Gamma} \sim \model{A} := (\model{A} = \model{\U})$.
%   Our choice makes the definitions more concise, and similarly
%  in the definition of the term components we assume that the input semantic
%  context and type are already related to their associated well-formedness judgments.
 % However, for some fields, we cannot avoid the first approach.
  % Unfortunately, we need sometimes to require some equalities that are actually
  % redundant with this external assumption.
  %, although we know that these equalities are
  % already satisfied by our assumption.
  % As we argued before, in the  of empty substitution $\epsilon$: we require the equality
  % $(e_C : \model{E} = \model{\cdot})$, although
  % our claimed external assumption that $\model{E}$ is  related to the
  % canonical proof $\epsilonw$ of the typing judgment $\epsilon \vdash$ should imply
  % it.
% Let us comment the first variable case:
  % Another example is the first variable case, where we follow the first verbose
  % approach:
  % the relation gives a semantic context $\model\Gamma$ that must be related
  % to the syntactic $\Gamma$, and a semantic
  % type $\model{A}$ related to the syntactic $A$. The input semantic context is
  % then required to equal $\model{\Gamma}\model{\ext}\model{A}$.   As
  % $\blank \model{\ext}\blank$ is not guaranteed to be injective, these
  % relations are required to show that the relation is right unique.
Next, we prove that this relation is right unique.
Then, we show that the relation is stable under weakening and substitution.
  The last step consists of giving a related semantic counterpart to any
  well-typed context, type or term.
  Everything is done by induction on the typing judgments.
  \subsubsection{Right uniqueness}
  \label{sec:right_uniqueness}
  We show by recursion that the relation is right unique in the following sense:
  \begin{alignat*}{5}
    &
    \isp{\Sigma{\sim}}
    && : (\wf\Gamma : \Gamma \vdash ) \ra
    \isaprop\, (\sum_{\model{\Gamma}} \wf{\Gamma} \sim \model{\Gamma})
    \\
    &
    \isp{\Sigma{\sim}}
    && : (\wf A : \Gamma \vdash A) \ra
    \isaprop\, (\sum_{\model{A}} \wf{A} \sim \model{A})
    \\
    &
    \isp{\Sigma{\sim}}
    && : (\wf t : \Gamma \vdash t \in A) \ra
    \isaprop\, (\sum_{\model{t}} \wf{t} \sim \model{t})
    \\
    &
    \isp{\Sigma{\sim}}
    && : (\wf x : \Gamma \vdash x \invar A) \ra
    \isaprop\, (\sum_{\model{x}} \wf{x} \sim \model{x})
    \\
    &
    \isp{\Sigma{\sim}}
    && : (\wf \sigma : \Gamma \vdash \sigma \Rightarrow \Delta) \ra
    \isaprop\, (\sum_{\model{\sigma}} \wf{\sigma} \sim \model{\sigma})
  \end{alignat*}
  We mentioned that in order to avoid a recursive-recursive definition, we
  removed some hypotheses in the list of arguments of the relation. Such
  hypotheses are sometimes missed, for example in the case of the empty
  substitution or in the case of variables, requiring us to state additional
  equalities. Because of this, we need UIP to show that
  $\sum_{\model{\Gamma}}\Gamma\sim\model{\Gamma}$ and
  $\sum_{\model{A}}A\sim\model{A}$ are propositions.  One may think that the use
  of UIP could be avoided by using the alternative verbose definition that we
  suggested before, expecting that
  $\sum_{\model{\Gamma}}\sum_{\model{A}}A\sim\model{A}$, rather than
  $\sum_{\model{A}}A\sim\model{A}$, is a proposition.  However, this is not
  obvious. For example, we were not able to define
  $\Elw\wf{\Gamma}\wf{a}\sim \model{A}$ in this fashion. In related work,
  Hugunin investigated constructing IITs without
  UIP\cite{hugunin2019constructing}, and demonstrated that well-formedness
  predicates used in syntactic models can be subtly incompatible with UIP. Also,
  while Hugunin does not use UIP, he only shows a weak version of dependent
  elimination for the constructed IITs. Hence, the question whether IITs are
  reducible to inductive types in a UIP-free setting remains open.




%% in this fashion so that
%%   UIP can be avoided to prove this statement.


  % \todo[inline]{András: Hugunin's paper has an example (which is
  %    likely generalizable) that a notion of redundancy in
  %    well-formedness relations implies that UIP is required.}
  \subsubsection{Stability under weakening}
  % Let us argue that it is necessary to show s
  Stability of the relation under
  weakening must be proved before stability under substitution.
  Indeed, in the proof of stability under substitution, the $\Pi$ case
  requires to show that $\Pi \,A[\sigma] \,B[\keep\,\sigma]$ is related to
  $\model{\Pi}\,\model{A}\model{[\sigma]}
  \model{A}\model{[\model{\keep}\,\sigma]}$.
  We would like to apply the induction hypothesis, so we need to show that
  $\keep\,\sigma=\varu\,0\consu\wkO\,\sigma$ is related to
  $\model{\keep}\,\model{\sigma}$, knowing that $\sigma$ is
  related to $\model{\sigma}$.
  As $\keep\,\sigma=\varu 0\consu \wkO\,\sigma$, we are left with showing that
  $\wkO\,\sigma=\sigma\circ\wk$ (where $\wk=\wkO \id$)
  relates to its semantic counterpart.

%   {-

% Suppose that Γ ^^ Δ ⊢ and Γ ⊢ E
% The following function computes both Γ ▶ E ^^ wk_E Δ and a substitution
% from this context to Γ ^^ Δ.
% I don't see how to avoid constructing these two components simultaneously

% -}
% ΣwkTel⇒ᵐ :
%   ∀ {Γ}{Γw :  Γ ⊢}(Γm : ∃ (Con~ Γw) )
%     (Em : M.Ty (₁ Γm))
%         {Δ }{Δw : Γ ^^ Δ ⊢}(Δm : ∃ (Con~ Δw)) →
        % ∃ λ T → M.Sub T (₁ Δm)


  To do that, we show that $\wkO$ preserves the relation, for types and terms.
  This requires to generalise a bit and show that $\wk_n$ preserves the relation,
  as $\wk_0(\Pi\,A\,B)=\Pi\,(\wk_0\,A)(\wk_1\,B)$.
  But remember that $\wk_n$ performs a weakening in the middle of a context, so
  we first define the semantic counterpart of this:
\begin{alignat*}{5}
  & \model{\Sigma\wkO{\Rightarrow}}  && :
   % & && \qquad
  (\wf{\Gamma} : \Gamma \vdash) \ra
  (\wf\Gamma \sim \model{\Gamma}) \ra
  \\
  & && \qquad
  (\wf{\Delta} : \Gamma\merge\Delta \vdash) \ra
  (\wf\Delta \sim \model{\Delta}) \ra
  \\ &&& \qquad
  (\model{A} : \model\Ty\model\Gamma)\ra
  (\model{\Delta'} : \model{\Con}) \times (\model{\Sub}\model{\Delta'}{\model{\Delta}})
\end{alignat*}
Here, $\model{\Delta'}$ should be thought of as the context $\model\Delta$ where
the weakening has happened in the middle of the context, by inserting the type
$\model{A}$ after the prefix $\model{\Gamma}$. Indeed, we expect that
$\model{\Gamma}$ is a prefix of $\model{\Delta}$, as $\model{\Gamma}$ relates to
$\Gamma$ and $\model{\Delta}$ to $\Gamma\merge\Delta$.
% $\wf{\Gamma}\sim\model\Gamma$
% and $\wf\Delta\sim\model\Delta$, and $\Gamma$ is a prefix of $\Gamma\merge\Delta$.
The substitution from the weakened context to the original one must
be computed at the same time otherwise the recursion hypothesis is not strong enough.
Then, we seperate the two components under the same (implicit) hypotheses:
\begin{alignat*}{5}
  & \model{\wkO}\,\model{A}\,\model\Delta  && :
  % (\model{A} : \model\Ty\model\Gamma)\ra
  % (\wf{\Gamma} : \Gamma \vdash) \ra
  % (\wf{\Delta} : \Gamma\merge\Delta \vdash) \ra
  % \\
  % & && \qquad
  % (\wf\Gamma \sim \model{\Gamma}) \ra
  % (\wf\Delta \sim \model{\Delta}) \ra
  % \\ &&& \qquad
   \model{\Con}
   \\
  & \model{\wk{\Rightarrow}}\,\model{A}\,\model\Delta  && :
  % (\model{A} : \model\Ty\model\Gamma)\ra
  % (\wf{\Gamma} : \Gamma \vdash) \ra
  % (\wf{\Delta} : \Gamma\merge\Delta \vdash) \ra
  % \\
  % & && \qquad
  % (\wf\Gamma \sim \model{\Gamma}) \ra
  % (\wf\Delta \sim \model{\Delta}) \ra
  % \\ &&& \qquad
   \model{\Sub}(\model{\wkO}\model{A}\,\model\Delta)\model\Delta
\end{alignat*}
Note that if recursion-recursion is available in the metatheory, $\model\wkO$
and $\wk\model{\Rightarrow}$ can be defined directly without introducing this
intermediate $\model{\Sigma\wkO\Rightarrow}$.


Now, we are ready to prove by mutual recursion on well-typed judgments that
weakening preserves typing. The following statements are all under the
hypotheses
  $(\wf{\Gamma} : \Gamma \vdash)$,
  $(\wf\Gamma \sim \model{\Gamma})$,
  $(\wf{\Delta} : \Gamma\merge\Delta \vdash)$,
  $(\wf\Delta \sim \model{\Delta})$,
  $(\wf{A} : \Gamma \vdash A)$,
  and
  $(\wf{A} \sim \model{A})$.
  % , except the last one about weakening substitutions,
  % which do not require that $\model\Delta$ is related to any well-formed context.
\begin{alignat*}{5}
  & \wkO{\sim}  &&
   :
  % (\wf{\Gamma} : \Gamma \vdash) \ra
  % (\wf\Gamma \sim \model{\Gamma}) \ra
  % \\
  %  & && \qquad
  % (\wf{\Delta} : \Gamma\merge\Delta \vdash) \ra
  % (\wf\Delta \sim \model{\Delta}) \ra
  % \\
  % & && \qquad
  % (\wf{A} : \Gamma \vdash A) \ra
  % (\wf{A} \sim \model{A}) \ra
  % \\
  % & && \qquad
  \wf{\wkO}\,\wf{A}\,\wf{\Delta}\sim\model{\wkO}\model{A}\model{\Delta}
  \\
  % wk
  & \wk{\sim} && :
  % (\wf{\Gamma} : \Gamma \vdash) \ra
  % (\wf\Gamma \sim \model{\Gamma}) \ra
  % \\
  %  & && \qquad
  % (\wf{\Delta} : \Gamma\merge\Delta \vdash) \ra
  % (\wf\Delta \sim \model{\Delta}) \ra
  % \\
  % & && \qquad
  % (\wf{A} : \Gamma \vdash A) \ra
  % (\wf{A} \sim \model{A}) \ra
  % \\
  % &&& \qquad
  (\wf{T} : \Gamma\merge\Delta \vdash T) \ra
  (\wf{T} \sim \model{T}) \ra
  % \\ &&& \qquad
  \wf{\wk}\,\wf{A}\,\wf{T}\sim\model{T}\model{[\model{\wkO{\Rightarrow}}\model{A}\model{\Delta}]}
  \\
    % wk
  & \wk{\sim} && :
  % (\wf{\Gamma} : \Gamma \vdash) \ra
  % (\wf\Gamma \sim \model{\Gamma}) \ra
  % \\
  %  & && \qquad
  % (\wf{\Delta} : \Gamma\merge\Delta \vdash) \ra
  % (\wf\Delta \sim \model{\Delta}) \ra
  % \\
  % & && \qquad
  % (\wf{A} : \Gamma \vdash A) \ra
  % (\wf{A} \sim \model{A}) \ra
  % \\
  % &&& \qquad
  (\wf{t} : \Gamma\merge\Delta \vdash t\in T) \ra
  (\wf{t} \sim \model{t}) \ra
  % \\ &&& \qquad
  \wf{\wk}\,\wf{A}\,\wf{t}\sim\model{t}\model{[\model{\wkO{\Rightarrow}}\model{A}\model{\Delta}]}
  \\
    % wk
  & \wk{\sim} && :
  % (\wf{\Gamma} : \Gamma \vdash) \ra
  % (\wf\Gamma \sim \model{\Gamma}) \ra
  % \\
  %  & && \qquad
  % (\wf{\Delta} : \Gamma\merge\Delta \vdash) \ra
  % (\wf\Delta \sim \model{\Delta}) \ra
  % \\
  % & && \qquad
  % (\wf{A} : \Gamma \vdash A) \ra
  % (\wf{A} \sim \model{A}) \ra
  % \\
  % &&& \qquad
  (\wf{x} : \Gamma\merge\Delta \vdash t\invar T) \ra
  (\wf{x} \sim \model{x}) \ra
  % \\ &&& \qquad
  \wf{\wk}\,\wf{A}\,\wf{x}\sim\model{x}\model{[\model{\wkO{\Rightarrow}}\model{A}\model{\Delta}]}
  % \\ &&& \qquad
  % \wf{\wk}\,\wf{A}\,\wf{x}\sim\model{x}\model{[\model{\wkO{\Rightarrow}}\model{A}\model{\Delta}]}
\end{alignat*}
Then we deduce, still by induction, that weakening of substitution preserves the
relation:
\begin{alignat*}{5}
    % wk
  & \wkO{\sim} && :
  (\wf{\Gamma} : \Gamma \vdash) \ra
  (\wf\Gamma \sim \model{\Gamma}) \ra
  % \\
  %  & && \qquad
  % (\wf{\Delta} : \Gamma\merge\Delta \vdash) \ra
  % (\wf\Delta \sim \model{\Delta}) \ra
  % \\
  % & && \qquad
  (\wf{A} : \Gamma \vdash A) \ra
  (\wf{A} \sim \model{A}) \ra
  \\
  &&& \qquad
  (\wf{\sigma} : \Gamma\vdash \sigma \Rightarrow \Delta) \ra
  (\wf{\sigma} \sim \model{\sigma}) \ra
  % \\
  % &&& \qquad
  (\wf{\wkO}\wf{A}\wf\sigma \sim \model{\sigma}\model{\circ}\model{\wk})
  \end{alignat*}



%   But to give a precise meaning to the statement that $\wk_n$ (which weakens at
%   the middle of context) preserve the
%   relation, we define (inductively) the notion of semantic telescopes.
%   First, we define the notion of being a prefix:
% \begin{alignat*}{5}
%   & \blank\leq\blank  && : \model\Con\ra \model\Con\ra\Set \\
%   & {\leq}\cdot  && : \model{\Gamma} \leq \model{\Gamma} \\
%   & \blank{\leq}{\ext}\blank  && : \model{\Gamma} \leq \model{\Delta} \ra (A :
%   \model{\Ty}\Delta) \ra \model{\Gamma}\leq\model{\Delta}\model{\ext}\model{A}
% \end{alignat*}
% Now, a telescope on a semantic context $\model{\Gamma}$ is a semantic context
% $\model{\Delta}$ such that $\model{\Gamma}\leq \model{\Delta}$.

% \begin{alignat*}{5}
%   & \Tel\,\model{\Gamma}  && := (\model{\Delta} : \model{\Con})\times (\model{\Gamma}\leq \model{\Delta})
% \end{alignat*}
% Similarly to the untyped case, we define the merging of a context and a telescope by recursion on the telescope:
% \begin{alignat*}{5}
%   & \blank\model{\merge}\blank  && := (\model{\Gamma} : \model{\Con}) \ra
%   (\model{\Delta} : \Tel\,\model\Gamma) \ra \model{\Con} \\
%   & \model\Gamma \merge (\model\Gamma,{\leq}\cdot) && :=  \model\Gamma \\
%   & \model\Gamma \merge (\model\Delta\model{\ext}\model{A}, \Delta_{\leq} {\leq}{\ext} \model{A}) && :=  (\model\Gamma \model\merge \model\Delta)\model\ext \model{A}\\
% \end{alignat*}
% We then define a relevant relation between syntactic telescopes and semantic telescopes.
% Untyped syntactic telescopes are modeled by untyped contexts, because they are
% list of types.
% A syntactic telescope $\Delta$ is then well formed in a context $\Gamma$ if
% $\Gamma\merge\Delta$ is well formed.
% The relation is defined by recursion on the untyped syntactic telescope:
% \begin{alignat*}{5}
%   & \blank\sim\blank  && := ( \Gamma \merge \Delta \vdash) \ra (\model\Gamma\leq\model\Delta)
%   \ra \Set \\
%   & (\wf\Gamma : \Gamma \merge \emptyu \vdash)\sim\model\Delta  && :=
%   \model\Delta =  {\leq}\cdot\\
%   & (\wf\Delta \extw \wf{A} : \Gamma \merge (\Delta \extu A) \vdash)\sim\model{E}  && :=
%   \sum_{\model\Delta} (\wf\Delta\sim\model\Delta)
%   \times
%   \sum_{\model A} (\wf{A}\sim\model{A}) \times
%   \\
%   & && \qquad
%   \model{E} = \model\Delta {\leq}{\ext} \model{A}
% \end{alignat*}

\subsubsection{Stability under substitution}

 We want to prove that given any well-typed substitution $\sigma
: \Sub\,\Gamma\,\Delta$ and semantic contexts $\model{\Gamma}$ and
$\model{\Delta}$, respectively related to $\Gamma$ and $\Delta$, there exists a
semantic substitution which is related to $\sigma$.  In the extension case
$\Gamma\vdash \sigma \consu\,t\Rightarrow \Delta\,\extu\,A$, the induction
hypothesis provides $\model{\sigma}$, $\model{\Delta}$, $\model{A}$ related to
their syntactic counterpart. However, the premises of the induction hypothesis
for getting a relevant $\model{t}$ require showing that the type
$\model{A}\model{[\model{\sigma}]}$ is related to the syntactic type
$A[\sigma]$. We first establish preservation of the relation by substitution for
variables:
\begin{alignat*}{5}
    % []V~
  & []{\sim} && :
  (\wf{\sigma} : \Gamma \vdash \sigma \Rightarrow \Delta) \ra
  (\wf\sigma \sim \model{\sigma}) \ra
  (\wf{x} : \Delta \vdash x \invar A) \ra
  (\wf{x} \sim \model{x}) \ra
  \\ & && \qquad
  % \\
  % & && \qquad
   \wf{[]}\wf{x}\wf{\sigma} \sim \model{x}\model{[\model{\sigma}]}
\end{alignat*}
Then we show it for terms and types  by mutual induction
under the common hypotheses
  $(\wf{\sigma} : \Gamma \vdash \sigma \Rightarrow \Delta)$,
  $(\wf\sigma \sim \model{\sigma})$,
  $(\wf{\Gamma} : \Gamma \vdash )$,
  $(\wf{\Gamma} \sim \model{\Gamma})$,
  % \\
  % &&&\qquad
    $(\wf{\Delta} : \Delta \vdash )$,
    and
  $(\wf{\Delta} \sim \model{\Delta})$:
\begin{alignat*}{5}
     & []{\sim} && :
    (\wf{A} : \Delta \vdash A) \ra
  (\wf{A} \sim \model{A}) \ra
  % \\
  % & && \qquad
   \wf{[]}\wf{\Gamma}\wf{A}\wf{\sigma} \sim \model{A}\model{[\model{\sigma}]}
   \\
     & []{\sim} && :
    (\wf{t} : \Delta \vdash t \in A ) \ra
  (\wf{t} \sim \model{t}) \ra
  % \\
  % & && \qquad
   \wf{[]}\wf{\Gamma}\wf{t}\wf{\sigma} \sim \model{t}\model{[\model{\sigma}]}
  \end{alignat*}
  Eventually, we show under the same hypotheses the following statement
  \begin{alignat*}{5}
     & {\circ}{\sim} && :
    (\wf{E} : E \vdash ) \ra
  (\wf{E} \sim \model{E}) \ra
  % \\
  % &&&  \qquad
    (\wf{\delta} : \Delta \vdash \delta \Rightarrow E) \ra
  (\wf{\delta} \sim \model{\delta}) \ra
  \\
  & && \qquad
   \wf{{\circ}}\wf{\Gamma}\wf{\delta}\wf{\sigma} \sim \model{\delta}\model{\circ}\model{\sigma}
  \end{alignat*}
  and the fact that identity preserves the relation:
  \begin{alignat*}{5}
     & {\id}{\sim} && :
    (\wf{\Gamma} : \Gamma \vdash ) \ra
  (\wf{\Gamma} \sim \model{\Gamma}) \ra
   \wf{{\id}}\wf{\Gamma} \sim {\id}_{\model{\Gamma}}
  \end{alignat*}
% $\model{t}\model\oldapp\model{u}$, but it has type
% $\model{B}\model{[0 := \model{u}]}$
% instead of type $\model{T}$. Fortunately, we know by hypothesis that $\model{T}$
% relates to $T=B[0:=u]$ so, by uniqueness of the related semantic counterpart,
% we can deduce that $\model{T}=\m$
% Thus, we need to show that $\model{T}=\model{B}\model{[0 := \model{u}]}$ if we
% know that this last type is related to $B[0 := u]$.


\subsection{The Recursor}
For the recursor, we build a morphism from the syntactic model to the semantic
one. The image of a syntactic context is a unique semantic context which is
related to it, and similarly for types and terms.
% We construct it as the semantic
% Induction on the typing judgments shows that any QIIT morphism from the
% syntactic model $S$  to a semantic model $M$ sends well-typed syntax on a related
% semantic counter-part.
Thus, as a first step, we use induction on well-formedness judgments to construct
semantic counterparts:
  \begin{alignat*}{5}
     & \Sigma\Con{\sim} && :
    (\wf{\Gamma} : \Gamma \vdash ) \ra
    \sum_{\model{\Gamma}} \wf{\Gamma} \sim \model{\Gamma}
    \\
     & \Sigma\Ty{\sim} && :
    (\wf{\Gamma} : \Gamma \vdash ) \ra
    (\wf{\Gamma}\sim \model\Gamma) \ra
    (\wf{A} : \Gamma \vdash A ) \ra
    ({\model{A}:\model{\Ty}\model\Gamma})\times ( \wf{A} \sim \model{A})
    \\
     & \Sigma\Tm{\sim} && :
    (\wf{\Gamma} : \Gamma \vdash ) \ra
    (\wf{\Gamma}\sim \model\Gamma) \ra
    (\wf{A} : \Gamma \vdash A ) \ra
    ( \wf{A} \sim \model{A}) \ra
    \\ & && \qquad
    (\wf{t} : \Gamma \vdash t \in A ) \ra
    ({\model{t}:\model{\Tm}\model\Gamma}\model{A})\times ( \wf{t} \sim \model{t})
    \\
     & \Sigma\Var{\sim} && :
    (\wf{\Gamma} : \Gamma \vdash ) \ra
    (\wf{\Gamma}\sim \model\Gamma) \ra
    (\wf{A} : \Gamma \vdash A ) \ra
    ( \wf{A} \sim \model{A}) \ra
    \\ & && \qquad
    (\wf{x} : \Gamma \vdash x \invar A ) \ra
    ({\model{x}:\model{\Tm}\model\Gamma}\model{A})\times ( \wf{x} \sim \model{x})
    \\
     & \Sigma\Sub{\sim} && :
    (\wf{\Gamma} : \Gamma \vdash ) \ra
    (\wf{\Gamma}\sim \model\Gamma) \ra
    (\wf{\Delta} : \Delta \vdash  ) \ra
    ( \wf{\Delta} \sim \model{\Delta}) \ra
    \\ & && \qquad
    (\wf{\sigma} : \Gamma \vdash \sigma \Rightarrow \Delta ) \ra
    ({\model{\sigma}:\model{\Sub}\model\Gamma}\model{\Delta})\times ( \wf{\sigma} \sim \model{\sigma})
  \end{alignat*}
  The right uniqueness of the relation is used in this induction. It is then
straightforward to show (without induction) that the first projection of these
constructions yield a model morphism from the syntax to the model, again using
right uniqueness.
\subsection{Uniqueness}
Our goal is to show that the syntactic model is initial. Thus, it remains to
show that the morphism constructed in the previous section is unique. We exploit
right uniqueness of the relation: it is enough to show that any such morphism
maps a syntactic context to a related semantic context, and similarly for types
and terms.

More formally, we assume an arbitrary morphism from the
syntax to the model, inducing the following maps:
\begin{alignat*}{5}
  &
  \mor{\Con}
  && :
   (\Gamma \vdash) \ra \model\Con
   \\
  &
  \mor{\Ty}
  && :
   (\wf\Gamma:\Gamma \vdash) \ra (\Gamma\vdash A)\ra\model\Ty\,(\mor\Con\wf\Gamma)
   \\
  &
  \mor{\Tm}
  && :
  (\wf\Gamma:\Gamma \vdash) \ra (\wf{A}:\Gamma\vdash A)\ra
  (\Gamma\vdash t \in A)\ra\model\Tm\,(\mor\Con\wf\Gamma)\,
  (\mor\Ty\wf\Gamma\,\wf{A})
   \\
  &
  \mor{\Var}
  && :
  (\wf\Gamma:\Gamma \vdash) \ra (\wf{A}:\Gamma\vdash A)\ra
  (\Gamma\vdash x \invar A)\ra\model\Tm\,(\mor\Con\wf\Gamma)\,
  (\mor\Ty\wf\Gamma\,\wf{A})
   \\
  &
  \mor{\Sub}
  && :
  (\wf\Gamma:\Gamma \vdash) \ra
  (\wf\Delta:\Delta \vdash) \ra
  (\Gamma\vdash \sigma \Rightarrow \Delta)\ra\model\Sub\,(\mor\Con\wf\Gamma)\,
  (\mor\Con\wf\Delta)
\end{alignat*}
Then, we show by induction on typing judgments that
the image is related:
\begin{alignat*}{5}
  &
  {\sim}\mor{\Con}
  && :
  (\wf\Gamma : \Gamma \vdash) \ra \wf\Gamma\sim \mor\Con\,\wf\Gamma
  \\
  &
  {\sim}\mor{\Ty}
  && :
  (\wf\Gamma : \Gamma \vdash) \ra
  (\wf{A} : \Gamma \vdash A) \ra
  \wf\Gamma\sim \mor\Ty\,\wf\Gamma\,\wf{A}
  \\
  &
  {\sim}\mor{\Tm}
  && :
  (\wf\Gamma : \Gamma \vdash) \ra
  (\wf{A} : \Gamma \vdash A) \ra
  (\wf{t} : \Gamma \vdash t \in A) \ra
  \wf\Gamma\sim \mor\Tm\,\wf\Gamma\,\wf{A}\,\wf{t}
  \\
  &
  {\sim}\mor{\Var}
  && :
  (\wf\Gamma : \Gamma \vdash) \ra
  (\wf{A} : \Gamma \vdash A) \ra
  (\wf{x} : \Gamma \vdash x \invar A) \ra
  \wf\Gamma\sim \mor\Var\,\wf\Gamma\,\wf{A}\,\wf{x}
  \\
  &
  {\sim}\mor{\Sub}
  && :
  (\wf\Gamma : \Gamma \vdash) \ra
  (\wf{\Delta} : \Delta \vdash) \ra
  (\wf{\sigma} : \Gamma \vdash \sigma \Rightarrow \Delta) \ra
  \wf\Gamma\sim \mor\Sub\,\wf\Gamma\,\wf{\Delta}\,\wf{\sigma}
\end{alignat*}
This justifies the uniqueness of the morphism, by right uniqueness of
$\blank\sim\blank$.


% \begin{description}
%   \item[untyped syntax and well typed judgments as an algebra]
%   \begin{enumerate}
%   \item define untyped syntax as an inductive datatype
%   \item define operations on the syntax, such as substitution, by recursion
%   \item define well-typed judgments as an inductive datatype indexed over the untyped syntax
%   \item show the dependant pair of untyped syntax with well-typed judgments
%     define an algebra
%   \end{enumerate}
%   \item[specification of the initial morphism as a functional relation]
%   \begin{enumerate}
%   \item given an algebra, define the functional relation enjoyed by the recursor
%     from the syntax to the algebra by recursion over the well-typed judgment
%   \item show right-uniqueness of the relation
%   \item show left-totality of the relation
%   \end{enumerate}
%   \item[existence and uniqueness of the morphism from the syntax to a model]
%     \begin{enumerate}
%   \item show uniqueness of such a morphism
%   \item extract an algebra morphism from the relation
%     \end{enumerate}
% \end{description}


\section{Constructing IITs from the Theory of IIT Signatures}
\label{sec:andras}

We construct all IITs describable by the theory of IIT signatures through the
term model construction of \cite{Kaposi:2019:CQI:3302515.3290315}. There, it is
shown that from the syntax of the theory of signatures for finitary QIITs, one
can construct any particular finitary QIIT. The idea is that for a signature
$\Gamma$, the initial algebra can be built from sets of terms of the form
$\Tm\,\Gamma\,(\El\,a)$.

In this section, we only need to check that the signatures in
\cite{Kaposi:2019:CQI:3302515.3290315} and the term model construction restricts
to IITs, i.e.\ that from a syntax for the theory of IIT signatures all finitary
IITs are constructible.

\begin{enumerate}
\item
  Finitary IIT signatures are obtained exactly by dropping equality constructors from
  finitary QIIT signatures. It follows that any model for the theory of QIIT signatures
  restricts to IITs, hence we inherit the categorical semantics of QIITs.
\item
  The term model construction also restricts in a straightforward way. For QIITs,
  term algebras are constructed by induction on the syntax of signatures, and
  then another induction constructs the eliminator (i.e.\ dependent induction
  principle). In both cases, simply dropping equality constructors from the
  induction yields restriction to IITs.
\end{enumerate}

However, we shall mention that the Agda formalisation for the current paper
cannot be directly plugged into the Agda code for
\cite{Kaposi:2019:CQI:3302515.3290315} (which includes most of the term model
construction). One reason is that the QIIT formalisation uses strict computation
rules for induction over signatures, while here we show propositional
computation. This mismatch can be in principle remedied by noting that both
formalisations use UIP and function extensionality, hence we can switch between
strict and propositional equalities, via the known translations between
extensional and intensional type
theories\cite{hofmann95extensional,winterhalter2019eliminating}.

Also, the current paper proves initiality, i.e.\ unique recursion for
signatures, while \cite{Kaposi:2019:CQI:3302515.3290315} uses dependent
elimination. We expect that the two notions are equivalent. An extension of
\cite{Kaposi:2019:CQI:3302515.3290315} to large infinitary QIITs would entail
this equivalence, since that would cover the (large, infinitary) theory of IIT
signatures.


\section{Further Work}

\subsection{Infinitary IITs}.

The current work only concerns finitary IITs. An extension would be to also
allow infinitely branching inductive types such as W-types. This would first
require giving semantics for infinitary IITs (to our knowledge there is no
previously published semantics that we can borrow), and also giving a term model
construction analogously to finitary QIITs. These steps seem feasible. However,
it seems to be more difficult to construct the syntax of infinitary IIT
signatures without using quotients. The reason is that such syntax would not be
strictly restricted to neutral terms: we would need $\lambda$-abstraction and
$\beta\eta$-rules for infinitary $\Pi$ types, in order to allow a term model
construction for infinitary IITs. A definition for normal preterms and typing
judgments on them may still be possible, but it appears to be much more
complicated than before (the current authors attempted this without conclusive
success).

\todo{Without UIP, without funext,
for QIITs (in the setoid model).}


\bibliography{b}

\end{document}
