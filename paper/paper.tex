
\documentclass[a4paper,UKenglish,cleveref, autoref]{lipics-v2019}
%This is a template for producing LIPIcs articles.
%See lipics-manual.pdf for further information.
%for A4 paper format use option "a4paper", for US-letter use option "letterpaper"
%for british hyphenation rules use option "UKenglish", for american hyphenation rules use option "USenglish"
%for section-numbered lemmas etc., use "numberwithinsect"
%for enabling cleveref support, use "cleveref"
%for enabling cleveref support, use "autoref"
\usepackage{amssymb}
\usepackage{amsmath}
\usepackage{amsthm}
\usepackage{hyperref}
\usepackage{todonotes}
\presetkeys{todonotes}{inline}{}
\usepackage{csquotes} %for \llbracket and \rrbracket
\usepackage{stmaryrd}
% \usepackage{pdflscape} % if we want \begin{landscape} ... \end{landscape}

%\graphicspath{{./graphics/}}%helpful if your graphic files are in another directory

\bibliographystyle{plainurl}% the mandatory bibstyle

\title{For Finitary Induction-Induction, \\ Induction is Enough} %TODO Please add

\titlerunning{For Finitary Induction-Induction, Induction is Enough}%optional, please use if title is longer than one line

\author{Ambrus Kaposi}{E{\"o}tv{\"o}s Lor{\'a}nd University, Budapest, Hungary}{akaposi@inf.elte.hu}{https://orcid.org/0000-0001-9897-8936}{this author was supported by the National Research,
Development and Innovation Fund of Hungary, financed under the
Thematic Excellence Programme funding scheme, Project
no.\ ED18-1-2019-0030 (Application-specific highly reliable IT
solutions), by the New National Excellence Program of the Ministry
for Innovation and Technology, Project no.\ ÚNKP-19-4-ELTE-874, and
by the Bolyai Fellowship of the Hungarian Academy of Sciences,
Project no.\ BO/00659/19/3.}%TODO mandatory, please use full name; only 1 author per \author macro; first two parameters are mandatory, other parameters can be empty. Please provide at least the name of the affiliation and the country. The full address is optional
\author{Andr{\'a}s Kov{\'a}cs}{E{\"o}tv{\"o}s Lor{\'a}nd University, Budapest, Hungary}{kovacsandras@inf.elte.hu}{https://orcid.org/0000-0002-6375-9781}{this author was supported by the European Union, co-financed by the European Social Fund (EFOP-3.6.3-VEKOP-16-2017-00002).}
\author{Ambroise Lafont}{IMT Atlantique, Inria, LS2N CNRS, Nantes, France}{ambroise.lafont@inria.fr}{https://orcid.org/0000-0002-9299-641X}{}

\authorrunning{A. Kaposi, A. Kov{\'a}cs and A. Lafont}%TODO mandatory. First: Use abbreviated first/middle names. Second (only in severe cases): Use first author plus 'et al.'

\Copyright{A. Kaposi, A. Kov{\'a}cs and A. Lafont}%TODO mandatory, please use full first names. LIPIcs license is "CC-BY";  http://creativecommons.org/licenses/by/3.0/

\ccsdesc[500]{Theory of computation~Logic~Type theory}%TODO mandatory: Please choose ACM 2012 classifications from https://dl.acm.org/ccs/ccs_flat.cfm

\keywords{type theory, inductive types, inductive-inductive types}%TODO mandatory; please add comma-separated list of keywords

\category{}%optional, e.g. invited paper

\relatedversion{}%optional, e.g. full version hosted on arXiv, HAL, or other respository/website
%\relatedversion{A full version of the paper is available at \url{...}.}

\supplement{}%optional, e.g. related research data, source code, ... hosted on a repository like zenodo, figshare, GitHub, ...

%\funding{(Optional) general funding statement \dots}%optional, to capture a funding statement, which applies to all authors. Please enter author specific funding statements as fifth argument of the \author macro.

\acknowledgements{The authors would like to thank Thorsten Altenkirch, Rafa{\"e}l Bocquet, Simon Boulier, Fredrik Nordvall-Forsberg and Jakob von Raumer for discussions on the topics of this paper. We also thank the anonymous reviewers for their helpful comments and suggestions.}%optional

%\nolinenumbers %uncomment to disable line numbering

%\hideLIPIcs  %uncomment to remove references to LIPIcs series (logo, DOI, ...), e.g. when preparing a pre-final version to be uploaded to arXiv or another public repository

%Editor-only macros:: begin (do not touch as author)%%%%%%%%%%%%%%%%%%%%%%%%%%%%%%%%%%
\EventEditors{John Q. Open and Joan R. Access}
\EventNoEds{2}
\EventLongTitle{42nd Conference on Very Important Topics (CVIT 2016)}
\EventShortTitle{CVIT 2016}
\EventAcronym{CVIT}
\EventYear{2016}
\EventDate{December 24--27, 2016}
\EventLocation{Little Whinging, United Kingdom}
\EventLogo{}
\SeriesVolume{42}
\ArticleNo{23}
%%%%%%%%%%%%%%%%%%%%%%%%%%%%%%%%%%%%%%%%%%%%%%%%%%%%%%

%ppp metatheory
\newcommand{\blank}{\mathord{\hspace{1pt}\text{--}\hspace{1pt}}} %from the book
\newcommand{\ra}{\rightarrow}
\newcommand{\Set}{\mathsf{Set}}
\newcommand{\Prop}{\mathsf{Prop}}

% object theory: universal QIIT
\newcommand{\Con}{\mathsf{Con}}
\newcommand{\Ty}{\mathsf{Ty}}
\newcommand{\Sub}{\mathsf{Sub}}
\newcommand{\Tm}{\mathsf{Tm}}
\newcommand{\Var}{\mathsf{Var}}
\newcommand{\id}{\mathsf{id}}
\newcommand{\ass}{\mathsf{ass}}
\newcommand{\idl}{\mathsf{idl}}
\newcommand{\idr}{\mathsf{idr}}
\newcommand{\ext}{\rhd}
\newcommand{\p}{\mathsf{p}}
\newcommand{\q}{\mathsf{q}}
\newcommand{\U}{\mathsf{U}}
\newcommand{\El}{\mathsf{El}}
\newcommand{\app}{\mathsf{app}}
\newcommand{\oldapp}{\mathop{{\scriptstyle @}}}
\newcommand{\Pim}{\hat{\Pi}}
\newcommand{\appm}{\mathop{\hat{{\scriptstyle @}}}}
\newcommand{\Pii}{\tilde{\Pi}}
\newcommand{\appi}{\mathop{\tilde{{\scriptstyle @}}}}
\newcommand{\Id}{\mathsf{Id}}
\newcommand{\reflect}{\mathsf{reflect}}

% untyped syntax
\newcommand{\untyp}[1]{{#1}^{\mathsf{p}}}
\newcommand{\Conu}{\untyp{\Con}}
\newcommand{\Tyu}{\untyp{\Ty}}
\newcommand{\Subu}{\untyp{\Sub}}
\newcommand{\Tmu}{\untyp{\Tm}}
\newcommand{\idu}{\untyp{\id}}
\newcommand{\emptyu}{\untyp{\cdot}}
\newcommand{\epsilonu}{\untyp{\epsilon}}
\newcommand{\extu}{\mathbin{\untyp{\ext}}}
\newcommand{\consu}{\mathbin{\untyp{\cons}}}
\newcommand{\varu}{\untyp{\mathsf{var}}}
\newcommand{\Uu}{\untyp{\U}}
\newcommand{\Elu}{\untyp{\El}}
\newcommand{\Piu}{\untyp{\Pi}}
\newcommand{\appu}{\mathbin{\untyp{{\scriptstyle @}}}}
\newcommand{\Pipu}{\untyp{\Pi}}
\newcommand{\Pimu}{\untyp{\Pim}}
\newcommand{\Piiu}{\untyp{\Pii}}
\newcommand{\appmu}{\mathop{\hat{\tilde{{\scriptstyle @}}}}}
\newcommand{\erru}{\untyp{\mathsf{err}}}
\newcommand{\wku}{\untyp{\wk}}
\newcommand{\idru}{\untyp{\idr}}
\newcommand{\idlu}{\untyp{\idl}}

% well-typed judgement
\newcommand{\wf}[1]{{#1}^{\mathsf{w}}}
\newcommand{\Conw}{\wf{\Con}}
\newcommand{\Tyw}{\wf{\Ty}}
\newcommand{\Subw}{\wf{\Sub}}
\newcommand{\Tmw}{\wf{\Tm}}
\newcommand{\invar}{\in_\N}
\newcommand{\idw}{\wf{\id}}
\newcommand{\emptyw}{\wf{\cdot}}
\newcommand{\extw}{\mathbin{\wf{\ext}}}
\newcommand{\consw}{\wf{\cons}}
\newcommand{\epsilonw}{\wf{\epsilon}}
\newcommand{\varw}{\wf{\mathsf{var}}}
\newcommand{\vzw}{\wf{\mathsf{0}}}
\newcommand{\vsw}{\wf{\mathsf{S}}}
\newcommand{\Uw}{\wf{\U}}
\newcommand{\Elw}{\wf{\El}}
\newcommand{\Piw}{\wf{\Pi}}
\newcommand{\appw}{\wf{\app}}
\newcommand{\Pipw}{\wf{\Pi}}
\newcommand{\Pimw}{\wf{\Pim}}
\newcommand{\appmw}{\wf{\hat{\app}}}
\newcommand{\Piiw}{\wf{\Pii}}
\newcommand{\appiw}{\wf{\tilde{\app}}}

% model
\newcommand{\model}[1]{{#1}^M}

% morphism of model
\newcommand{\mor}[1]{{#1}^f}

% syntax model
\newcommand{\syn}[1]{{#1}^{\mathsf{S}}}

% functional relation
\newcommand{\rel}[1]{{#1}^r}
\newcommand{\Conr}{\rel{\Con}}
\newcommand{\Tyr}{\rel{\Ty}}
\newcommand{\Subr}{\rel{\Sub}}
\newcommand{\Tmr}{\rel{\Tm}}
\newcommand{\emptyr}{\rel{\cdot}}
\newcommand{\extr}{\rel{\ext}}
\newcommand{\varr}{\rel{\mathsf{var}}}
\newcommand{\vzr}{\rel{0}}
\newcommand{\vsr}{\rel{S}}
\newcommand{\Ur}{\rel{\U}}
\newcommand{\Elr}{\rel{\El}}
\newcommand{\Pir}{\rel{\Pi}}
\newcommand{\appr}{\rel{\app}}
\newcommand{\Pipr}{\rel{\Pi}}
\newcommand{\Pimr}{\rel{\Pim}}
\newcommand{\appmr}{\rel{\hat{\app}}}
\newcommand{\Piir}{\rel{\Pii}}
\newcommand{\appir}{\rel{\tilde{\app}}}

\newcommand{\relT}[3]{#1 \sim_{#2} #3}
\newcommand{\relt}[4]{#1 \sim_{#2\vdash #3} #4}
\newcommand{\relV}[4]{#1 \sim_{#2\vdash #3} #4}
\newcommand{\relS}[4]{#1 \sim_{#2\Rightarrow #3} #4}




\newcommand{\A}{\mathsf{A}}
\newcommand{\F}{\mathsf{F}}
\renewcommand{\S}{\mathsf{S}}
\renewcommand{\O}{\mathsf{O}}
\newcommand{\M}{\mathsf{M}}
\newcommand{\con}{\mathsf{con}}
\newcommand{\elim}{\mathsf{elim}}
\newcommand{\nat}{\mathsf{nat}}
\newcommand{\map}{\mathsf{map}}
\newcommand{\List}{\mathsf{List}}
\newcommand{\lb}{\langle}
\newcommand{\rb}{\rangle}
\newcommand{\wk}{\mathsf{wk}}
\newcommand{\wkO}{\wk_0}
\newcommand{\keep}{\mathsf{keep}}
\newcommand{\cons}{,}
\newcommand{\nil}{\mathsf{nil}}
\renewcommand{\merge}{;}
\newcommand{\length}[1]{\| #1 \|}
\newcommand{\vz}{\mathsf{vz}}
\newcommand{\vs}{\mathsf{vs}}
\newcommand{\Ra}{\Rightarrow}
\renewcommand{\tt}{\mathsf{tt}}
\newcommand{\proj}{\mathsf{proj}}
\newcommand{\refl}{\mathsf{refl}}
\newcommand{\J}{\mathsf{J}}
\newcommand{\tr}{\mathsf{tr}}
\newcommand{\trans}{\mathbin{\raisebox{0.2ex}{$\displaystyle\centerdot$}}}
\newcommand{\ap}{\mathsf{ap}}
\newcommand{\apd}{\mathsf{apd}}
\newcommand{\rec}{\mathsf{rec}}
\newcommand{\C}{\mathsf{C}}
\newcommand{\R}{\mathsf{R}}
\newcommand{\E}{\mathsf{E}}
\newcommand{\transp}{\mathsf{transp}}
\newcommand{\Ram}{\mathbin{\hat{\Ra}}}
\newcommand{\funext}{\mathsf{funext}}
\newcommand{\UIP}{\mathsf{UIP}}
\newcommand{\coe}{\mathsf{coe}}
\newcommand{\LET}{\mathsf{let}}
\newcommand{\IN}{\mathsf{in}}
\newcommand{\N}{\mathbb{N}}
\newcommand{\D}{\mathsf{D}}
\newcommand{\K}{\mathsf{K}}
\newcommand{\Eq}{\mathsf{Eq}}
\newcommand{\mk}{\mathsf{mk}}
\newcommand{\unk}{\mathsf{unk}}
\newcommand{\0}{\mathit{0}}
\newcommand{\1}{\mathit{1}}
\newcommand{\eqreflect}{\mathsf{eqreflect}}

\newcommand{\isaprop}{\mathsf{is\text{-}prop}}
\newcommand{\isp}[1]{{#1}^{\mathsf{p}}}

\mathchardef\mhyphen="2D

\renewcommand{\ll}{\llbracket}
\newcommand{\rr}{\rrbracket}

\newcommand{\SignAlg}{\mathsf{SignAlg}}
\newcommand{\SignMor}{\mathsf{SignMor}}
\newcommand{\Sign}{\mathsf{Sign}}
\newcommand{\ADS}{\mathsf{ADS}}
\newcommand{\Bool}{\mathsf{Bool}}
\newcommand{\I}{\mathsf{I}}
\newcommand{\IE}{\mathsf{IE}}



\allowdisplaybreaks

\begin{document}
\maketitle

\begin{abstract}
  TODO: rewrite. Inductive-inductive types (IITs) are a generalisation of inductive types in
  type theory. They allow the mutual definition of types with multiple sorts
  where later sorts can be indexed by previous ones. An example is the
  Chapman-style syntax of type theory with conversion relations for each sort
  where e.g.\ the sort of types is indexed by contexts. It follows from previous
  work by Altenkirch, Kaposi, and Kov{\'a}cs, that all finitary IITs can be constructed from a quotient
  inductive-inductive type (QIIT), namely the theory of IIT signatures. This is
  a small domain-specific type theory where a context is a signature for an
  IIT. In this paper we construct the theory of IIT signatures using only inductive types,
  thereby showing a reduction of all finitary IITs to inductive types.  We rely
  on function extensionality and uniqueness of identity proofs. We formalised
  the construction of the theory of IIT signatures in Agda. TODO: rewrite
\end{abstract}

%%%%%%%%%%%%%%%%%%%%%%%%%%%%%%%%%%%%%%%%%%%%%%%%%%%%%%%%%%%%%%%%%%%%%%%%%%%

\section{Introduction}
\label{sec:intro}

Many mutual inductive types can be reduced to indexed inductive types, where the
index disambiguates different sorts. For example, consider the mutual inductive datatype
with two sorts
$\mathsf{isEven}$ and $\mathsf{isOdd}$, defined by the following
constructors.
\begin{alignat*}{5}
  & \mathsf{isEven} && : \N\ra\Set \\
  & \mathsf{isOdd} && : \N\ra \Set \\
  & \mathsf{zeroEven} && : \mathsf{isEven}\,\mathsf{zero} \\
  & \mathsf{sucEven} && : (n:\N)\ra\mathsf{isOdd}\,n\ra\mathsf{isEven}\,(\mathsf{suc}\,n) \\
  & \mathsf{sucOdd} && : (n:\N)\ra\mathsf{isEven}\,n\ra\mathsf{isOdd}\,(\mathsf{suc}\,n)
\end{alignat*}
This can be reduced to the following single inductive family where
$\mathsf{isEven?}\,\mathsf{true}$ represents $\mathsf{isEven}$ and
$\mathsf{isEven?}\,\mathsf{false}$ represent $\mathsf{isOdd}$.
\begin{alignat*}{5}
  & \mathsf{isEven?} && : \mathsf{Bool}\ra\N\ra\Set \\
  & \mathsf{zeroEven} && : \mathsf{isEven?}\,\mathsf{true}\,\mathsf{zero} \\
  & \mathsf{sucEven} && : (n:\N)\ra\mathsf{isEven?}\,\mathsf{false}\,n\ra\mathsf{isEven?}\,\mathsf{true}\,(\mathsf{suc}\,n) \\
  & \mathsf{sucOdd} && : (n:\N)\ra\mathsf{isEven?}\,\mathsf{true}\,n\ra\mathsf{isEven?}\,\mathsf{false}\,(\mathsf{suc}\,n)
\end{alignat*}

Inductive-inductive types (IITs \cite{forsberg-phd}) allow the mutual
definition of a type and a family of types over the first one. IITs
were originally introduced to represent the well-typed syntax of type
theory itself, and the main example is still the Chapman-style
\cite{chapman09eatitself} syntax for a type theory. A minimised version
is the IIT of contexts and types given by the following constructors.
\begin{alignat*}{5}
  & \Con && : \Set \\
  & \Ty && : \Con\ra\Set \\
  & \mathsf{empty} && : \Con \\
  & \mathsf{ext} && : (\GG:\Con)\ra\Ty\,\GG\ra\Con \\
  & \mathsf{U} && : (\GG:\Con)\ra\Ty\,\GG \\
  & \mathsf{El} && : (\GG:\Con)\ra\Ty\,(\mathsf{ext}\,\GG\,(\mathsf{U}\,\GG))
%  & \mathsf{Pi} && : (\GG:\Con)(A:\Ty\,\GG)\ra\Ty\,(\mathsf{ext}\,\GG\,A)\ra\,\Ty\,\GG
\end{alignat*}
This type has two sorts, $\Con$ and $\Ty$. The $\mathsf{ext}$ constructor of
$\Con$ refers to $\Ty$ and the $\Ty$-constructor $\mathsf{U}$ refers to $\Con$,
hence the two sorts have to be defined simultaneously. Moreover, $\Ty$ is
indexed over $\Con$. This precludes a reduction analogous to the reduction of
$\mathsf{isEven}$--$\mathsf{isOdd}$, as we would get a type indexed over
itself. Another unique feature of IITs (which also holds for higher inductive
types \cite{HoTTbook}) is that later constructors can refer to previous
constructors: in our case, $\mathsf{El}$ mentions
$\mathsf{ext}$.
% \[
%   \mathsf{Con?} : (b:\mathsf{Bool})\ra\mathsf{if}\,b\,\mathsf{then}\,\Set\,\mathsf{else}\,(\mathsf{Con?}\,\mathsf{false}\ra\Set),
% \]

The elimination principle for the above IIT has the following two
motives (one for each sort) and four methods (one for each
constructor).
\begin{alignat*}{5}
  & Con^D && : \Con\ra\Set \\
  & Ty^D && : Con^D\,\GG\ra\Ty\,\GG\ra\Set \\
  & {empty}^D && : Con^D\,\mathsf{empty} \\
  & {ext}^D && : (\GG^D:Con^D\,\GG)\ra Ty^D\,\GG^D\,A\ra Con^D\,(\mathsf{ext}\,\GG\,A) \\
  & U^D && : (\GG^D:Con^D\,\GG)\ra Ty^D\,\GG^D\,(\mathsf{U}\,\GG) \\
  & El^D && : (\GG^D:Con^D\,\GG)\ra\,Ty^D\,({ext}^D\,\GG^D\,(U^D\,\GG^D))\,(\mathsf{El}\,\GG)
%  & Pi^D && : (\GG^D:Con^D\,\GG)(A^D:\Ty^D\,\GG^D\,A)\ra \Ty^D\,(\mathsf{ext}^D\,\GG^D\,A^D)\ra\,\Ty^D\,\GG^D\,(\mathsf{Pi}\,\GG\,A\,B)
\end{alignat*}
Above we used implicit quantifications for $\GG:\Con$ and
$A:\Ty\,\GG$ to ease readability, e.g.\ $Ty^D$ has an implicit
parameter $\GG$ before its explicit parameter of type
$Con^D\,\GG$.

Given the above motives and methods the elimination principle provides
two functions
\begin{alignat*}{5}
  & \mathsf{elimCon} && : (\GG:\Con)\ra Con^D\,\GG \\
  & \mathsf{elimTy} && : (A:\Ty\,\GG)\ra Ty^D\,(\mathsf{elimCon}\,\GG)\,A
\end{alignat*}
with the following computation rules.
\begin{alignat*}{5}
  & \mathsf{elimCon}\,\mathsf{empty} && = {empty}^D \\
  & \mathsf{elimCon}\,(\mathsf{ext}\,\GG\,A) && = {ext}^D\,(\mathsf{elimCon}\,\GG)\,(\mathsf{elimTy}\,A) \\
  & \mathsf{elimTy}\,(\mathsf{U}\,\GG) && = U^D\,(\mathsf{elimCon}\,\GG) \\
  & \mathsf{elimTy}\,(\mathsf{El}\,\GG) && = El^D\,(\mathsf{elimCon}\,\GG)
\end{alignat*}
The functions $\mathsf{elimCon}$ and $\mathsf{elimTy}$ are an example
of a \emph{recursive-recursive} definition (using nomenclature from
\cite{forsberg-phd}). This means two mutually defined functions where
the type of the second function depends on the first function. The
proof assistant Agda \cite{norell07thesis} allows defining such
functions (even from non-IITs) and it is currently the only proof
assistant supporting IITs\footnote{An experimental version of Coq with
  IITs is also available on GitHub.}.

Reducing IITs to inductive types (more precisely, to indexed W-types)
is an open problem. Forsberg \cite{forsberg-phd} showed a reduction in
extensional type theory however this only provides a simpler,
non-recursive-recursive elimination principle. Hugunin \cite{jasper}
reduced several IITs to inductive types, working inside a cubical type
theory, but he also only constructed the simple eliminator. To
illustrate the difference, we list the motives and methods and the
simple elimination principle for the $\Con$--$\Ty$ example. Again, we
use implicit quantifications.
\begin{alignat*}{5}
  & Con^S && : \Con\ra\Set \\
  & Ty^S && : \Ty\,\GG\ra\Set \\
  & {empty}^S && : Con^S\,\mathsf{empty} \\
  & {ext}^S && : Con^S\,\GG\ra Ty^S\,\,A\ra Con^S\,(\mathsf{ext}\,\GG\,A) \\
  & U^S && : Con^S\,\GG\ra Ty^S\,(\mathsf{U}\,\GG) \\
  & El^S && : Con^S\,\GG\ra\,Ty^S\,(\mathsf{El}\,\GG) \\
  & \mathsf{selimCon} && : (\GG:\Con)\ra Con^S\,\GG \\
  & \mathsf{selimTy} && : (A:\Ty\,\GG)\ra Ty^S\,A
\end{alignat*}
This simple elimination principle is too weak e.g.\ to define the
standard (metacircular) interpretation of our small syntax
\cite{ttintt}. Using pattern matching notation, this interpretation is
the following:
\begin{alignat*}{5}
  & \ll\blank\rr && : \Con\ra\Set_1 \\
  & \ll\blank\rr && : \ll\GG\rr\ra\Set_1 \\
  & \ll\mathsf{empty}\rr && := \top \\
  & \ll\mathsf{ext}\,\GG\,A\rr && := (\gamma:\ll\GG\rr)\times\ll A\rr\,\gamma \\
  & \ll\mathsf{U}\,\GG\rr\,\gamma && := \Set \\
  & \ll\mathsf{El}\,\GG\rr\,(\gamma,X) && := X
\end{alignat*}
The reason that we need the general elimination principle to define
$\ll\blank\rr$ is that $\ll\blank\rr$ for types refers to
$\ll\blank\rr$ for contexts, hence this function is
recursive-recursive.

Kaposi, Kov{\'a}cs, and Altenkirch
\cite{Kaposi:2019:CQI:3302515.3290315} introduced a small type theory
called the theory of signatures to describe quotient
inductive-inductive types (QIIT). QIITs are generalisations of IITs
where equality constructors are also allowed. A QIIT signature is a
context in the theory of QIIT signatures, for example natural numbers
are specified by the context $(Nat:\U,zero:Nat,suc:Nat\ra Nat)$ of
length three ($Nat$, $zero$ and $suc$ are variable names). The theory
of QIIT signatures is itself a QIIT. In ibid., it is proved that if a
model of extensional type theory supports the theory of QIIT
signatures, then it supports all QIITs.

By omitting the equality type former from the theory of QIIT
signatures, we obtain a theory of IIT signatures and the construction
is still valid. It follows that if a model of extensional type theory
supports the theory of IIT signatures, it supports all IITs.

In this paper we show that any model of extensional type theory with
indexed W-types supports the theory of IIT signatures, and as a
consequence all IITs. The difficulty in this construction is that the
theory of IIT signatures is itself a QIIT, it is both
inductive-inductive and has equality constructors. However it can be
seen as the well-typed syntax of a small type theory without any
computation rules. Hence we can represent the syntax of normal forms
without quotienting. We construct this well-typed normal syntax using
preterms and typing relations from indexed W-types. Finally, we prove
the elimination principle in the style of the initiality proof of
Streicher \cite{streichersemantics}.

Just as \cite{Kaposi:2019:CQI:3302515.3290315}, we only consider
finitary IITs, that is, constructors can only have a finite number of
recursive arguments. An example constructor for $\Con$--$\Ty$ which is
not allowed is the following:
\[
  \Pi_\infty : (\GG : \Con) \rightarrow (\mathbb{N} \rightarrow \Ty\,\GG) \rightarrow \Ty\,\GG
\]

\paragraph*{Structure of paper and list of contributions}

We describe related work in \Cref{sec:related}, and explain our
notation and Agda formalisaton in \Cref{sec:notation}. Then the
following three sections describe our three contributions:
\begin{itemize}
\item \Cref{sec:theory_of_signatures}. We define what it means for a
  model of extensional type theory (ETT, \Cref{def:ett}) to support
  all inductive-inductive types (IITs): \Cref{def:hasiits}. The novel
  contribution here is a (predicative) Church encoding of signatures
  following \cite{DBLP:conf/lics/AwodeyFS18}.
\item \Cref{sec:constructingiits}. We show that if a model of ETT
  supports the theory of IIT signatures
  (\Cref{def:theoryofsignatures}), then it supports IITs in the above
  sense: \Cref{th:tosToIITs}. This is an adaptation of a proof in
  \cite{Kaposi:2019:CQI:3302515.3290315}.
\item \Cref{sec:ambroise}. Our main contribution is showing that if a
  model of ETT supports indexed W-types, then it supports the theory
  of IIT signatures (\Cref{TODO}), and hence, all IITs (\Cref{TODO}).
\end{itemize}
We conclude in \Cref{sec:conclusions}.

The contents of this paper were presented at the TYPES 2019 conference
in Oslo \cite{types}.

\subsection{Related Work}
\label{sec:related}

The current work builds heavily on the work of Kaposi et al.\
\cite{Kaposi:2019:CQI:3302515.3290315} on finitary quotient
inductive-inductive types (QIITs); we reuse both QIIT syntax and
semantics by restricting to IITs, and we reuse the term model
construction of QIITs as well. We also make use of the extesion to
infinitary QIITs \cite{large_inf_qiit} to derive the specification of
the elimination principle for the theory of IIT signatures.

IITs (however not by this name) were first used to describe the
well-typed syntax of type theory \cite{nisse,chapman09eatitself}. Agda
supported these general inductive definitions even before they were
named IITs and given semantics by Nordvall Forsberg and Setzer
\cite{nordvallforsbergSetzer2010inductiveinductive}. Nordvall
Forsberg's thesis \cite{forsberg-phd} contains a specification similar
in style to Dybjer and Setzer's codes for inductive-recursive types
\cite{Dybjer99afinite}. He also develops a categorical semantics based
on dialgebras and provides a reduction of IITs to indexed inductive
types, however only constructs the simple elimination principle as
opposed to the general one. Altenkirch et al.\ \cite{gabe} define
signatures for QIITs (thus IITs as well) and their categorical
semantics, however without proving existence of initial
algebras. Their notion of signature, just as Nordvall Forsberg's
involves more encoding overhead than ours.

Cartmell \cite{gat} introduced generalised algebraic theories using a
type-theoretic syntax. Removing equations from his signatures, we
obtain finitary IIT signatures similar to ours. He does not consider
constructing initial algebras using simpler classes of inductive
types.

Hugunin \cite{jasper} constructs several IITs in cubical Agda from
inductive types. In this setting, the lack of UIP makes constructions
significantly more involved, and essentially involves
coinductive-coinductive well-formedness predicates defined as homotopy
limits. Hugunin does not consider a generic syntax of IITs and only
works on specific examples (although the examples vary greatly). He
also only constructs the simple elimination principle.

Streicher \cite{streichersemantics} presents an interpretation of the well-formed
presyntax of a type theory into a categorical model, which is an important
ingredient in constructing an initial model, although he does not present
details on the construction of the term model or its initiality proof. Our
initiality proof can be seen as an indexed variant of his construction
(see Subsection \ref{sec:streicher} for a comparison).

Voevodsky was interested in constructing initial models of type
theories from presyntaxes. Inspired by this, Brunerie et al.\ \cite{brunerie}
formalised Streicher's proof in Agda for a type theory with $\Pi$,
$\Sigma$, $\mathbb{N}$, identity types and an infinite hierarchy of
universes. He used UIP, function extensionality and quotient types in
his formalisation. In this paper we construct a type theory without
computation rules, hence we avoid using quotients.

Intrinsic (well-typed) syntax for type theories were constructed using
IITs \cite{chapman09eatitself}, inductive-recursive types
\cite{nisse,Altenkirch:2014:CO:2631172.2631176} and QIITs
\cite{ttintt}. In this paper we avoid using such general classes of
inductive types as our goal is to reduce IITs to indexed inductive
types.

Reducing general classes of inductive types to simpler classes has a
long tradition in type theory. Indexed W-types were reduced to W-types
\cite{indexedcont} (using the essentially Streicher's idea of preterms
and a typing predicate), small inductive-recursive types to indexed
W-types \cite{malatasta13smallir}, mutual inductive types to indexed
W-types \cite{mutual}, W-types to natural numbers and quotients
\cite{Ahrens2019}. (Q)IITs can be reduced to quotient inductive types
using the reduction of generalised algebraic theories to essentially
algebraic theories \cite{gat}. Using the same reduction as mutual
inductive types to indexed inductive types, (Q)IITs with more than two
sorts can be reduced to (Q)IITs with only two sorts \cite{szumiemail}.

Awodey, Frey and Speight \cite{DBLP:conf/lics/AwodeyFS18} construct
inductive types using a restricted Church encoding in a type theory
with an impredicative universe. We use the predicative version of
their encoding to define IIT signatures.

Our reduction of IITs to indexed inductive types goes through two
steps: first we construct a concrete QIIT using inductive types, then
we construct all IITs from this particular QIIT. A more direct
approach is proposed by \cite{erasure}: here the initial algebra would
be constructed directly for any IIT signature without going through an
intermediate step.

\subsection{Notation and Formalisation}
\label{sec:notation}

\begin{definition}[Model of extensional type theory (ETT)]\label{def:ett}
  By a model of ETT we mean a category with families (CwF)
  \cite{Dybjer96internaltype,Hofmann97syntaxand} with an infinite
  predicative hierarchy of universes closed under the following type
  formers: $\Pi$, $\Sigma$, $\top$ and an identity type with
  uniqueness of identity proofs and equality reflection.
\end{definition}

We will use Agda-like type theoretic syntax to work in the internal
language of models of ETT:
\begin{itemize}
\item Universes are written $\Set_i$. We usually omit level indices in this paper.
\item $\Pi$ types are notated as $(x : A)\ra B$, or as $A \ra B$ when
  non-dependent. We sometimes omit function arguments, by implicitly
  generalising over variables.
\item $\Sigma$-types, notated either as $(x : A)\times B$, or as
  $\sum\limits_{x} B$ when we want to leave the type of the first
  projection implicit. Projections are either named or given by
  $\proj_1$ and $\proj_2$. We use $A \times B$ for non-dependent
  pairs.
\item The unit type $\top$ has the constructor $\tt$ which is
  definitionally equal to all elements of $\top$.
\item The equality (identity) type is written $t = u$, it has a
  constructor $\refl : t = t$, and equality reflection, hence we use
  the same $=$ sign for definitional equality. We occasionally
  indicate by $_{e_1,\dots,e_n \#}t$ that $t$ is well-typed thanks to
  the equalities $e_1$,\dots,$e_n$. To construct proofs, sometimes we
  write equational reasoning, e.g. $f a \overset{e}{=} f b$ where
  $e : a = b$. We also have uniqueness of identity proofs (UIP),
  expressing $(e : t = t)\ra e = \refl$. Note that function
  extensionality, expressing $((x : A) \ra f\,x = g\,x) \ra f = g$ is
  derivable.
\end{itemize}

The contents of Section \ref{sec:ambroise} were formalised in Agda,
the formalisation is available at
\url{https://github.com/amblafont/UniversalII}. Agda's pattern
matching mechanism implies uniqueness of identity proofs, we assumed
function extensionality as an axiom and used rewrite rules
\cite{cockxsprinkles} to obtain limited equality reflection.

%%%%%%%%%%%%%%%%%%%%%%%%%%%%%%%%%%%%%%%%%%%%%%%%%%%%%%%%%%%%%%%%%%%%%%%%%%%

\section{A Definition of Inductive-Inductive Types}
\label{sec:theory_of_signatures}

In this section we say what it means that a model of ETT supports
IITs. We first define the notion of IIT signature. Signatures for
algebraic theories are usually given by inductive definitions. On the
one hand, we take this even further: our notion of signature is given
by a small type theory tailer-made to describe signatures, we call
this the \emph{theory of IIT signatures}. On the other hand we would
like to avoid using a complicated inductive definition (a type theory
is a quotient inductive-inductive type \cite{ttintt}) to describe a
simpler class of inductive types. Hence we use a Church encoding
\cite{DBLP:conf/lics/AwodeyFS18} of the theory of IIT signatures thus
avoiding the need for pre-existing inductive definitions. Another
feature of our signatures is that they can include types from the
model of ETT (such as $\N$ in the
$\mathsf{isEven}$--$\mathsf{isOdd}$). This is why signatures are
specified internally to the particular model of ETT.\footnote{There is
  another method inspired by Capriotti \cite{paolo} which allows
  stating what it means that any CwF $\mathcal{C}$ (not necessarily a
  model of ETT) supports IITs with definitional computation rules. In
  this method, signatures are described in the internal language of
  $\hat{\mathcal{C}}$, the presheaf model over $\mathcal{C}$. We don't
  use this approach because it is more technical, and it would not
  strengthen our main theorem (TODO: refer) as theorem (TODO: refer)
  needs $\mathcal{C}$ to be a model of ETT.}

We define the theory of IIT signatures by saying what its algebras
(models) are. We call the \emph{theory of IIT signatures algebras}
simply \emph{signature algebras}. The theory of signatures is a small
type theory consisting of a (1) a substitution calculus (category with
families, CwF \cite{Dybjer96internaltype}) equipped with (2) a
universe, (3) a function space where the domain is in the universe and
(4) another function space with external domain. We explain the usage
of these type formers through examples after the definition.

\begin{definition}[Signature algebra, $\SignAlg$]\label{def:algebra}
  In a model of ETT, a signature algebra is an iterated $\Sigma$ type
  consisting of the following four (families of) sets, 17 operations
  and 18 equalities.
\begin{alignat*}{5}
  & \rlap{(1) Substitution calculus} \\
  & \Con && : \Set \\
  & \Ty  && : \Con\ra\Set \\
  & \Sub  && : \Con\ra\Con\ra\Set \\
  & \Tm  && : (\GG:\Con)\ra\Ty\,\GG\ra\Set \\
  & \id && : \Sub\,\GG\,\GG \\
  & \blank\circ\blank && : \Sub\,\GT\,\GD\ra\Sub\,\GG\,\GT\ra\Sub\,\GG\,\GD \\
  & \ass && : (\sigma \circ \delta) \circ \nu = \sigma \circ (\delta \circ \nu) \\
  & \idl && : \id\circ\sigma = \sigma \\
  & \idr && : \sigma\circ\id = \sigma \\
  & \blank[\blank] && : \Ty\,\GD\ra\Sub\,\GG\,\GD\ra\Ty\,\GG \\
  & \blank[\blank] && : \Tm\,\GD\,A\ra(\sigma:\Sub\,\GG\,\GD)\ra\Tm\,\GG\,(A[\sigma]) \\
  & [\id] && : A[\id] = A \\
  & [\circ] && : A[\sigma \circ \delta] = A[\sigma][\delta] \\
  & [\id] && : t[\id] = t \\
  & [\circ] && : t[\sigma \circ \delta] = t[\sigma][\delta] \\
  & \boldsymbol{\cdot} && : \Con \\
  & \epsilon && : \Sub\,\GG\,\boldsymbol{\cdot} \\
  & {\boldsymbol{\cdot}\eta} && : (\sigma : \Sub\,\GG\,\boldsymbol{\cdot}) \ra \sigma = \epsilon \\
  & \blank\ext\blank && : (\GG:\Con)\ra\Ty\,\GG\ra\Con \\
  & \blank,\blank && : (\sigma:\Sub\,\GG\,\GD)\ra\Tm\,\GG\,(A[\sigma])\ra\Sub\,\GG\,(\GD\ext A) \\
  & \pi_1 && : \Sub\,\GG\,(\GD\ext A)\ra \Sub\,\GG\,\GD \\
  & \pi_2 && : (\sigma : \Sub\,\GG\,(\GD\ext A))\ra \Tm\,\GG\,(A[\pi_1 \sigma]) \\
  & {\pi_1\beta} && : \pi_1(\sigma, t) = \sigma \\
  & {\pi_2\beta} && : \pi_2(\sigma, t) = t \\
  & {\pi\eta} && : (\pi_1\,\sigma, \pi_2\,\sigma) = \sigma \\
  & {,\circ} && : (\sigma, t)\circ\delta = (\sigma\circ\delta, t[\delta]) \\
  & \rlap{(2) Universe} \\
  & \U && : \Ty\,\GG \\
  & \El && : \Tm\,\GG\,\U\ra\Ty\,\GG \\
  & {\U[]} && : \U[\sigma] = \U \\
  & {\El[]} && : (\El\,a)[\sigma] = \El\,(a[\sigma]) \\
  & \rlap{(3) Inductive parameters} \\
  & \Pi && : (a:\Tm\,\GG\,\U)\ra\Ty\,(\GG\ext\El\,a)\ra\Ty\,\GG \\
  & \blank\oldapp\blank && : \Tm\,\GG\,(\Pi\,a\,B)\ra (u : \Tm\,\GG\,(\El\,a))
  \ra \Tm\,\GG\,(\El\,(B[\id,\,u])) \\
  & {\Pi[]} && : (\Pi\,a\,B)[\sigma] = \Pi\,(a[\sigma])\,(B[\sigma\circ\p,\q]) \\
  & {\oldapp[]} && : (t\oldapp\alpha)[\sigma] = (t[\sigma])\mathop{\oldapp}(\alpha[\sigma]) \\
  & \rlap{(4) External parameters} \\
  & \Pim && : (T:\Set)\ra(T\ra\Ty\,\GG)\ra\Ty\,\GG \\
  & \blank\appm \blank && : \Tm\,\GG\,(\Pim\,T\,B)\ra(\alpha:T) \ra\Tm\,\GG\,(B\,\alpha) \\
  & {\Pim[]} && : (\Pim\,T\,B)[\sigma] = \Pim\,T\,(\lambda \alpha.(B\,\alpha)[\sigma]) \\
  & {\appm[]} && : (t\appm\alpha)[\sigma] = (t[\sigma])\mathop{\appm}\alpha
\end{alignat*}
Given an $M:\SignAlg$, we denote its components by $\Con^M$, $\Ty^M$,
$\Sub^M$, $\Tm^M$, $\id^M$, and so on. We omit the indices if there is
only one signature algebra in scope (e.g.\ in
\Cref{def:abbrevs,ex:contexts}).
\end{definition}

\begin{definition}[Abbreviations]\label{def:abbrevs}
  For a signature algebra, we use $\wk : \Sub\,(\GG\ext A)\,\GG$
  to mean $\pi_1\,\id$. We recover de Bruijn indices by setting
  $0:= \pi_2 \,\id$ and $1+n := n[\wk]$. $\Pi\,a\,(B[\wk])$ is
  abbreviated by $a\Ra B$, $\Pim\,T\,(\lambda\_.B)$ by $T\Ram B$.
\end{definition}

\begin{example}[Example contexts in a signature algebra]\label{ex:contexts}
  Given a signature algebra, we can define a context in which
  specifies natural numbers. For readability, the an informal version
  of the same context is displayed on the right using variable names.
  \[
    \boldsymbol{\cdot}\,\ext\,\U\,\ext\,z : \El\,0\,\ext\,s : 1 \Ra \El\,1
    \hspace{4.9em}
    \boldsymbol{\cdot}\,\ext\,N:\U\,\ext\,z : \El\,N\,\ext\,s : N \Ra \El\,N
  \]
  We start with the empty context $\boldsymbol{\cdot}$, then we
  declare a sort $\U$, then we declare an operator producing an
  element of the sort denoted by $\El\,0$ where $0$ is the de Bruijn
  index referring to the sort. Finally, we declare an operator which
  takes as input an element of the sort (now it became de Bruijn index
  $1$) and produces an element of the same sort. Note the assymetry of
  the function type $\Ra$: on the left there needs to be an element of
  $\U$, while on the right any type can appear (including another
  function type). This ensures strict positivity of the operators.

  Lists with elements of a given $T : \Set$ type are given by the
  following context. Here we use the function space with external
  domain $\Ram$ to include a $T$ in the signature. For readability, we
  omit the $\lambda$ and the superscripts and we don't write the
  compatibility condition. On the right we list the same signature
  with variable names.
  \begin{alignat*}{10}
    & \boldsymbol{\cdot}\,\ext\,\U\,\ext\,\El\,0\,\ext\,T\Ram 1\Ra \El\,1 \hspace{3.56em} \boldsymbol{\cdot}\,\ext\,L:\U\,\ext\,nil : \El\,L\,\ext\,cons : T\Ram L\Ra \El\,L
  \end{alignat*}
 
  The $\Con$--$\Ty$ example from \Cref{sec:intro} is given by the
  following context.
  \begin{alignat*}{5}
    & \boldsymbol{\cdot} \,\ext                                                 && \boldsymbol{\cdot} \,\ext                                                                \\
    & \U\,\ext                                                                  && Con && : \U\,\ext                                                                        \\
    & 0\Ra\U\,\ext                                                              && Ty && : Con\Ra\U\,\ext                                                                   \\
    & \El\,1\,\ext                                                              && empty && : \El\,Con\,\ext                                                                \\
    & \Pi\,2\,(2\oldapp 0 \Ra \El\,3)\ext                                       && ext && : \Pi\,(\GG : Con)\,(Ty\oldapp\GG \Ra \El\,Con)\ext                         \\
    & \Pi\,3\,(\El\,(3\oldapp 0))\,\ext                                         && U && : \Pi\,(\GG:Con)\,(\El\,(Ty\oldapp\GG))\,\ext                                 \\
    & \Pi\,4\,(\El\,(4\oldapp(2\oldapp 0\oldapp (1\oldapp 0)))) \hspace{4.79em} && El && : \Pi\,(\GG : Con)\,(\El\,(Ty\oldapp(ext\oldapp\GG\oldapp (U\oldapp\GG))))
  \end{alignat*}
\end{example}

The above examples are contexts in any signature algebra, and we could
take this as a definition of signature: $(M:\SignAlg)\ra\Con^M$ is the
usual Church-encoding of contexts. However (as we will see in TODO)
the notion of constructor for such signatures would be too
strong. Another approach would be to assume that there is a syntax for
signature algebras (an initial signature algebra), and then a
signature would be a context in this signature algebra. We will define
syntactic signatures using this approach in the next section
(\Cref{def:syntacticsignature}), but for now we don't want to assume
the existence of any inductive type. Instead, we will use a restricted
Church encoding which needs the notion of morphism of signatures.

The notion of morphism is determined by the notion of algebra
\cite{large_inf_qiit}, but we include it here for completeness.
\begin{definition}[Signature morphism, $\SignMor$]\label{def:morphism}
  A morphism from signature algebras $M$ to $N$ denoted
  $\SignMor\,M\,N$ consists of four functions and 17 equalities
  expressing that the functions preserve the operations of the two
  algebras. We use the same naming as in \Cref{def:algebra} and use
  superscripts to denote which algebra is meant.
\begin{alignat*}{5}
  & \rlap{(1) Substitution calculus} \\
  & \Con && : \Con^M && \ra\Con^N \\
  & \Ty  && : \Ty^M\,\GG && \ra\Ty^N\,(\Con\,\GG) \\
  & \Sub  && : \Sub^M\,\GG\,\GD && \ra\Sub^N(\Con\,\GG)\,(\Con\,\GD) \\
  & \Tm  && : \Tm^M\,\GG\,A && \ra\Tm^N\,(\Con\,\GG)\,(\Ty\,A) \\
  & \id && : \Sub\,\id^M && = \id^N \\
  & {\circ} && : \sigma\circ^M\delta && = \Sub\,\sigma\circ^N\Sub\,\delta \\
  & [] && : A[\sigma]^M && = \Ty\,A[\Sub\,\sigma]^N \\
  & [] && : t[\sigma]^M && = \Tm\,t[\Sub\,\sigma]^N \\
  & \boldsymbol{\cdot} && : \Con\,\boldsymbol{\cdot}^M && = \boldsymbol{\cdot}^N \\
  & \epsilon && : \Sub\,\epsilon^M && = \epsilon^N \\
  & \ext && : \Con\,(\GG\ext^M A) && = \Con\,\GG\ext^N\Ty\,A \\
  & , && : \Sub\,(\sigma,^M t) && = \Sub\,\sigma ,^N\Tm\,t \\
  & \pi_1 && : \Sub\,(\pi_1^M\,\sigma) && = \pi_1^N\,(\Sub\,\sigma) \\
  & \pi_2 && : \Tm\,(\pi_2^M\,\sigma) && = \pi_2^N\,(\Sub\,\sigma) \\
  & \rlap{(2) Universe} \\
  & \U && : \Ty\,\U^M && = \U^N \\
  & \El && : \Ty\,(\El^M\,a) && = \El^N\,(\Tm\,a) \\
  & \rlap{(3) Inductive parameters} \\
  & \Pi && : \Ty\,(\Pi^M\,a\,B) && = \Pi^N\,(\Tm\,a)\,(\Ty\,B) \\
  & \oldapp && : \Tm\,(t\mathbin{{\oldapp}^M} u) && = \Tm\,t\mathbin{{\oldapp}^N}\,\Tm\,u \\
  & \rlap{(4) External parameters} \\
  & \Pim && : \Ty\,(\Pim^M\,T\,B) && = \Pim^N\,T\,(\lambda\alpha.\Ty\,(B\,\alpha)) \\
  & \appm && : \Tm\,(t\mathbin{{\appm}^M}\alpha) && = \Tm\,t\mathbin{{\appm}^N}\alpha
\end{alignat*}
Given an $f:\SignMor\,M\,N$, we denote its first four components just
by $f_\Con$, $f_\Ty$, $f_\Sub$, $f_\Tm$ or just write $f$ if it is
clear which one is meant.
\end{definition}

We define IIT signatures using the Church encoding introduced by
Awodey, Frey and Speight \cite{DBLP:conf/lics/AwodeyFS18}. A
difference is that we avoid the usage of impredicativity because it is
not a problem for our use case that signatures are not in the lowest
universe.
\begin{definition}[IIT signature]\label{defn:sign}
  An IIT signature is a context in an arbitrary signature algebra that
  is compatible with morphisms:
  \begin{alignat*}{6}
    & \Sign :=\,\, && \big({sig}:(M:\SignAlg)\ra\Con^M\big)\times \\
    & && \big((M\,N:\SignAlg)(f:\SignMor\,M\,N)\ra f_\Con\,({sig}\,M) = {sig}\,N \big).
  \end{alignat*}
\end{definition}
The compatibility condition says that if we obtain an $M$-context
using ${sig}$ in model $M$ and then we transport it to $N$ using $f$,
we get the same $N$-context as directly applying ${sig}$ to $N$.

Our notion of signatures don't form a signature algebra.
\begin{lemma}
  There is no $M : \SignAlg$, in which $\Con^M = \Sign$.
\end{lemma}
\begin{proof}
  If the $\Con$ component in $\SignAlg$ is $\Set_i$, then $\SignAlg$
  is in $\Set_{i+1}$, but as $\Sign$ is defined as
  $(\SignAlg\ra\dots)\times\dots$, it is at least in $\Set_{i+1}$, so
  we can't choose $\Con^M : \Set_i$ to be $\Sign : \Set_{i+1}$.
\end{proof}

Note that the notion of IIT signature is relative to a model of ETT:
it is expressed as a term (of a function type) in the model. This is
necessary because of the function space $\Pim$ which has as domain an
arbitrary type in the model. We make use of $\Pim$ in signatures with
external parameters, like the type of the elements in lists.

\begin{example}[Example signature]\label{ex:signatures}
  Now we can formally describe the contexts given in
  \Cref{ex:contexts} as signatures. For natural numbers, we have the
  following pair of functions. The second function returns an equality
  proof which we describe using equational reasoning.
  \begin{alignat*}{5}
    & \rlap{$(nat,natc) := $} \\
    & \hspace{1em} && \big( && \rlap{$\lambda M.(\boldsymbol{\cdot}^M\,\ext^M\,\U^M\,\ext^M\,\El^M\,0^M\,\ext^M\,1^M \Ra^M \El^M\,1^M),$} \\
    & && && \lambda M\,N\,f\,.\, && f_\Con\,(\boldsymbol{\cdot}^M\,\ext^M\,\U^M\,\ext^M\,\El^M\,0^M\,\ext^M\,1^M \Ra^M \El^M\,1^M) = \\
    & && && && f_\Con\,(\boldsymbol{\cdot}^M\,\ext^M\,\U^M\,\ext^M\,\El^M\,0^M)\ext^N f_\Ty\,(1^N\Ra^N \El^N\,1^N) = \\
    & && && && f_\Con\,(\boldsymbol{\cdot}^M\,\ext^M\,\U^M)\,\ext^N\,f_\Ty\,(\El^M\,0^M)\ext^N f_\Tm\,1^N\Ra^M f_\Ty\,(\El^N\,1^N) = \\
    & && && && f_\Con\,\boldsymbol{\cdot}^M\,\ext^N\,f_\Ty\,\U^M\,\ext^N\,\El^N\,(f_\Tm\,0^M)\ext^N 1^M\Ra^M \El^M\,(f_\Tm\,1^N) = \\
    & && && && \boldsymbol{\cdot}^N\,\ext^N\,\U^N\,\ext^N\,\El^N\,0^N\,\ext^N\,1^N \Ra^N \El^N\,1^N \big)
  \end{alignat*}
  The first component builds the context describing natural numbers in
  $M$, the second one uses the fact that $f$ is a morphism, that is,
  it preserves all operations.

  The signatures for lists and $\Con$--$\Ty$ can be given analogously.
\end{example}

Given a model of ETT and an IIT signature in it, we would like to say
what it means that the model supports the given IIT. For this we
define the signature algebra $\ADS$ which will provide notions of
algebras, displayed algebras and sections for each signature. This is
the same as the $\blank^\A$, $\blank^\D$ and $\blank^\S$ operations in
\cite{Kaposi:2019:CQI:3302515.3290315}. Before defining $\ADS$, we
illustrate its usage by an example.

\begin{example}[Algebras, displayed algebras and sections for natural
  numbers]\label{ex:adsnat}
  For the signature of natural numbers as given in
  \Cref{ex:signatures}, algebras are given by the $\Sigma$-type
  $(N:\Set)\times N\times(N\ra N)$. A displayed algebra over $(N,z,s)$
  is given by the $\Sigma$-type
  \[
    (N^D:N\ra\Set)\times N^D\,z \times((n:N)\ra N^D\,n\ra N^D\,(s\,n)),
  \]
  and a section of a displayed algebra $(N^D,z^D,s^D)$ over $(N,z,s)$
  is given by the $\Sigma$-type
  \[
    (N^S:(n:N)\ra N^D\,n)\times (N^S\,z = z^D) \times((n:N)\ra N^S\,(s\,n) = s^D\,n\,(N^S\,n)).
  \]
  Displayed algebras over the initial algebra are called motives and
  methods of the eliminator, while a section of a displayed algebra
  over the initial algebra is the eliminator together with its
  computation rules.
\end{example}
  
\begin{definition}[The signature algebra $\ADS$]\label{def:ads}
  We define an element of $\SignAlg$ by listing all its components
  $\Con$, $\Ty$, $\Sub$, and so on, one per row. Each such component
  has three parts denoted by $^\A$, $^\D$ and $^\S$, respectively. The
  equality components of $\SignAlg$ are omitted as they are all
  reflexivity.
  \begin{alignat*}{10}
    & (\GG^\A:\Set) && \hspace{-0.3em}\times && (\GG^\D:\GG^\A\ra\Set) && \hspace{-0.3em}\times && (\GG^\S:(\gamma:\GG^\A)\ra\GG^\D\,\gamma\ra\Set) \\
    & (A^\A:\GG^\A\ra\Set) && \hspace{-0.3em}\times && (A^\D:\GG^\D\,\gamma\ra A^\A\,\gamma\ra\Set) && \hspace{-0.3em}\times && (A^\S:\GG^\S\,\gamma\,\gamma^D\ra(\alpha:A^\A\,\gamma)\ra \\
    & && && && && \hspace{1em} A^\D\,\gamma^D\,\alpha\ra\Set) \\
    & (\sigma^\A:\GG^\A\ra\GD^\A) && \hspace{-0.3em}\times && (\sigma^\D:\GG^\D\,\gamma\ra\GD^\D\,(\sigma^\A\,\gamma)) && \hspace{-0.3em}\times && (\sigma^\S:\GG^\S\,\gamma\,\gamma^D\ra \\
    & && && && && \hspace{1em} \GD^\S\,(\sigma^\A\,\gamma)\,(\sigma^\D\,\gamma^D)) \\
    & (t^\A:(\gamma:\GG^\A)\ra A^\A\,\gamma) && \hspace{-0.3em}\times && (t^\D:(\gamma^D:\GG^\D\,\gamma)\ra && \hspace{-0.3em}\times && (t^\S:(\gamma^S:\GG^\S\,\gamma\,\gamma^D)\ra \\
    & && && \hspace{1em} A^\D\,\gamma^D\,(t^\A\,\gamma)) && && \hspace{1em}  A^\S\,(t^\A\,\gamma)\,(t^\D\,\gamma^D)) \\
    & \id^\A\,\gamma := \gamma && && \id^\D\,\gamma^D:=\gamma^D && && \id^\S\,\gamma^S := \gamma^S \\
    & (\sigma\circ\delta)^\A\,\gamma := \sigma^\A\,(\delta^\A\,\gamma) && && (\sigma\circ\delta)^\D\,\gamma^D := \sigma^\D\,(\delta^\D\,\gamma^D) && && (\sigma\circ\delta)^\S\,\gamma^S := \sigma^\S\,(\delta^\S\,\gamma^S) \\
    & (A[\sigma])^\A\,\gamma := A^\A\,(\sigma^\A\,\gamma) && && (A[\sigma])^\D\,\gamma^D := A^\D\,(\sigma^\D\,\gamma^D) && && (A[\sigma])^\S\,\gamma^S := A^\S\,(\sigma^\S\,\gamma^S) \\
    & (t[\sigma])^\A\,\gamma := t^\A\,(\sigma^\A\,\gamma) && && (t[\sigma])^\D\,\gamma^D := t^\D\,(\sigma^\D\,\gamma^D) && && (t[\sigma])^\S\,\gamma^S := t^\S\,(\sigma^\S\,\gamma^S) \\
    & \cdot^\A := \top && && \cdot^\D\,\_ := \top && && \cdot^\S\,\_\,\_ := \top \\
    & \epsilon^\A\,\_ := \tt && && \epsilon^\D\,\_ := \tt && && \epsilon^\S\,\_ := \tt \\
    & (\GG\rhd A)^\A := && && (\GG\rhd A)^\D\,(\gamma,\alpha) :=  && && (\GG\rhd A)^\S\,(\gamma,\alpha)\,(\gamma^D,\alpha^D) := \\
    & \hspace{1em} (\gamma:\GG^\A)\times A^\A\,\gamma && && \hspace{1em} (\gamma^D:\GG^\D\,\gamma)\times A^\D\,\gamma^D\,\alpha && && \hspace{1em} (\gamma^S:\GG^\S\,\gamma\,\gamma^D)\times A^\S\,\gamma^S\,\alpha\,\alpha^D \\
    & (\sigma,t)^\A\,\gamma := (\sigma^\A\,\gamma,t^\A\,\gamma) && && (\sigma,t)^\D\,\gamma^D := (\sigma^\D\,\gamma^D,t^\D\,\gamma^D) && && (\sigma,t)^\S\,\gamma^S := (\sigma^\S\,\gamma^S,t^\S\,\gamma^S) \\
    & (\pi_1\,\sigma)^\A\,\gamma := \proj_1\,(\sigma^\A\,\gamma) && && (\pi_1\,\sigma)^\D\,\gamma^D := \proj_1\,(\sigma^\D\,\gamma^D) && && (\pi_1\,\sigma)^\S\,\gamma^S := \proj_1\,(\sigma^\S\,\gamma^S) \\
    & (\pi_2\,\sigma)^\A\,\gamma := \proj_2\,(\sigma^\A\,\gamma) && && (\pi_2\,\sigma)^\D\,\gamma^D := \proj_2\,(\sigma^\D\,\gamma^D) && && (\pi_2\,\sigma)^\S\,\gamma^S := \proj_2\,(\sigma^\S\,\gamma^S) \\
    & \U^\A\,\gamma := \Set && && \U^\D\,\gamma^D\,T := T\ra\Set && && \U^\S\,\gamma^S\,T\,T^D := (\alpha:T)\ra T^D\,\alpha \\
    & (\El\,a)^\A\,\gamma := a^\A\,\gamma && && (\El\,a)^\D\,\gamma^D\,\alpha := a^\D\,\gamma^D\,\alpha && && (\El\,a)^\S\,\gamma^S\,\alpha\,\alpha^D := (a^\S\,\gamma^S\,\alpha = \alpha^D) \\
    & (\Pi\,a\,B)^\A\,\gamma := && && (\Pi\,a\,B)^\D\,\gamma^D\,f := && && (\Pi\,a\,B)^\S\,\gamma^S\,f\,f^D := (\alpha:a^\A\,\gamma)\ra \\
    & \hspace{1em} (\alpha:a^\A\,\gamma)\ra B^\A\,(\gamma,\alpha) && && \hspace{1em}  (\alpha^D:a^\D\,\gamma^D\,\alpha)\ra && && \hspace{1em} B^\S\,(\gamma^S,\refl_{a^\S\,\gamma^S\,\alpha})\,(f\,\alpha)\,\\
    & && && \hspace{1em}  B^\D\,(\gamma^D,\alpha^D)\,(f\,\alpha) && && \hspace{1em}\hphantom{B^\S\,}(f^D\,(a^\S\,\gamma^S\,\alpha)) \\
    & (t\oldapp u)^\A\,\gamma := t^\A\,\gamma\,(u^\A\,\gamma) && &&  (t\oldapp u)^\D\,\gamma^D := t^\D\,\gamma^D\,(u^\D\,\gamma^D) && &&  (t\oldapp u)^\S\,\gamma^S := _{u^\S\,\gamma^S \#} t^\S\,\gamma^\S\,(u^\A\,\gamma) \\
    & (\Pim\,T\,B)^\A\,\gamma := && && (\Pim\,T\,B)^\D\,\gamma^D\,f := && && (\Pim\,T\,B)^\S\,\gamma^S\,f\,f^D := (\alpha:T)\ra \\
    & \hspace{1em} (\alpha:T)\ra(B\,\alpha)^\A\,\gamma && && \hspace{1em} (\alpha:T)\ra(B\,\alpha)^\D\,\gamma^D\,(f\,\alpha) && && \hspace{1em} (B\,\alpha)^\S\,\gamma^S\,(f\,\alpha)\,(f^D\,\alpha) \\
    & (t\appm \alpha)^\A\,\gamma := t^\A\,\gamma\,\alpha && && (t\appm \alpha)^\D\,\gamma^D := t^\D\,\gamma^D\,\alpha && && (t\appm \alpha)^\S\,\gamma^S := t^\S\,\gamma^S\,\alpha
  \end{alignat*}
\end{definition}
\Cref{def:ads} can be explained by columns (see
\cite[Sections 4 and 6]{Kaposi:2019:CQI:3302515.3290315} for more details) or by rows
(see \cite[Section 7.4]{Kaposi:2019:CQI:3302515.3290315}).

We first explain it by columns: the first column ($^\A$ components)
corresponds to the standard model (set model, metacircular
interpretation \cite{ttintt}): contexts are sets, types are families,
terms are functions, the universe $\U$ is given by $\Set$, function
spaces are given by the external function space. The $^\D$ column is a
logical predicate interpretation, $^\A$ and $^\D$ together are a unary
version of the parametric model for dependent types
\cite{10.1145/2535838.2535852}. Contexts are predicates, types are
families of predicates, terms say that the $^\A$ interpretation
respects the predicates (this is ususally called fundamental lemma of
the logical predicate). $\U$ is given by predicate space, the
predicate at a $\Pi$ type holds for a function if it respects the
predicates. For $\Pim$, the predicate is defined pointwise. The last
column $^\S$ is a modified dependent logical relation which refers to
both $^A$ and $^\D$. Contexts are binary relations where the second
parameter depends on the first one, types are dependent variants of
this, terms say that the relation is respected by $^\A$ and $^\D$,
respectively. $\U$ is however not relation space, but a function and
$(\El\,a)^\S$ is the graph of the function $a^\S$. $\Pi^\S$ for a
function again says that the function respects the relation, however
we don't simply say
\[
  (\Pi\,a\,B)^\S\,\gamma^S\,f\,f^D := (\alpha:a^\A\,\gamma)(\alpha^D:a^\D\,\gamma^D\,\alpha)(\alpha^\S:(\El\,a)^\S\,\gamma^S\,\alpha\,\alpha^D)\ra B^\S\,\dots,
\]
as $(\El\,a)^\S\,\gamma^S\,\alpha\,\alpha^D$ is just an equality
$a^\S\,\gamma^S\,\alpha = \alpha^D$ which we can singleton
contract. So we omit $\alpha^D$ and this equality as an input and
replace $\alpha^D$ by $a^\S\,\gamma^S\,\alpha$ in the definition.

When viewing $\ADS$ by rows, we can see that it is a part of the CwF
model of type theory \cite[Section
7.4]{Kaposi:2019:CQI:3302515.3290315}. In the CwF model, a context is
given by a CwF. Now, from the category part of the CwF, we only have
objects ($\GG^\A$), and from the families, we have the families for
types $\GG^\D$ and terms $\GG^\S$. Types are the corresponding
parts of displayed CwFs, substitutions are parts of CwF morphisms,
terms are parts of CwF sections. $\U$ is part of the CwF of sets,
$\El\,a$ is the part of the discrete displayed CwF coming from $a$
(which is a CwF-morphism from $\GG$ to the CwF of sets). $\Pi$ is
given by a dependent product of displayed CwFs where it is essential
that the domain is discrete, $\Pim$ is the pointwise direct product.

\begin{definition}[The set signature algebra $\A$]\label{def:a}
  $\A : \SignAlg$ is given by the first $^\A$ components of $\ADS$
  (\Cref{def:ads}), that is, $\Con^\A := \Set$,
  $\Ty^\A\,\GG := \GG\ra\Set$,
  $\Sub^\A\,\GG\,\GD := \GG\ra\GD$, and so on. There is a
  morphism from $\ADS$ to $\A$ defined by $\blank^\A$ at each
  component, which we also denote by $\blank^\A : \SignMor\,\ADS\,\A$.
\end{definition}

\begin{definition}[A model of ETT supports IITs]\label{def:hasiits}
  A model of ETT supports IITs if for any signature $(sig,sigc):\Sign$
  there is a
  \[
    \con_{sig}: ({sig}\,\ADS)^\A
  \]
  and an
  \[
    \elim_{sig} : (\gamma^D:({sig}\,\ADS)^\D\,\con_{sig})\ra ({sig}\,\ADS)^\S\,\con_{sig}\,\gamma^D.
  \]
\end{definition}
In other words, for any signature, we have an algebra called $\con$
(constructors) and for any displayed algebra over the constructors, we
have a section (called the eliminator).

One can check that \Cref{def:hasiits} gives the right
notion of constructors and elimination principle for the signatures in
\Cref{ex:signatures}.
\begin{example}[A model of ETT supports natural numbers]
  For the signature $({nat},{natc})$ of natural numbers in
  \Cref{ex:signatures}, the type of $\con_{nat}$ is
  \begin{alignat*}{5}
    & ({nat}\,\ADS)^\A = \\
    & (\boldsymbol{\cdot}^\ADS\,\ext^\ADS\,\U^\ADS\,\ext^\ADS\,\El^\ADS\,0^\ADS\,\ext^\ADS\,1^\ADS \Ra^\ADS \El^\ADS\,1^\ADS)^\A = \\
    & \Big(\big((\boldsymbol{\cdot}\,\ext\,\U)\,\ext\,\El\,(\pi_2\,\id)\big)\,\ext\,\big(\pi_2\,(\pi_1\,\id)\big) \Ra \El\,\big(\pi_2\,(\pi_1\,\id)\big)\Big)^\A = \\
    & \Big(\gamma'':\big(\gamma':((\gamma:\boldsymbol{\cdot}^\A)\times \U^\A\,\gamma)\big)\times (\El\,(\pi_2\,\id))^\A\,\gamma'\Big)\times\Big(\Pi\,\big(\pi_2\,(\pi_1\,\id)\big)\,\big(\pi_2\,(\pi_1\,(\pi_1\,\id))\big)\Big)^\A\,\gamma'' = \\
    & \Big(\gamma'':\big(\gamma':((\gamma:\top)\times\Set)\big)\times(\proj_2\,\gamma')\Big)\times\big(\proj_2\,(\proj_1\,\gamma'')\ra\proj_2\,(\proj_1\,\gamma'')\big),
  \end{alignat*}
  which is a left-nested $\Sigma$ type isomorphic to its
  right-nested counterpart
  \[
    (N:\Set)\times \big(N \times (N\ra N)\big).
  \]
  Writing $(((\tt,\mathsf{Nat}),\mathsf{zero}),\mathsf{suc})$ for
  $\con_{nat}$, the type of $\elim_{nat}$ computes as follows.
  \begin{alignat*}{5}
    & \rlap{$(\gamma^D: ({nat}\,\ADS)^\D\,\con_{nat})\ra ({nat}\,\ADS)^\S\,\con_{nat}\,\gamma^D =$} \\
    & \rlap{$\bigg(\gamma^D: \Big( \big((\boldsymbol{\cdot}\,\ext\,\U)\,\ext\,\El\,(\pi_2\,\id)\big)\,\ext\,\big(\pi_2\,(\pi_1\,\id)\big) \Ra \El\,\big(\pi_2\,(\pi_1\,\id)\big)\Big)^\D\,\con_{nat}\bigg)\ra$} \\
    & \rlap{$\Big( \big((\boldsymbol{\cdot}\,\ext\,\U)\,\ext\,\El\,(\pi_2\,\id)\big)\,\ext\,\big(\pi_2\,(\pi_1\,\id)\big) \Ra \El\,\big(\pi_2\,(\pi_1\,\id)\big)\Big)^\S\,\con_{nat}\,\gamma^D =$} \\
    & \rlap{$\bigg(\gamma^D: \Big( \big((\boldsymbol{\cdot}\,\ext\,\U)\,\ext\,\El\,(\pi_2\,\id)\big)\,\ext\,\big(\pi_2\,(\pi_1\,\id)\big) \Ra \El\,\big(\pi_2\,(\pi_1\,\id)\big)\Big)^\D\,$} \\
    & \rlap{$\hspace{1em} \big(((\tt,\mathsf{Nat}),\mathsf{zero}),\mathsf{suc}\big)\bigg)\ra$} \\
    & \rlap{$\Big(\big((\boldsymbol{\cdot}\,\ext\,\U)\,\ext\,\El\,(\pi_2\,\id)\big)\,\ext\,\big(\pi_2\,(\pi_1\,\id)\big) \Ra \El\,\big(\pi_2\,(\pi_1\,\id)\big)\Big)^\S\,\big(((\tt,\mathsf{Nat}),\mathsf{zero}),\mathsf{suc}\big)\,\gamma^D = $} \\
    & \bigg(\big(((\tt,N^D),z^D),s^D\big): && \Big( {\gamma^D}'':\big({\gamma^D}':((\gamma^D:\boldsymbol{\cdot}^\D\,\tt)\times \U^\D\,\gamma^D\,\mathsf{Nat})\big)\times (\El\,(\pi_2\,\id))^\D\,{\gamma^D}'\,\mathsf{zero}\Big)\times \\
    & && \Big(\Pi\,\big(\pi_2\,(\pi_1\,\id)\big)\,\big(\pi_2\,(\pi_1\,(\pi_1\,\id))\big)\Big)^\D\,{\gamma^D}''\,\mathsf{suc}\bigg)\ra \\
    & \rlap{$ \Big(\big((\boldsymbol{\cdot}\,\ext\,\U)\,\ext\,\El\,(\pi_2\,\id)\big)\,\ext\,\big(\pi_2\,(\pi_1\,\id)\big) \Ra \El\,\big(\pi_2\,(\pi_1\,\id)\big)\Big)^\S\,(((\tt,\mathsf{Nat}),\mathsf{zero}),\mathsf{suc})\,$} \\
    & \rlap{$\hspace{1em} \big(((\tt,N^D),z^D),s^D\big) =$} \\
    & \bigg(\big(((\tt,N^D),z^D),s^D\big): && \Big( {\gamma^D}'':\big({\gamma^D}':((\gamma^D:\boldsymbol{\cdot}^\D\,\tt)\times \U^\D\,\gamma^D\,\mathsf{Nat})\big)\times (\El\,(\pi_2\,\id))^\D\,{\gamma^D}'\,\mathsf{zero}\Big)\times \\
    & && \Big(\Pi\,\big(\pi_2\,(\pi_1\,\id)\big)\,\big(\pi_2\,(\pi_1\,(\pi_1\,\id))\big)\Big)^\D\,{\gamma^D}''\,\mathsf{suc}\bigg)\ra \\
    & \rlap{$ \Big( {\gamma^S}'':\big({\gamma^S}':((\gamma^S:\boldsymbol{\cdot}^\S\,\tt\,\tt)\times \U^\S\,\gamma^S\,\mathsf{Nat}\,N^D)\big)\times (\El\,(\pi_2\,\id))^\S\,{\gamma^S}'\,\mathsf{zero}\,z^D\Big)\times$} \\
    & \rlap{$ \Big(\Pi\,\big(\pi_2\,(\pi_1\,\id)\big)\,\big(\pi_2\,(\pi_1\,(\pi_1\,\id))\big)\Big)^\S\,{\gamma^S}''\,\mathsf{suc}\,s^D = $} \\
    & \bigg(\big(((\tt,N^D),z^D),s^D\big): && \Big( {\gamma^D}'':\big({\gamma^D}':((\gamma^D:\top)\times (\mathsf{Nat}\ra\Set))\big)\times \proj_2\,{\gamma^D}'\,\mathsf{zero}\Big)\times \\
    & && \big(\proj_2\,(\proj_1\,{\gamma^D}'')\,n\ra\proj_2\,(\proj_1\,{\gamma^D}'')\,(\mathsf{suc}\,n)\big)\bigg)\ra \\
    & \rlap{$ \Big( {\gamma^S}'':\big({\gamma^S}':((\gamma^S:\top)\times ((n:\mathsf{Nat})\ra N^D\,n))\big)\times \proj_2\,{\gamma^S}'\,\mathsf{zero} = z^D\Big)\times$} \\
    & \rlap{$ \Big(\big(n:\mathsf{Nat}\big)\ra\proj_2\,(\proj_1\,(\proj_1\,{\gamma^S}''))\,(\mathsf{suc}\,n) = s^D\,\big(\proj_2\,(\proj_1\,(\proj_1\,{\gamma^S}''))\,n\big)\Big)$}
  \end{alignat*}
  This is again a left-nested version of the expected elimination
  principle
  \begin{alignat*}{10}
    & (N^D:\mathsf{Nat}\ra\Set)(z^D:N^D\,\mathsf{zero})\big(s^D:(n:\mathsf{Nat})\ra N^D\,n\ra N^D\,(\mathsf{suc}\,n)\big)\ra \\
    & \big(N^S:(n:\mathsf{Nat})\ra N^D\,n\big)\times(N^S\,\mathsf{zero} = z^D)\times \big((n:\mathsf{Nat})\ra N^S\,(\mathsf{suc}\,n) = s^D\,(N^S\,n)\big)
  \end{alignat*}
\end{example}
\begin{remark}
  The computation rules of the elimination principle are only expected
  up to the internal equality type, but as we work with a model of
  ETT, we also get them as definitional equalities by equality
  reflection.
\end{remark}

%%%%%%%%%%%%%%%%%%%%%%%%%%%%%%%%%%%%%%%%%%%%%%%%%%%%%%%%%%%%%%%%%%%%%%%%%%%

\section{Constructing all IITs from the Theory of IIT Signatures}
\label{sec:constructingiits}

In the previous section, using the notions of signature algebras and
signature morphisms, we defined IIT signatures and what it means for a
model of ETT to support all IITs. In this section we show that if a
model of ETT supports the theory of IIT signatures, then it supports
all IITs. In the sense of the Church encoding of \Cref{defn:sign},
every model of ETT can describe ITT signatures. In contrast, in
\Cref{def:theoryofsignatures}, we will require existence of an initial
signature algebra.

The contents of this section are an adjustment of \cite[Sections 4 and
6]{Kaposi:2019:CQI:3302515.3290315} to our setting.

\begin{definition}\label{def:theoryofsignatures}
  A model of ETT supports the theory of IIT signatures if there is a
  signature algebra $\I : \SignAlg$ equipped with a unique morphism
  $\ll\blank\rr_{M} : \SignMor\,\I\,M$ into any algebra $M$. Sometimes
  we omit the subscript $_M$. We call $\I$ the syntax or initial
  algebra, the morphism $\ll\blank\rr$ is called recursor.
\end{definition}

\begin{definition}[Syntactic signatures]\label{def:syntacticsignature}
  In a model of ETT supporting the theory of ITT signatures, we call
  elements of $\Con^\I$ syntactic signatures.
\end{definition}

One may wonder what is the relationship between the two notion of
signatures.
\begin{lemma}
  In a model of ETT supporting the theory of ITT signatures,
  signatures and syntactic signatures are isomorphic.
\end{lemma}
\begin{proof}
  We can turn a $({sig},{sigc}) : \Sign$ into $\Con^\I$ by ${sig}\,\I$
  and an $\GO:\Con^\I$ into a $\Sign$ by
  $\Big(\lambda M.\ll\GO\rr_M, \lambda M\,N\,f.\big(f\,\ll\GO\rr_M =
  (f\circ\ll\blank\rr_M)\,\GO = \ll\GO\rr_N\big)\Big)$ where the equality
  proof in the second component comes from uniqueness of the recursor
  (we have to define composition of morphisms $\circ$ for this). The
  compositions of these two maps are the identities: $({sig},{sigc})$
  is mapped to
  $(\lambda M.\ll{sig}\,I\rr_M,\dots) = (\lambda
  M.\ll\blank\rr_M\,({sig}\,I),\dots)$ which is equal to
  $(\lambda M.{sig}\,M,\dots)$ because of ${sigc}$; $\GO$ is mapped
  to $\ll\GO\rr_\I = \GO$ by uniqueness of $\ll\blank\rr$.
\end{proof}

We will define the term signature algebra using which we obtain the
constructors $\con$ for any IIT signature. Then we will define another
signature algebra which provides the eliminator. Before doing these,
we illustrate the idea of both constructions on natural numbers.
\begin{example}\label{eg:nat_con}
  For natural numbers, we will define the constructors $\con$ as the
  following natural number algebra
  $(\mathsf{Nat},\mathsf{zero},\mathsf{suc})$. We write variable names
  instead of de Bruijn indices for readability.
  \begin{alignat*}{5}
    & \mathsf{Nat} && := \Tm^\I\,(\cdot\rhd N:\U\rhd z:\El\,N\rhd s:N\Ra\El\,N)\,(\El\,N) \\
    & \mathsf{zero} && := z \\
    & \mathsf{suc} && := \lambda t. (s\oldapp t)
  \end{alignat*}
  Natural numbers are simply $\I$-terms of type $\El\,N$ in the
  context which is the syntactic signature for natural numbers. In
  this context, the only way to define a term of type $\El\,N$ is to
  use $z$ and $s$, corresponding to the $\mathsf{zero}$ and
  $\mathsf{suc}$ constructors.

  To define the action of the eliminator on a natural number
  $n : \mathsf{Nat}$, let's look at the type of the displayed algebra
  interpretation of the number:
  \[
    {\ll n \rr_\ADS}^\D : (\gamma^D : \ll\cdot\rhd N:\U\rhd z:\El\,N\rhd s:N\Ra\El\,N\rr^\D\,\con) \ra \ll\El\,N\rr^\D\,(\ll n\rr^\A\,\con)
  \]
  This says that for a displayed algebra $\gamma^D=(N^D,z^D,s^D)$ over
  $\con$ (i.e.\ the motives and methods of the eliminator), we get a
  witness of the predicate $\ll\El\,N\rr^\D = N^D$ at the algebra
  interpretation of $n$. This is not yet good, as we would like to get
  $N^D\,n$ instead of $N^D\,(\ll n\rr^\A\,\con)$ as a result. However,
  interpretation into the term signature algebra will imply that
  $n = \ll n\rr^\A\,\con$.
\end{example}

\begin{definition}[Term signature algebra $\IC_{\blank}$]
  For an $\GO:\Con^I$, we define $\IC_\GO:\SignAlg$ which we
  call the term signature algebra. It is equipped with a morphism
  $\blank^\I : \SignMor\,(\IC_\GO)\,\I$. We define $\IC_{\GO}$
  by listing its components $\Con$, $\Ty$, $\Sub$, and so on, one per
  row. Each component has two parts denoted by $^\I$ and $^\C$. The
  $^\I$ part just reuses the corresponding components from $\I$, and
  thus the morphism $\blank^\I$ is defined as the obvious
  projection. We omit the equality components, as they come from UIP
  or are trivial. We also omit the components for terms and
  substitutions as their $^\C$ parts are uninformative equational
  reasonings.
  \begin{alignat*}{10}
    & \GG^\I:\Con^\I && \hspace{1em} && \GG^\C: \Sub^\I\,\GO\,\GG^\I\ra \ll\GG\rr_\A \\
    & A^\I:\Ty^\I\,\GG^\I && && A^\C:(\nu:\Sub^\I\,\GO\,\GG^\I)\ra\Tm^\I\,\GO\,(A^\I[\nu])\ra \ll A\rr_\A\,(\GG^\C\,\nu) \\
    & \sigma^\I:\Sub^\I\,\GG^\I\,\GD^\I && && \sigma^\C:\GD^\C\,(\sigma^\I\circ\nu)=\ll\sigma\rr_\A\,(\GG^\C\,\nu) \\
    & t^\I:\Tm^\I\,\GG^\I\,A^\I && && t^\C : A^\C\,\nu\,(t^\I[\nu])= \ll t\rr_\A\,(\GG^\C\,\nu) \\
    & (A[\sigma])^\I := A^\I[\sigma^\I]^\I && && (A[\sigma])^\C\,\nu\,t := A^\C\,(\sigma^\I\circ\nu)\,t \\
    & \cdot^\I := \cdot^\I && && \cdot^\C\,\nu := \tt \\
    & (\GG\rhd A)^\I := \GG^\I\rhd^\I A^\I && && (\GG\rhd A)^\C\,\nu := (\GG^\C\,(\pi_1\,\nu), A^\C\,(\pi_1\,\nu)\,(\pi_2\,\nu)) \\
    & \U^\I := \U^\I && && \U^\C\nu\,a := \Tm^\I\,\GO\,(\El^\I\,a) \\
    & (\El\,a)^\I := \El^\I\,a^\I && && (\El\,a)^\C\,\nu\,t := _{a^\C\,\nu\#}t \\
    & (\Pi\,a\,B)^\I := \Pi^\I\,a^\I\,B^\I && && (\Pi\,a\,B)^\C\,\nu\,t := \lambda\alpha.B^\C\,(\nu, _{a^\C\,\nu\#}\alpha)\,(t\oldapp {_{a^\C\,\nu\#}\alpha}) \\
    & (\Pim\,T\,B)^\I := \Pim^\I\,T\,B^\I && && (\Pim\,T\,B)^\C\,\nu\,t := \lambda\alpha.(B\,\alpha)^\C\,\nu\,(t\appm\alpha)
  \end{alignat*}
\end{definition}
\begin{example}\label{ex:constructors}
  Now given a syntactic signature $\GO:\Con^\I$, we get the
  constructors as an $\GO$-algebra by
  $\omega:=(\ll\GO\rr_{\IC_\GO})^\C\,\id^\I :
  \ll\GO\rr_\A$. If $\GO$ is the syntactic signature for natural
  numbers, we get the constructors as in \Cref{eg:nat_con}.

  An $a : \Tm^\I\,\GO\,\U$ is a sort term for the syntactic
  signature $\GO$. If $\GO$ is the syntactic signature for
  natural numbers, $a$ can only be $N$ ($1$ as a de Bruijn index). If
  $\GO$ is the syntactic signature for $\Con$--$\Ty$
  (\Cref{ex:contexts}), $a$ can be $Con$, $Ty\oldapp{empty}$,
  $Ty\oldapp({ext}\oldapp {empty}\oldapp(U\oldapp{empty}))$, and so
  on. In any case, for such an $a$, we obtain
  $(\ll a\rr_{\IC_\GO})^\C\,\id^\I : \Tm^\I\,\GO\,(\El\,a) = \ll
  a\rr_\A\,\omega$. That is, the algebra interpretation of a sort term
  at the constructors is equal to terms of that sort.

  A $t : \Tm^\I\,\GO\,(\El\,a)$ is a term of a sort type $a$
  constructed using the constructors in $\GO$. For natural numbers,
  such a $t$ can only be $s$ applied iteratively to $z$. For such a
  $t$, we obtain
  $(\ll t\rr_{\IC_\GO})^\C\,\id^\I : (t = \ll
  t\rr_\A\,\omega)$. That is, a constructor term is equal to its
  algebra interpretation at the constructors. This is exactly the
  equation needed at the end of \Cref{eg:nat_con}.
\end{example}

\begin{definition}[Eliminator signature algebra $\IE_{\blank}$]
  Given an $\GO:\Con^\I$, we use the abbreviation
  $\omega := \ll\GO\rr_{\IC_\GO}\,\id^\I$ as in
  \Cref{ex:constructors}. Assuming an
  $\omega^D : (\ll\GO\rr_{\ADS})^\D\,\omega$, we define the
  signature algebra $\IE_{\omega^\D}$. It is equipped with a morphism
  $\blank^\I : \SignMor\,\IE_{\omega^D}\,\I$. We define
  $\IE_{\omega^D}$ by listing its components $\Con$, $\Ty$, $\Sub$,
  and so on, one per row. Each component has two parts denoted by
  $^\I$ and $^\E$. The $^\I$ part just reuses the corresponding
  components of $\I$, thus the morphism $\blank^\I$ is defined as the
  obvious projection. We omit the equality components, as they come
  from UIP or are trivial. We also omit the components for terms and
  substitutions as their $^\E$ parts are uninformative equational
  reasonings.
  \begin{alignat*}{10}
    & \GG^\I:\Con^\I && \hspace{1em} && \GG^\E:(\nu:\Sub^\I\,\GO\,\GG^\I)\ra \ll\GG\rr^\S\,(\ll\nu\rr^\A\,\omega)\,(\ll\nu\rr^\D\,\omega^D) \\
    & A^\I:\Ty^\I\,\GG^\I && && A^\E:(\nu:\Sub^\I\,\GO\,\GG^\I)(t:\Tm^\I\,\GO\,(A^\I[\nu]))\ra \\
    & && && \hspace{1em} \ll A\rr^\S\,(\GG^\E\,\nu)\,(\ll t\rr^\A\,\omega)\,(\ll t\rr^\D\,\omega^D) \\
    & \sigma^\I:\Sub^\I\,\GG^\I\,\GD^\I && && \sigma^\E:\GD^\E\,(\sigma^\I\circ\nu)=\ll\sigma\rr^\S\,(\GG^\E\,\nu) \\
    & t^\I:\Tm^\I\,\GG^\I\,A^\I && && t^\E : A^\E\,\nu\,(t^\I[\nu])= \ll t\rr^\S\,(\GG^\E\,\nu) \\
    & (A[\sigma])^\I := A^\I[\sigma^\I]^\I && && (A[\sigma])^\E\,\nu\,t := A^\E\,(\sigma^\I\circ\nu)\,t \\
    & \cdot^\I := \cdot^\I && && \cdot^\E\,\nu := \tt \\
    & (\GG\rhd A)^\I := \GG^\I\rhd^\I A^\I && && (\GG\rhd A)^\E\,\nu := (\GG^\E\,(\pi_1\,\nu), A^\E\,(\pi_1\,\nu)\,(\pi_2\,\nu)) \\
    & \U^\I := \U^\I && && \U^\E\nu\,a := \lambda\alpha . _{\ll\alpha\rr^\C\,\id\#}\big(\ll_{\ll\alpha\rr^\C\,\id\#}\alpha\rr^\D\,\omega^D\big) \\
    & (\El\,a)^\I := \El^\I\,a^\I && && (\El\,a)^\E\,\nu\,t := \big(\ll a\rr^\S\,(\GG^\E\,\nu)\,(\ll t\rr^\A\,\omega) \overset{\ll t\rr^\C\,\id}{=} \ll a\rr^\S\,(\GG^\E\,\nu)\,t \overset{a^\E\,\nu}{=} \ll t\rr^\D\,\omega^D\big) \\
    & (\Pi\,a\,B)^\I := \Pi^\I\,a^\I\,B^\I && && (\Pi\,a\,B)^\E\,\nu\,t := \\
    & && && \hspace{1em} \lambda\alpha.{_{\ll\alpha\rr^\C\,\id\#}}\big(B^\E\,(\nu,_{\ll a\rr^\C\,\id,\ll\nu\rr^\C\,\id\#}\alpha)\,(t\oldapp {_{\ll a\rr^\C\,\id,\ll\nu\rr^\C\,\id\#}u})\big) \\
    & (\Pim\,T\,B)^\I := \Pim^\I\,T\,B^\I && && (\Pim\,T\,B)^\E\,\nu\,t := \lambda\alpha.(B\,\alpha)^\E\,\nu\,(t\appm\alpha)
  \end{alignat*}
\end{definition}
\begin{example}
  Given the assumptions $\GO$, $\omega^D$ of $\IE$, we obtain the
  eliminator by
  $\ll\GO\rr_{\IE_{\omega^\D}}\,\id^\I :
  \ll\GO\rr^\S\,\omega\,\omega^D$. The eliminator is a section of
  the displayed algebra $\omega^D$, that is, a dependent function
  together with equalities witnessing that all the operations are
  preserved. If $\GO$ is the syntactic signature for natural
  numbers, we get the eliminator of \Cref{eg:nat_con}.

  For a sort term $a:\Tm^\I\,\GO\,\U$, the interpretation
  $(\ll a\rr_{\IE_{\omega^D}})^\E\,\id$ says that
  $(\lambda\alpha.\ll\alpha\rr^\D\,\omega^D) = \ll
  a\rr^\S\,(\ll\GO\rr^\E\,\id)$, that is, the function for the sort
  $a$ in the eliminator section is the displayed algebra
  interpretation at $\omega^D$ (motives and methods). For natural
  numbers, this is the same as
  $\big(\lambda n.\ll n\rr^D\,(N^D,z^D,s^D)\big) = \big(\lambda
  n.\mathsf{elimNat}\,(N^D,z^D,s^D)\,n)\big)$.

  The interpretation of a constructor term
  $t:\Tm^\I\,\GO\,(\El\,a)$ is uninteresting as it provides an
  equality between two different equality proofs of the computation
  ($\beta$) rule for $t$.
\end{example}

\begin{theorem}\label{th:tosToIITs}
  If a model of type theory supports the theory of IIT signatures,
  then it supports all IITs.
\end{theorem}
\begin{proof}
  For a signature $({sig},{sigc})$, we define constructors as
  \[
    \con_{sig}:= (\ll{sig}\,\I\rr_{\IC_{{sig}\,\I}})^\C\,\id^\I : ({sig}\,\ADS)^\A
  \]
  This typechecks as
  $\ll{sig}\,\I\rr_\A = \ll\blank\rr_\A\,({sig}\,\I)
  \overset{{sigc}}{=} {{sig}\,\A} = ({sig}\,\ADS)^\A$. We define the
  eliminator by
  and an
  \[
    \elim_{sig}\,\gamma^D := (\ll{sig}\,\I\rr_{\IE_{\gamma^D}})^\E\,\id^\I  : ({sig}\,\ADS)^\S\,\con_{sig}\,\gamma^D.
  \]
  This typechecks firstly because the type of $\gamma^D$ matches the
  type of the parameter of $\IE$:
  \[
    ({sig}\,\ADS)^\D\,\con_{sig} \overset{{sigc}}{=}
    (\ll\blank\rr_{\ADS}\,({sig}\,\I))^\D\,\con_{sig} =
    (\ll{sig}\,\I\rr_{\ADS})^\D\,\con_{sig},
  \]
  and the result also has the correct type:
  \[
    \ll{sig}\,\I\rr^\S\,\con_{sig}\,\gamma^D = (\ll\blank\rr_{\ADS}\,({sig}\,\I))^\S\,\con_{sig}\,\gamma^D \overset{{sigc}}{=} ({sig}\,\ADS)^\S\,\con_{sig}\,\gamma^D.
  \]
\end{proof}

%%%%%%%%%%%%%%%%%%%%%%%%%%%%%%%%%%%%%%%%%%%%%%%%%%%%%%%%%%%%%%%%%%%%%%%%%%%

\section{Constructing the Theory of IIT Signatures}
\label{sec:ambroise}

In this section we show that any model of ETT which supports indexed
W-types also supports the theory of signatures, and as a consequence
of \Cref{th:tosToIITs}, all IITs. For this, we work in the internal
language of a model of ETT supporting indexed W-types
\cite{indexedcont}. Indexed W-types correspond to the usual notion of
(possibly mutual) indexed inductive types. We will use Agda-style
notation for defining such inductive families: we list the sorts and
constructors and use pattern matching when eliminating from them. qFor
an encoding of mutual inductive families as families as indexed
W-types, see e.g.\ \cite{mutual}.



% We are also careful to explain where UIP or functional extensionality that is
% available in extensional type theory are used.

The construction consists of the following steps:
\begin{enumerate}
\item Definition of untyped syntax (as a family of inductive datatypes) together
  with typing judgments (as inductive relations on the untyped
  syntax), and construction of a model of the theory of IIT
  signatures from well-formed terms, denoted $S$.
\item Construction of a morphism $\rec:S\to M$ for arbitrary $M$, by:
  \begin{enumerate}
  \item defining a relation $\blank \sim \blank$ between the (well-typed) syntax and a given
    model. The idea is that given a syntactic context $\Gamma$ and a semantic
    context $\model{\Gamma}$ of the model $M$, we have $\Gamma \sim
    \model{\Gamma}$ if and only if $\rec\,\,\Gamma = \model{\Gamma}$, and
    similarly for types, terms, and substitutions;
    % semantic context $\model{\Gamma}$ of
    % the model $M$ if and only if $\Gamma$ relates to $\rec\,{\Gamma}$;
    % enjoyed by the initial morphism from the
    % syntax to a given model;
  \item showing that this relation is functional.
    \end{enumerate}
  \item Proving the uniqueness of this morphism by showing that any morphism
    $f:S\to M$ satisfies the relation. For example, for any syntactic context
    $\Gamma$ we have $\Gamma\sim f\,\Gamma$.
      % TODO donner composantes sur les contextes
\end{enumerate}
The next sections detail each of these steps.
\subsection{Syntactic Model}

The goal is to define the syntactic model where contexts are pairs of a
precontext together with a well-formedness proof, and similarly for types, terms
and substitutions.

Crucially, we do not have conversion relations for typed syntax, nor do we need
to use quotients when giving the syntactic model. This is possible because there
are no $\beta$-rules in the theory of signatures. Hence, we consider only normal
terms in the untyped syntax, and define weakening and substitution by recursion.
Avoiding quotients is important for two reasons. First, it greatly simplifies
formalisation. Second, we aim to reduce IITs to a minimal feature set, and we
get a stronger result if we do not use quotients.

The next sections present the definition of the untyped syntax and the
associated typing judgments.

\subsubsection{Untyped syntax}
The untyped syntax is defined as the following inductive datatype. Variables are
modeled as de Bruijn indices, i.e.\ as natural numbers pointing to a position in
the context. We use the additional default constructor $\erru:\Tmu$ in case of
error (ill-scoped substitution). The typing judgments will not mention $\erru$.
\todo{András: prettify below maybe}
\begin{alignat*}{5}
  & \rlap{(1) Substitution calculus} \\[0.5em]
  & \Conu && : \Set \\
  & \Tyu  && : \Set \\
  & \Subu  && : \Set \\
  % & \Subpre  && : \Set \\
  & \Tmu  && : \Set \\
  & \emptyu && : \Conu \\
    & \epsilonu && : \Subu \\
  & \blank\extu\blank && : \Conu\ra\Tyu\ra\Conu \\
  & \blank\consu\blank && : \Subu\ra\Tmu\ra\Subu \\
  & \varu  && : \N \ra \Tmu \\
  & \rlap{(2) Sorts and operators} \\[0.5em]
  & \Uu && : \Tyu \\
  & \Elu && : \Tmu\ra\Tyu \\
  & \rlap{(3) Inductive parameters} \\[0.5em]
  & \Piu && : \Tmu\ra\Tyu\ra\Tyu \\
  & \blank\appu\blank && : \Tmu\ra\Tmu\ra \Tmu \\
  & \rlap{(4-5) Metatheoretic parameters} \\[0.5em]
  & \Pimu && : (T:\Set)\ra(T\ra\Tyu)\ra\Tyu \\
  & \Piiu && : (T:\Set)\ra(T\ra\Tmu)\ra\Tmu \\
  & \blank\appmu \blank && : \Tmu\ra(\alpha:T) \ra\Tmu\\
  & \rlap{(6) Default value} \\[0.5em]
  & \erru && : \Tmu
\end{alignat*}


% The type of (untyped) substitution is then defined as $\Subu = \List\,\Tmu$.
% Note that $\Conu$ could also be defined as a list of types in a similar fashion.

\subsubsection{Untyped weakening}


% Substitutions of types and terms are defined recursively.


Note that $(\Piu \,A\,B)[\sigma]$ should be defined as $\Piu
\,A[\sigma]\,B[\varu\, 0 \,\consu\, \wkO\,\sigma]$, and thus we need to define
$\wk0$, the weakening of substitutions.  The basic idea is to increment the de
Bruijn indices of all the variables.  Actually this is not so simple because of
the $\Piu$ type: indeed, we want to define $\wk(\Piu\,A\,B)$ as the $\Pi$ type
of the weakening of $A$ and $B$, but here, $B$ must be weakened with respect to
the second last variable of the context, rather than the last one.  For this
reason, we need to generalize the weakening as occuring anywhere in the context.
\begin{alignat*}{5}
  & \wk_n && :  \Tyu\ra\Tyu \\
  & \wk_n && :  \Tmu\ra\Tmu \\
  % & \wk && : \N \ra \N\ra\N \\
  & \wkO && : \Subu\ra \Subu
  \end{alignat*}
  The natural number $n$ specifies at which position of the context the
  weakening occurs.
  Here, $\wkO$ weakens with respect to the last variable.

Later, in Section~\ref{ss:weaken-typing}, we show that weakening preserves
typing. Stating a typing rule for this operation requires to have a way to
express the weakening occuring at the middle of a context. We consider pairs of
untyped contexts, which should be thought of as a splitting of a context at some
position. The full context is recovered by merging the two components:
% (which appears in the typing rule of a weakening occuring in the middle of a context),
% and the length $\length{\Gamma}$ of a
% context $\Gamma$:
\begin{alignat*}{5}
  & \blank \merge \blank && :  \Conu\ra\Conu\ra\Conu \\
  & \Gamma \merge \cdot && :=  \Gamma \\
  & \Gamma \merge (\Delta \extu A) && :=  (\Gamma \merge \Delta)\extu A
\end{alignat*}
We think of the second context as a telescope over the first one. We also define
weakening for telescopes, which will be used to give typing rules for telescopes
in Section \ref{sec:typing_judgments}:
\begin{alignat*}{5}
  & \wkO && : \Conu\ra \Conu
  \\
  & \wkO\,\emptyu && := \emptyu
  \\
  & \wkO\,(\Delta\,\extu A) && := \wkO\,\Delta\,\extu \wk_{\length{\Delta}}\,A
  \end{alignat*}
  \subsubsection{Untyped substitution}
  We define single substitution by recursion on presyntax:
  % & \rlap{(1) Substitution calculus} \\[0.5em]
\begin{alignat*}{5}
  & \blank[\blank := \blank] && : \Tyu\ra\N\ra\Tmu\ra\Tyu \\
  & \blank[\blank := \blank] && : \Tmu\ra\N\ra\Tmu\ra\Tmu
  % \\
  % & \blank[\blank := \blank] && : \N\ra\N\ra\Tmu\ra\Tmu
  \end{alignat*}
  This is enough to define the typing judgments: indeed, the typing rule for application
  involves only a unary substitution.

  However, to construct the initial model of the QIIT, we need to define
  the full substitution calculus:
\begin{alignat*}{5}
  & \blank[\blank] && : \Tyu\ra\Subu\ra\Tyu \\
  & \blank[\blank] && : \Tmu\ra\Subu\ra\Tmu \\
  & \blank \circ \blank && : \Subu\ra\Subu\ra \Subu
  \end{alignat*}
These can be defined either by iterating unary substitutions, or by recursion
on untyped syntax: the two ways yield provably equal definitions. In the
following, we assume that it is defined by recursion. We also make use of the
following definition:
  \begin{alignat*}{5}
  % liftV (S (n + p)) (liftV n q) ≡ liftV n (liftV (n + p) q)
  & \keep && : \Subu \ra \Subu \\
   &&& := \lambda \sigma. \varu\, 0 \,\consu\, \wkO\,\sigma
  \end{alignat*}
  The idea is that if $\sigma$ is a substitution between contexts $\Gamma$ and
  $\Delta$, then $\keep\,\sigma$ is a substitution between contexts $\Gamma, A[\sigma]$
  and $\Delta,A$ for any type $A$. This occurs when defining
  $(\Piu\,A\,B)[\sigma]$ as $\Piu (A[\sigma])(B[\keep \,\sigma])$.




  We can define the identity substitution on a context $\Gamma$ as follows,
  where $\length{\Gamma}$ is the length of the context $\Gamma$, and
  $\keep^{\length{\Gamma}}$ is $\keep$ iterated $\length{\Gamma}$ times:
\begin{alignat*}{5}
  & \length{\Gamma} && : \Conu\ra\N
  \\
  & \idu && : \Conu \ra \Subu \\
  & && := \lambda \Gamma.\keep^{\length{\Gamma}} \epsilonu
  \end{alignat*}


  \subsubsection{Exchange laws: weakening and substitution}
  Many lemmas for types and terms are shown by induction on the untyped syntax.
  Below, $Z$ denotes either a term or a type.
\begin{alignat*}{5}
  % liftV (S (n + p)) (liftV n q) ≡ liftV n (liftV (n + p) q)
  & \wk\mhyphen\wk && : \wk_{n+p+1} (\wk_n\,Z) = \wk_n(\wk_{n+p}\,Z)\\
  % n-subZ-wkZ : ∀ n A z → (lifZZ n A) [ n ↦ z ]Z  ≡ A
  & \wk_n[n] && : (\wk_{n} Z)[ n := z] = Z \\
  % lifZ-l-subV : ∀ n p u x → lifZZ (n + p) (x [ n ↦ u ]V) ≡ (lifZV (S (n + p)) x) [ n ↦ (lifZZ p u) ]V
  & \wk_+[] && : (\wk_{n+p+1} Z)[ n := \wk_p u] = \wk_{n+p}(Z[n := u]) \\
  % l-subV-lifZV : ∀ Δ u n x → (lifZV n x) [ (S (n + Δ)) ↦ u ]V  ≡ lifZZ n (x [ (n + Δ) ↦ u ]V)
  & \wk[+] && : (\wk_{n} Z)[ n+p+1 :=  u] = \wk_{n}(Z[n+p := u]) \\
  % l-subV-l-subV : ∀ n p z u x →  (x [ n ↦ u ]V) [ (n + p) ↦ z ]Z  ≡ (x [ (S (n + p)) ↦ z ]V) [ n ↦ (u [ p ↦ z ]Z) ]Z
  & [][+] && : Z[n :=u][n+p:=z] = Z[n+p+1:=z][n:= (u[p:=z])]
  \end{alignat*}
  Below, $\keep^n$ denotes $\keep$ iterated $n$ times.
\begin{alignat*}{5}
  % liftV (S (n + p)) (liftV n q) ≡ liftV n (liftV (n + p) q)
  % liftV=wkS
  & [\keep^n\mhyphen \wkO] && : Z[\keep^n (\wkO\,\sigma)] = \wk_n (Z[\keep^n\,\sigma])
  \\
  % lifZₛV : ∀ n xp σp Zp → (lifZV n xp [ iZer n keep (Zp ∷ σp) ]V) ≡ (xp [ iZer n keep σp ]V)
  & \wk_n[\keep^n\mhyphen\cons]  & &
  :
  (\wk_n Z) [\keep^n (u \consu \sigma)] = Z [ \keep^n \sigma]
  \\
  % l-sub[]V : ∀ n x z σ
    % ((x [ n := z ]V) [ iZer n keep σ ]Z) = (x [ iZer (S n) keep σ ]V) [ n := (z [ σ ]Z) ]Z
   & [:=][\keep] && : Z [ n:= u][\keep^n \sigma] =  Z[\keep^{n+1}\,\sigma][n := u[\sigma]]
  \end{alignat*}
As particular cases for $n=0$, we get
\begin{alignat*}{5}
  & {}{\circ}{}\wkO & & : \sigma \circ (\wkO \tau) = \wkO (\sigma \circ \tau) \\
  & \wkO{}{\circ}{}, && : \wkO\,\sigma \circ (t \consu \tau ) = \sigma \circ \tau \\
  & [\wkO] && : t[\wkO\,\sigma] = \wk_0 (t[\sigma]) \\
  & \wk_0[\cons] && : (\wk_0 Z)[u \consu \sigma] = Z[\sigma] \\
  & [0:=][] && : Z [ 0:= u][ \sigma] =  Z[\keep\,\sigma][0 := u[\sigma]]
\end{alignat*}
Finally, we show the following:
\begin{alignat*}{5}
  & [][] && : Z[\sigma][\tau] = Z[\sigma\circ\tau]
  \\
  & \ass && : (\sigma\circ \delta)\circ\tau = \sigma\circ (\delta\circ\tau)
  \end{alignat*}
We defer laws for identity substitutions after the definition of the typing
judgments, as the proofs require that some inputs are well-typed.

  % The two ways yield provably equal definitions, but
  % The advantage of the second
  % one in a type theory with UIP and function extensionality is that we get
  % definitional equalities such as $(t\,\appu\,u)[\sigma]=\app\,t[\sigma]\,u[\sigma]$.

  % \\
  % & \blank[\blank := \blank] && : \N\ra\N\ra\Tmu\ra\Tmu


  \subsubsection{Typing judgments}
  \label{sec:typing_judgments}
  The typing judgments are defined as the following inductive datatype indexed over the
  untyped syntax:
\begin{alignat*}{5}
  & \rlap{(1) Substitution calculus} \\[0.5em]
  & \blank \vdash && : \Conu\ra \Set \\
  & \blank \vdash \blank  && : \Conu\ra \Tyu\ra\Set \\
  & \blank \vdash \blank \in \blank  && : \Conu\ra \Tmu\ra \Tyu\ra \Set \\
  & \blank \vdash \blank \invar \blank  && : \Conu\ra \N \ra \Tyu\ra \Set \\
  & \blank \vdash \blank \Rightarrow \blank  && : \Conu\ra \Subu \ra \Conu\ra \Set \\
  % & \Subpre  && : \Set \\
  & \emptyw && : \emptyu \vdash \\
  & \epsilonw && : \Gamma\vdash \epsilonu \Rightarrow \emptyu \\
  & \blank\extw\blank && : ( \Gamma\vdash )  \ra  (\Gamma\vdash A)  \ra
  \Gamma\extu A \vdash \\
  & \consw && :
    (\Delta \vdash) \ra
    (\Gamma\vdash \sigma \Rightarrow\Delta)\ra
    ( \Delta \vdash A) \ra
    ( \Gamma \vdash t \in A[\sigma]) \ra
    \Gamma \vdash t \consu \sigma \Rightarrow \Delta\extu A
   \\
  & \varw  && : (\Gamma \vdash n \invar A) \ra \Gamma \vdash \varu n \in A \\
  & \vzw  && : (\Gamma\vdash)\ra(\Gamma\vdash A)\ra\Gamma \extu  A \vdash 0 \invar \wku \, A \\
  & \vsw  && : (\Gamma\vdash)\ra(\Gamma\vdash A)\ra(\Gamma \vdash n \invar A) \ra (\Gamma \vdash B) \ra \Gamma \extu  B
  \vdash S\,n \invar \wku \, A \\
  & \rlap{(2) Sorts and operators} \\[0.5em]
  & \Uw && : (\Gamma \vdash)\ra \Gamma \vdash \Uu \\
  & \Elw && : (\Gamma \vdash)\ra (\Gamma \vdash a \in \Uu)\ra\Gamma \vdash \Elu\,a \\
  & \rlap{(3) Inudctive parameters} \\[0.5em]
  & \Piw && :
    (\Gamma \vdash)\ra (\Gamma \vdash a \in \Uu)\ra(\Gamma \extu
  \Elu\,a \vdash B)\ra\Gamma \vdash \Piu \, a \, B \\
  & \appw && :
    (\Gamma \vdash)\ra (\Gamma \vdash a \in \Uu)\ra(\Gamma \extu
    \Elu\,a \vdash B)
    \\
    & &&
    \ra(\Gamma \vdash t \in \Piu \, a \, B )
    \ra
    (\Gamma \vdash u \in \Elu\,a)
    \ra
    \Gamma \vdash t \appu u \in  B [0 := u] \\
  & \rlap{(4) Inductive parameters} \\[0.5em]
  & \Pimw && :
    (T:\Set)\ra(A : T\ra \Tyu)\ra(\Gamma \vdash)\ra
    ((t : T)\ra\Gamma\vdash A\,t) \ra
    \Gamma \vdash \Pimu \,T\,A
    \\
    & \appmw  && :
    (T:\Set)\ra(A : T\ra \Tyu)\ra(\Gamma \vdash)\ra
    ((t : T)\ra\Gamma\vdash A\,t)
    \\
    & &&
    \ra(\Gamma \vdash t \in \Pimu \,T\,A)
    \ra (u : T) \ra \Gamma \vdash t\,\appmu\,u \in A\,u
\end{alignat*}
There is possibility of redundancy in the arguments of the constructors. Here,
we are ``{paranoid}'', so that we get more inductive hypotheses when performing
recursion.

% The substitution part could be moved after the mutual definition of terms, types,
% and contexts.

\subsubsection{Weakening preserves typing}
\label{ss:weaken-typing}
We prove by mutual induction that typing judgments are stable under weakening,
for contexts, types, terms, and substitutions:
% wkTelw : ∀ {Γp}{Ap}(Aw : Γp ⊢ Ap)Δp (Δw : (Γp ^^ Δp) ⊢) → ((Γp ▶p Ap) ^^ wkTel Δp) ⊢
% liftTw : ∀ {Γp}{Ap}(Aw : Γp ⊢ Ap)Δp{Bp}(Bw : (Γp ^^ Δp) ⊢ Bp) → ((Γp ▶p Ap) ^^ wkTel Δp) ⊢ (liftT ∣ Δp ∣ Bp)
\begin{alignat*}{5}
  & \wf{\wkO} && : (\Gamma \vdash A)\ra(\Gamma \merge \Delta\vdash) \ra
  \Gamma \extu A \merge \wkO\, \Delta \vdash \\
  % liftV (S (n + p)) (liftV n q) ≡ liftV n (liftV (n + p) q)
  & \wf{\wk} && : (\Gamma \vdash A)\ra(\Gamma \merge \Delta\vdash B) \ra
  \Gamma \extu A \merge \wkO\,\Delta \vdash \wk_{\length{\Delta}}\, B  \\
  & \wf{\wk} && : (\Gamma \vdash A)\ra(\Gamma \merge \Delta\vdash t\in B) \ra
  \Gamma \extu A \merge \wkO\,\Delta \vdash \wk_{\length{\Delta}}\, t \in \wk_{\length{\Delta}}\,B  \\
  & \wf{\wkO} && : (\Gamma \vdash A)\ra(\Gamma \vdash \sigma
  \Rightarrow \Delta) \ra
  \Gamma \extu A \vdash \wkO\, \sigma \Rightarrow \Delta
  \end{alignat*}
\subsubsection{Substitution preserves typing}
  First, we show that judgments are stable under substitution.
\begin{alignat*}{5}
  % Tyw[] : ∀ {Γp}{Ap}(Aw : Γp ⊢ Ap) {Δp}(Δw : Δp ⊢){σp}(σw :  Δp ⊢ σp ⇒ Γp) → Δp ⊢ (Ap [ σp ]T)
  & \wf{[]} && : (\Gamma \vdash)\ra(\Delta \vdash A)\ra
  (\Gamma \vdash \sigma \Rightarrow \Delta) \ra
  \Gamma \vdash A [ \sigma ]
  \\
  & \wf{[]} && : (\Gamma \vdash)\ra(\Delta \vdash t \in A)\ra
  (\Gamma \vdash \sigma \Rightarrow \Delta) \ra
  \Gamma \vdash t [\sigma] \in A [ \sigma ]
  \\
  & \wf{[]} && : (\Delta \vdash x \invar A)\ra
  (\Gamma \vdash \sigma \Rightarrow \Delta) \ra
  \Gamma \vdash x [\sigma] \in A [ \sigma ]
  % ∘w : ∀ {Γ} {Δ σ} (σw :  Δ ⊢ σ ⇒ Γ)
  %  {Y}(Yw : Y ⊢) {δ} (δw :  Y ⊢ δ ⇒ Δ) →
  %  Y ⊢  (σ ∘p δ) ⇒ Γ
  \\
  & \wf{{\circ}} && :
  (\Gamma \vdash) \ra
  (\Gamma \vdash \sigma \Rightarrow \Delta) \ra
  (\Delta \vdash \tau \Rightarrow E) \ra
  \Gamma \vdash \tau \circ \sigma \Rightarrow E
  % keepw : ∀ {Γp}(Γw : Γp ⊢){Δp}(Δw : Δp ⊢){σp}(σw :  Γp ⊢ σp ⇒ Δp) {Ap}(Aw : Δp ⊢ Ap ∈ Up ) → (Γp ▶p (Elp Ap [ σp ]T )) ⊢ (keep σp) ⇒ (Δp ▶p Elp Ap)
  % & \wf{keep-\El} && : (\Gamma \vdash)\ra(\Delta \vdash A)\ra
  % (\Gamma \vdash \sigma \Rightarrow \Delta) \ra
  % \Gamma \vdash A [ \sigma ]
  % \\
  \end{alignat*}
\subsubsection{Laws for identity substitutions}
We show category and functor laws involving identity substitution, for
well-formed types, terms and substitutions.
\begin{alignat*}{5}
  % [idp]V : ∀ {Γ}{A}{x}(xw : Γ ⊢ x ∈v A) → (x [ idp ∣ Γ ∣ ]V) ≡ V x
  & [\idu] && : (\Gamma \vdash A)\ra A [ \idu\,\Gamma]  \\
  & [\idu] && : (\Gamma \vdash x \invar A)\ra x [ \idu\,\Gamma] = V x \\
  & [\idu] && : (\Gamma \vdash t \in A)\ra t [ \idu\,\Gamma] = t \\
  & \idru && : (\Gamma \vdash \sigma \Rightarrow\Delta)\ra \sigma \circ \idu\,\Gamma = \sigma \\
  & \idlu && : (\Gamma \vdash \sigma \Rightarrow\Delta)\ra \idu\,\Delta\circ \sigma = \sigma
  \end{alignat*}
Finally, we show that the identity substitution itself is well-typed:
  \begin{alignat*}{5}
  & \idw && : (\Gamma \vdash)\ra \Gamma \vdash \idu\,\Gamma \Rightarrow \Gamma
  \end{alignat*}



\subsubsection{Proof irrelevance and unicity of typing}
\label{ss:uniq-typing-types}
A type is a proposition, or proof-irrelevant, if it has at most one inhabitant.
  \[
    \isaprop\,T := (a : T) \ra (a' : T) \ra a = a'
  \]
We show that each of the typing judgments is unique in the
following sense:

\todo{Prettify table.}
\begin{align*}
    \isp{\Conw}
  : \isaprop\, (\Gamma\vdash)
 & &
    \isp{\Tmw}
  : \isaprop\, (\Gamma\vdash t \in A)
    \\
    \isp{\Tyw}
  : \isaprop\, (\Gamma\vdash A)
    &&
    \isp{\wf\Var}
  : \isaprop\, (\Gamma\vdash x \invar A)
  \\
  &
        \isp{\wf\Sub}
  : \isaprop\, (\Gamma\vdash \sigma \Rightarrow \Delta)
  \end{align*}
We also show unicity of typing:
\begin{alignat*}{5}
  &
  \Tmw{=}\Ty & &:
  (\Gamma \vdash t \in A) \ra
  (\Gamma \vdash t \in B) \ra A = B
  \\
  &
  \wf\Var{=}\Ty & &:
  (\Gamma \vdash x \invar A) \ra
  (\Gamma \vdash x \invar B) \ra A = B
  \end{alignat*}
Let us consider for instance the application constructor $\appw$: for a codomain
type $B$ it yields an overall type $C=B[0 := u]$ for an application. Even
if $C$ is known a priori, there may be another $B$ for which $B[0 := u] = C$,
possibly leading to many proofs that $t\appu u$ has type $C$. Unicity of typing
solves this issue, as $B$ is then uniquely determined by the type $\Piu\,A\,B$
of $t$.

%% Actually, the application and successor cases require to show uniqueness of
%% types in a given context, before.
%% Let us consider for instance the application constructor $\appw$: it provides a
%% type $B$ such that the final type is $C=B[0 := u]$. Even if $C$ is known a
%% priori, there may be another $B$ for which $B[0 := u] = C$, possibly leading to
%% many proofs that $t\appu u$ has type $C$. Uniqueness of types
%% solves this issue, as $B$ is then uniquely determined by the type $\Piu\,A\,B$ of $t$.


% \subsubsection{UIP and functional extensionality}
 % , which is not applyable without
 % functional extensionality.

 \subsubsection{Syntactic model}
 The syntactic model is obtained by packing the untyped syntax with the typing
 judgments:
% % \subsection{QIIT model}
% Then, we show that they are stable by weakening and substitution.
% Packing the untyped syntax with well-formed typed judgments yields
% a model $S$ of the QIIT. Let us give the most important components:
\begin{alignat*}{5}
 & \syn\Con && := \sum_\Gamma \Gamma\vdash
 \\
 & \syn\Ty \,(\Gamma,\wf\Gamma)&& := \sum_A \Gamma\vdash A
 \\
 & \syn\Tm \,(\Gamma,\wf\Gamma)(A,\wf A)&& := \sum_t \Gamma\vdash t \in A
 \\
 & \syn\Sub \,(\Gamma,\wf\Gamma)(\Delta,\wf \Delta)&& := \sum_\sigma \Gamma\vdash \sigma \Rightarrow \Delta
\end{alignat*}
The other fields are given straightforwardly.  Regarding the equations, it is
enough to prove them only for the untyped syntactic part: as we argued in
Section~\ref{ss:uniq-typing-types}, the proofs of typing judgments are
automatically equal.

None of the constructions in this section use UIP. Functional extensionality is
necessary because the untyped metatheoretic $\Pi$ takes a metatheoretic function
as an argument. An example induction step that uses it is in the type
preservation proof for identity substitutions (this is an equation satisfied by
a model of the QIIT), in particular in the case $(\Pim\,T\,
A)[\id]=\Pim\,T\,A$. Indeed, the left hand side of this equation is equal to
$\Pim\,T\,(\lambda t.(A\,t)[\id])$ by definition, whereas the induction
hypothesis states that $(t:T)\ra (A\,t)[\id]=A\,t$.

\subsection{Relating the Syntax to a Model}
It remains to show that the constructed syntactic model is initial. To achieve
this, we first define a relation between the syntactic model and an arbitrary
model, then show that the relation is functional, which lets us extract a
homomorphism from the relation.

This approach is an alternative version of Streicher's method for interpreting
preterms in an arbitrary model\cite{streichersemantics}. Streicher first defines
a family of partial maps from the presyntax to a model, then shows that the maps
are total on well-formed input. We have found that our approach is significantly
easier to formalise. Too see why, note that the right notion of partial map in
type theory, which does not presume decidable definedness, is fairly
heavyweight:
\[
  \mathsf{PartialMap}\,A\,B := A\ra (P : \mathsf{hProp})\times(P \ra B)
\]
Here $\mathsf{hProp}$ denotes a type which is propositional. In the above
definition, we notice an opportunity for converting a fibered definition of a
type family into an indexed one; if we drop the propositionality for $P$ for the
time being, we may equivalently return a family indexed over $B$, which is
exactly just a relation $A\ra B\ra\Set$. Then, in our approach, we recover
uniqueness of $P$ through the functionality requirement on the $A\ra B\ra\Set$
relation, and totality by already assuming well-formedness of $A$. In type
theory, using indexed families instead of display maps is a common convenience,
since the former are natively supported in type theory, while the latter require
carrying around auxiliary propositional equalities.

\subsubsection{The functional relation}

Given a model $M$ of the QIIT, we define the functional relation satisfied by
the initial morphism $\rec : S\to M$ by recursion on the typing judgments.
% As these judgments are propositions, we will sometimes
If $\Gamma$ is a context in $S$ and $\model{\Gamma}$ is a semantic context
(i.e.\ a context of the model $M$), we want to define a type
$\Gamma\sim \model{\Gamma}$ equivalent to $\rec\,\,\Gamma=\model{\Gamma}$. Of
course, at this stage, $\rec$ is not available yet since the point of
defining this relation is to construct $\rec$ in the end.

For a type $A$ in a context $\Gamma$, we want to define a relation
$A\sim \model{A}$ that is equivalent to $\rec\,\,A=\model{A}$.
% where $\model{A}$ is a
% semantic type in a semantic context $\model{\Gamma}$.
For this equality to make sense, the semantic type $\model{A}$ must live
in the semantic context $\rec\,\,\Gamma$. But again, $\rec$ is not
yet available at this stage. Exploiting the expected equivalence between
$\Gamma\sim\model\Gamma$ and $\rec\,\,\Gamma=\model\Gamma$, we may consider defining $A\sim\model{A}$ under the
hypotheses that $\model{A}$ lies in a semantics context $\model\Gamma$ which is
related to $\Gamma$. Then, the type of the relation for types is
\[
  (\Gamma : \syn\Con)\ra (A : \syn\Ty\syn\Gamma) \ra
  (\model\Gamma : \model\Con)
  \ra
  (\Gamma \sim \model\Gamma)
  \ra
  (\model{A}:\model\Ty\model\Gamma)
  \ra
  \Set
\]
Note that the relation on contexts
must be defined mutually with the relation on types (see for example the case of
context extension), but here, the relation on contexts appears as the type of an
argument of the relation on types.
% We leave this possibility aside,
Our metatheory does not support this kind of recursive-recursive
definitions\footnote{And to our knowledge there is no established semantics for
it in existing literature.}, so we instead just remove the hypothesis
$\Gamma\sim\model\Gamma$ from the list of arguments.  We proceed similarly for
terms and substitutions. Actually, this removal is not without harm. For
example, consider relating the empty substitution
$\Gamma\vdash \epsilonu \Rightarrow \emptyu $ to a semantic substitution
$\model\sigma : \model\Sub\,\model\Gamma\,\model\Delta$. We would like to assert
that $\model\sigma$ equals the empty semantic substitution $\model\epsilon$, but
this is not possible because typechecking requires that $\model\Delta$ is the
empty semantic context. This is precisely what ensured the removed hypothesis
$\syn\cdot \sim \model\Delta$.  Our way out here is to state that $\model\sigma$
is related to the empty substitution if the target semantic context $\model\Delta$
is empty, and, acknowledging this equality, if $\model\sigma$ is the empty
substitution.
% The idea is that we will first prove that there is atmost one semantic type
% which relates to the syntactic one, and then later we will provide such a
% semantic type under the hypothesis that the contexts are related.

Let us mention another possible solution for avoiding recursion-recursion:
defining $A\sim\model{A}$ so that it is equivalent to $(e:\rec\,\Gamma
= \model{\Gamma})\times (\rec\,A = e\#\model{A})$.
% whereas the right hand side is of type $\model\Ty\model{\Gamma}$.
% For a substitution $\sigma$ between contexts $\Gamma$ and $\Delta$,
% a first idea consists in defining a relation $\sigma \sim_{\blank} \model{\sigma} : \Gamma
% \sim \model\Gamma \to \Set $ where $\model{A}$ is a semantic type in the
% semantic context $\model{\Gamma}$. We define the relation on types under the hypothesis
% that the contexts are already related.
% it is natural to define the relation with a semantic type $\model{A}$ in a
% semantic context $\model{\Gamma}$ such that it is equivalent to  $(\rec\,\Gamma = \model{\Gamma})\times (\rec\,A = \model{A})$ (note that the first equality is
% required for the second one to be well-typed).
% The discussion is similar for terms.
% Then, the ultimate goal is to prove that
% $\sum_{\model{\Gamma}}\Gamma\sim\model{\Gamma}$ and
% $\sum_{\model{\Gamma}}({\model{A}:\model{\Ty}\,\model{\Gamma}})\times(A\sim\model{A})$
% are contractible, i.e., have a unique inhabitant.
%The first step consists in showing that these types are propositions.
In contrast to this, our approach
% Actually, we take a different approach for the relation on types
% (and terms) which
yields a more concise definition of the relation.
For example, in the case of the universe, this would lead
to the definition
  $  \Uw\wf{\Gamma} \sim \model{A} := (\wf{\Gamma}\sim\model{\Gamma}) \times (\model{A} = \model{\U})$,
  instead of
  our definition
  $  \Uw\wf{\Gamma} \sim \model{A} := (\model{A} = \model{\U})$.
% Recall that our adapted goal, for types, is to prove that
% $\sum_{\model{A}}A\sim\model{A}$ is a proposition, and
% is inhabited if $\Gamma \sim \model{\Gamma}$ is.
%  This means that in the definition of the relation, we assume
% that contexts are already related, so that we don't need to enforce them to be,
% as it is the case in the original approach.
% We give the two versions of the universe
% case below, to explain the differences.


% Many of the terms below are well-typed because the target model satisfies some
% equalities that are reflected as definitional equalities.
% In the formalization, we even postulate rewrite rules for some of the equations that the
% model $M$ should satisfy (otherwise, we are faced with too many hellish transports).
% This is, in some sense, another place where UIP is needed, as it is known that
% extensional type theory can be translated in intensional type theory using UIP
% and functional extensionality. This requirement can be qualified: we
% could say that we only claim to eliminate to models that definitionally
% satisfies the equalities that we enforce, although we haven't checked that this is enough to
% construct any IIT from the universal one (and anyway, UIP and functional
% extensionality are needed to construct any IIT from the universal one).

Here we provide the definition of the relation by recursion on the typing judgments.
In the definitions, we abbreviate $\relT{\wf{A}}{\model\Gamma}{\model{A}}$ by
$\wf{A}\sim\model{A}$ when $\model{\Gamma}$ can be inferred, and similarly for
terms and substitutions.
\begin{alignat*}{5}
  & \rlap{(1) Substitution calculus} \\[0.5em]
  & \blank \sim{ \blank} && :  \Gamma \vdash \, \ra \model{\Con} \ra \Set \\
  & \relT{\blank}{\model{\Gamma}}{\blank} && :  \Gamma \vdash A \, \ra \model{\Ty}\,\model{\Gamma} \ra \Set \\
  & \relt{\blank}{\model{\Gamma}}{\model{A}}{\blank} && :  \Gamma \vdash t \in A \, \ra \model{\Tm}\,\model{\Gamma}\,\model{A} \ra \Set \\
  & \relV{\blank}{\model{\Gamma}}{\model{A}}{\blank}&& :  \Gamma \vdash x \invar A \, \ra \model{\Tm}\,\model{\Gamma}\,\model{A} \ra \Set \\
  & \relS{\blank}{\model{\Gamma}}{\model{\Delta}}{ \blank} && :  \Gamma \vdash \sigma \Rightarrow \Delta \, \ra \model{\Sub}\,\model{\Gamma}\,\model{\Delta} \ra \Set \\
  \\
  & \emptyw\sim \model{\Gamma} && := \model{\Gamma}=\model{\cdot} \\
  & \relS{\epsilonw}{\model{\Gamma}}{\model{E}}{\model{\delta}}
  % (  \vdash \model{\delta}\Rightarrow \model{E})
  &&
 := (e_E : \model{E} = \model{\cdot})\times (\model{\delta} = e_E \#
 \model{\epsilon})
  \\
  & (\wf{\Gamma}\extw \wf{A})\sim \model{\Delta} &&
   :=
    \sum_{\model{\Gamma}} (\wf{\Gamma}\sim\model{\Gamma}) \times
    \sum_{\model{A}} (\wf{A}\sim\model{A}) \times
    \\
    & && \qquad
    (\model{\Delta} = \model{\Gamma} \model{\ext} \model{A})
    \\
    &
    \relS{(\consw \wf{\Delta}\wf{\sigma}\wf{A}\wf{t})}{\model{\Gamma}}{\model{E}}{\model{\delta}}
    % (\consw \wf{\Delta}\wf{\sigma}\wf{A}\wf{t})\sim
    % \\
    % & \qquad
    % ( \model{\Gamma} \vdash \model{\delta}\Rightarrow \model{E})
    &&
   :=
    \sum_{\model{\Delta}} (\wf{\Delta}\sim\model{\Delta}) \times
    \sum_{\model{\sigma}} (\wf{\sigma}\sim\model{\sigma}) \times
    \\
    & && \qquad
    \sum_{\model{A}} (\wf{A}\sim\model{A}) \times
    \sum_{\model{t}} (\wf{t}\sim\model{t}) \times
    \\
    & && \qquad
    (e_E : \model{E} = \model{\Delta} \model{\ext} \model{A}) \times
    \\
    & && \qquad
    (\delta = e_E \# \model\sigma \model{\cons} \model{t} )
    \\
  & \varw\,\wf{x} \sim \model{t}
   && := \wf{x}\sim \model{t} \\
   & \relV{\vzw\wf{\Gamma}\wf{A}}{\model{\Delta}}{\model{B}}{\model{t}}
  %  \vzw\wf{\Gamma}\wf{A}
  % \sim \\
  % & \qquad (\model{\Delta}\vdash \model{t}\in \model{B} )
   && :=
    \sum_{\model{\Gamma}} (\wf{\Gamma}\sim\model{\Gamma}) \times
    \sum_{\model{A}} (\wf{A}\sim\model{A}) \times \\
    & &&
    \qquad
  (e_\Delta : \model{\Delta} = \model{\Gamma} \model{\ext} \model{A})
     \times
    \\
    & && \qquad
     (e_B : \model{B} = e_\Delta \# \model{\wk}\,\model{A}) \times
     % \\
     % & && \qquad
     (\model{t} = e_\Delta,e_B \# \model{\vz})
  \\
  &
   \relV{\vsw \wf{\Gamma} \wf{A} \wf{n} \wf{B}}{\model\Delta}{\model{C}}{\model{t}}
  % \vsw \wf{\Gamma} \wf{A} \wf{n} \wf{B} \sim
  % \\
  % & \qquad
  %    (\model{\Delta}\vdash \model{t}\in \model{C} )
   && :=
    \sum_{\model{\Gamma}} (\wf{\Gamma}\sim\model{\Gamma}) \times
    \sum_{\model{A}} (\wf{A}\sim\model{A}) \times
    \\
    & &&\qquad
    \sum_{\model{B}} (\wf{B}\sim\model{B}) \times
    \sum_{\model{n}} (\wf{n}\sim\model{n}) \times
    \\
    & &&
    \qquad
  (e_\Delta : \model{\Delta} = \model{\Gamma} \model{\ext} \model{B})
     \times
    \\
    & && \qquad
     (e_C : \model{C} = e_\Delta \# \model{\wk}\,\model{A}) \times
     % \\
     % & && \qquad
     (\model{t} = e_\Delta,e_C \# \model{\vs} \, \model{n})
  \\
  & \rlap{(2) Sorts and operators} \\[0.5em]
  & \Uw \wf{\Gamma} \wf{A} \sim \model{A} &&
   := \model{A} = \model{\U}
  \\
  & \Elw \wf{\Gamma} \wf{a} \sim \model{A}
  && :=
  \sum_{\model{a}} (\wf{a}\sim  \model{a}) \times
    (\model{A} = \model{\El}\,\model{a})
  \\
  & \rlap{(3) Parameters} \\[0.5em]
  & \Piw \wf{\Gamma} \wf{a} \wf{B} \sim \model{C} &&
   :=
      \sum_{\model{a}} (\wf{a}\sim\model{a})
      \times
      \sum_{\model{B}} (\wf{B}\sim\model{B})
      \\
      &&& \qquad \times (\model{C}= \model{\Pi}\,\model{a}\,\model{B})
  \\
  &
  \relt{\appw \wf{\Gamma} \wf{a} \wf{B} \wf{t} \wf{u}}{\model{\Gamma}}{\model{C}}{\model{x}}  &&
  % & \appw \wf{\Gamma} \wf{a} \wf{B} \wf{t} \wf{u} \sim \\
  % & \qquad (\model{\Gamma}\vdash x\in \model{C})  &&
    :=
      \sum_{\model{a}} (\wf{a}\sim\model{a})
      \times
      \sum_{\model{B}} (\wf{B}\sim\model{B})
      \times
      \\
      & && \qquad
      \sum_{\model{t}} (\wf{t}\sim\model{t})
      \times
      \sum_{\model{u}} (\wf{u}\sim\model{u})
      \times
      \\
      & &&
     \qquad  (e_C : \model{C} = \model{B}\model{[0 := \model{u}]})
      \times
      \\
      & &&
      \qquad
      (\model{x} = e_C \# \model{t} \model{\oldapp}\model{u})
    \\
  & \rlap{(4) Metatheoretic parameters} \\[0.5em]
  & \Pimw T\,A\,\wf{\Gamma}\wf{A} \sim \model{B} &&
    :=
      \sum_{\model{A}} ((t : T) \ra \wf{A}\sim\model{A}\, t)
      \times
      (\model{B} = \model{\Pim}\,T\,\model{A})
    \\
    & \relt{\appmw T\,A\,\wf{\Gamma}\wf{A} \wf{t} u}{\model{\Gamma}}{\model{B}}{\model{x}}
    % & \appmw T\,A\,\wf{\Gamma}\wf{A} \wf{t} u \sim
    % \\
    % & \qquad
    % (\model{\Gamma}\vdash x\in \model{B})
     &&
     :=
      \sum_{\model{A}} ((t : T) \ra \wf{A}\sim\model{A}\, t)
      \times
      \sum_{\model{t}} (\wf{t}\sim\model{t})
      \times
      \\ & &&
      \qquad
      (e_B : \model{B} = \model{\Pim}\,T\,\model{A})
      \times
      (\model{x} = e_B \# \model{t}\model{\hat{\oldapp}}u)
\end{alignat*}

% As discussed above, when writing the definition of the type components, we assume that the input semantic context is already
% related to the typing judgment of the syntactic context.
% The first version
% of the relation that we suggested at the beginning of the section
% would rather enforce them to be related.
% In the case of $\U$, this would lead
% to the definition
%   $  \Uw\wf{\Gamma} \sim \model{A} := (\wf{\Gamma}\sim\model{\Gamma}) \times (\model{A} = \model{\U})$,
%   instead of
%   the current definition
%   $  \Uw\wf{\Gamma} \sim \model{A} := (\model{A} = \model{\U})$.
%   Our choice makes the definitions more concise, and similarly
%  in the definition of the term components we assume that the input semantic
%  context and type are already related to their associated well-formedness judgments.
 % However, for some fields, we cannot avoid the first approach.
  % Unfortunately, we need sometimes to require some equalities that are actually
  % redundant with this external assumption.
  %, although we know that these equalities are
  % already satisfied by our assumption.
  % As we argued before, in the  of empty substitution $\epsilon$: we require the equality
  % $(e_C : \model{E} = \model{\cdot})$, although
  % our claimed external assumption that $\model{E}$ is  related to the
  % canonical proof $\epsilonw$ of the typing judgment $\epsilon \vdash$ should imply
  % it.
% Let us comment the first variable case:
  % Another example is the first variable case, where we follow the first verbose
  % approach:
  % the relation gives a semantic context $\model\Gamma$ that must be related
  % to the syntactic $\Gamma$, and a semantic
  % type $\model{A}$ related to the syntactic $A$. The input semantic context is
  % then required to equal $\model{\Gamma}\model{\ext}\model{A}$.   As
  % $\blank \model{\ext}\blank$ is not guaranteed to be injective, these
  % relations are required to show that the relation is right unique.
Next, we prove that this relation is right unique.
Then, we show that the relation is stable under weakening and substitution.
  The last step consists of giving a related semantic counterpart to any
  well-typed context, type or term.
  Everything is done by induction on the typing judgments.
  \subsubsection{Right uniqueness}
  \label{sec:right_uniqueness}
  We show by recursion that the relation is right unique in the following sense:
  \begin{alignat*}{5}
    &
    \isp{\Sigma{\sim}}
    && : (\wf\Gamma : \Gamma \vdash ) \ra
    \isaprop\, (\sum_{\model{\Gamma}} \wf{\Gamma} \sim \model{\Gamma})
    \\
    &
    \isp{\Sigma{\sim}}
    && : (\wf A : \Gamma \vdash A) \ra
    \isaprop\, (\sum_{\model{A}} \wf{A} \sim \model{A})
    \\
    &
    \isp{\Sigma{\sim}}
    && : (\wf t : \Gamma \vdash t \in A) \ra
    \isaprop\, (\sum_{\model{t}} \wf{t} \sim \model{t})
    \\
    &
    \isp{\Sigma{\sim}}
    && : (\wf x : \Gamma \vdash x \invar A) \ra
    \isaprop\, (\sum_{\model{x}} \wf{x} \sim \model{x})
    \\
    &
    \isp{\Sigma{\sim}}
    && : (\wf \sigma : \Gamma \vdash \sigma \Rightarrow \Delta) \ra
    \isaprop\, (\sum_{\model{\sigma}} \wf{\sigma} \sim \model{\sigma})
  \end{alignat*}
  We mentioned that in order to avoid a recursive-recursive definition, we
  removed some hypotheses in the list of arguments of the relation. Such
  hypotheses are sometimes missed, for example in the case of the empty
  substitution or in the case of variables, requiring us to state additional
  equalities. Because of this, we need UIP to show that
  $\sum_{\model{\Gamma}}\Gamma\sim\model{\Gamma}$ and
  $\sum_{\model{A}}A\sim\model{A}$ are propositions.  One may think that the use
  of UIP could be avoided by using the alternative verbose definition that we
  suggested before, expecting that
  $\sum_{\model{\Gamma}}\sum_{\model{A}}A\sim\model{A}$, rather than
  $\sum_{\model{A}}A\sim\model{A}$, is a proposition.  However, this is not
  obvious. For example, we were not able to define
  $\Elw\wf{\Gamma}\wf{a}\sim \model{A}$ in this fashion. In related work,
  Hugunin investigated constructing IITs without
  UIP\cite{hugunin2019constructing}, and demonstrated that well-formedness
  predicates used in syntactic models can be subtly incompatible with UIP. Also,
  while Hugunin does not use UIP, he only shows a weak version of dependent
  elimination for the constructed IITs. Hence, the question whether IITs are
  reducible to inductive types in a UIP-free setting remains open.




%% in this fashion so that
%%   UIP can be avoided to prove this statement.


  % \todo[inline]{András: Hugunin's paper has an example (which is
  %    likely generalizable) that a notion of redundancy in
  %    well-formedness relations implies that UIP is required.}
  \subsubsection{Stability under weakening}
  % Let us argue that it is necessary to show s
  Stability of the relation under
  weakening must be proved before stability under substitution.
  Indeed, in the proof of stability under substitution, the $\Pi$ case
  requires to show that $\Pi \,A[\sigma] \,B[\keep\,\sigma]$ is related to
  $\model{\Pi}\,\model{A}\model{[\sigma]}
  \model{A}\model{[\model{\keep}\,\sigma]}$.
  We would like to apply the induction hypothesis, so we need to show that
  $\keep\,\sigma=\varu\,0\consu\wkO\,\sigma$ is related to
  $\model{\keep}\,\model{\sigma}$, knowing that $\sigma$ is
  related to $\model{\sigma}$.
  As $\keep\,\sigma=\varu 0\consu \wkO\,\sigma$, we are left with showing that
  $\wkO\,\sigma=\sigma\circ\wk$ (where $\wk=\wkO \id$)
  relates to its semantic counterpart.

%   {-

% Suppose that Γ ^^ Δ ⊢ and Γ ⊢ E
% The following function computes both Γ ▶ E ^^ wk_E Δ and a substitution
% from this context to Γ ^^ Δ.
% I don't see how to avoid constructing these two components simultaneously

% -}
% ΣwkTel⇒ᵐ :
%   ∀ {Γ}{Γw :  Γ ⊢}(Γm : ∃ (Con~ Γw) )
%     (Em : M.Ty (₁ Γm))
%         {Δ }{Δw : Γ ^^ Δ ⊢}(Δm : ∃ (Con~ Δw)) →
        % ∃ λ T → M.Sub T (₁ Δm)


  To do that, we show that $\wkO$ preserves the relation, for types and terms.
  This requires to generalise a bit and show that $\wk_n$ preserves the relation,
  as $\wk_0(\Pi\,A\,B)=\Pi\,(\wk_0\,A)(\wk_1\,B)$.
  But remember that $\wk_n$ performs a weakening in the middle of a context, so
  we first define the semantic counterpart of this:
\begin{alignat*}{5}
  & \model{\Sigma\wkO{\Rightarrow}}  && :
   % & && \qquad
  (\wf{\Gamma} : \Gamma \vdash) \ra
  (\wf\Gamma \sim \model{\Gamma}) \ra
  \\
  & && \qquad
  (\wf{\Delta} : \Gamma\merge\Delta \vdash) \ra
  (\wf\Delta \sim \model{\Delta}) \ra
  \\ &&& \qquad
  (\model{A} : \model\Ty\model\Gamma)\ra
  (\model{\Delta'} : \model{\Con}) \times (\model{\Sub}\model{\Delta'}{\model{\Delta}})
\end{alignat*}
Here, $\model{\Delta'}$ should be thought of as the context $\model\Delta$ where
the weakening has happened in the middle of the context, by inserting the type
$\model{A}$ after the prefix $\model{\Gamma}$. Indeed, we expect that
$\model{\Gamma}$ is a prefix of $\model{\Delta}$, as $\model{\Gamma}$ relates to
$\Gamma$ and $\model{\Delta}$ to $\Gamma\merge\Delta$.
% $\wf{\Gamma}\sim\model\Gamma$
% and $\wf\Delta\sim\model\Delta$, and $\Gamma$ is a prefix of $\Gamma\merge\Delta$.
The substitution from the weakened context to the original one must
be computed at the same time otherwise the recursion hypothesis is not strong enough.
Then, we seperate the two components under the same (implicit) hypotheses:
\begin{alignat*}{5}
  & \model{\wkO}\,\model{A}\,\model\Delta  && :
  % (\model{A} : \model\Ty\model\Gamma)\ra
  % (\wf{\Gamma} : \Gamma \vdash) \ra
  % (\wf{\Delta} : \Gamma\merge\Delta \vdash) \ra
  % \\
  % & && \qquad
  % (\wf\Gamma \sim \model{\Gamma}) \ra
  % (\wf\Delta \sim \model{\Delta}) \ra
  % \\ &&& \qquad
   \model{\Con}
   \\
  & \model{\wk{\Rightarrow}}\,\model{A}\,\model\Delta  && :
  % (\model{A} : \model\Ty\model\Gamma)\ra
  % (\wf{\Gamma} : \Gamma \vdash) \ra
  % (\wf{\Delta} : \Gamma\merge\Delta \vdash) \ra
  % \\
  % & && \qquad
  % (\wf\Gamma \sim \model{\Gamma}) \ra
  % (\wf\Delta \sim \model{\Delta}) \ra
  % \\ &&& \qquad
   \model{\Sub}(\model{\wkO}\model{A}\,\model\Delta)\model\Delta
\end{alignat*}
Note that if recursion-recursion is available in the metatheory, $\model\wkO$
and $\wk\model{\Rightarrow}$ can be defined directly without introducing this
intermediate $\model{\Sigma\wkO\Rightarrow}$.


Now, we are ready to prove by mutual recursion on well-typed judgments that
weakening preserves typing. The following statements are all under the
hypotheses
  $(\wf{\Gamma} : \Gamma \vdash)$,
  $(\wf\Gamma \sim \model{\Gamma})$,
  $(\wf{\Delta} : \Gamma\merge\Delta \vdash)$,
  $(\wf\Delta \sim \model{\Delta})$,
  $(\wf{A} : \Gamma \vdash A)$,
  and
  $(\wf{A} \sim \model{A})$.
  % , except the last one about weakening substitutions,
  % which do not require that $\model\Delta$ is related to any well-formed context.
\begin{alignat*}{5}
  & \wkO{\sim}  &&
   :
  % (\wf{\Gamma} : \Gamma \vdash) \ra
  % (\wf\Gamma \sim \model{\Gamma}) \ra
  % \\
  %  & && \qquad
  % (\wf{\Delta} : \Gamma\merge\Delta \vdash) \ra
  % (\wf\Delta \sim \model{\Delta}) \ra
  % \\
  % & && \qquad
  % (\wf{A} : \Gamma \vdash A) \ra
  % (\wf{A} \sim \model{A}) \ra
  % \\
  % & && \qquad
  \wf{\wkO}\,\wf{A}\,\wf{\Delta}\sim\model{\wkO}\model{A}\model{\Delta}
  \\
  % wk
  & \wk{\sim} && :
  % (\wf{\Gamma} : \Gamma \vdash) \ra
  % (\wf\Gamma \sim \model{\Gamma}) \ra
  % \\
  %  & && \qquad
  % (\wf{\Delta} : \Gamma\merge\Delta \vdash) \ra
  % (\wf\Delta \sim \model{\Delta}) \ra
  % \\
  % & && \qquad
  % (\wf{A} : \Gamma \vdash A) \ra
  % (\wf{A} \sim \model{A}) \ra
  % \\
  % &&& \qquad
  (\wf{T} : \Gamma\merge\Delta \vdash T) \ra
  (\wf{T} \sim \model{T}) \ra
  % \\ &&& \qquad
  \wf{\wk}\,\wf{A}\,\wf{T}\sim\model{T}\model{[\model{\wkO{\Rightarrow}}\model{A}\model{\Delta}]}
  \\
    % wk
  & \wk{\sim} && :
  % (\wf{\Gamma} : \Gamma \vdash) \ra
  % (\wf\Gamma \sim \model{\Gamma}) \ra
  % \\
  %  & && \qquad
  % (\wf{\Delta} : \Gamma\merge\Delta \vdash) \ra
  % (\wf\Delta \sim \model{\Delta}) \ra
  % \\
  % & && \qquad
  % (\wf{A} : \Gamma \vdash A) \ra
  % (\wf{A} \sim \model{A}) \ra
  % \\
  % &&& \qquad
  (\wf{t} : \Gamma\merge\Delta \vdash t\in T) \ra
  (\wf{t} \sim \model{t}) \ra
  % \\ &&& \qquad
  \wf{\wk}\,\wf{A}\,\wf{t}\sim\model{t}\model{[\model{\wkO{\Rightarrow}}\model{A}\model{\Delta}]}
  \\
    % wk
  & \wk{\sim} && :
  % (\wf{\Gamma} : \Gamma \vdash) \ra
  % (\wf\Gamma \sim \model{\Gamma}) \ra
  % \\
  %  & && \qquad
  % (\wf{\Delta} : \Gamma\merge\Delta \vdash) \ra
  % (\wf\Delta \sim \model{\Delta}) \ra
  % \\
  % & && \qquad
  % (\wf{A} : \Gamma \vdash A) \ra
  % (\wf{A} \sim \model{A}) \ra
  % \\
  % &&& \qquad
  (\wf{x} : \Gamma\merge\Delta \vdash t\invar T) \ra
  (\wf{x} \sim \model{x}) \ra
  % \\ &&& \qquad
  \wf{\wk}\,\wf{A}\,\wf{x}\sim\model{x}\model{[\model{\wkO{\Rightarrow}}\model{A}\model{\Delta}]}
  % \\ &&& \qquad
  % \wf{\wk}\,\wf{A}\,\wf{x}\sim\model{x}\model{[\model{\wkO{\Rightarrow}}\model{A}\model{\Delta}]}
\end{alignat*}
Then we deduce, still by induction, that weakening of substitution preserves the
relation:
\begin{alignat*}{5}
    % wk
  & \wkO{\sim} && :
  (\wf{\Gamma} : \Gamma \vdash) \ra
  (\wf\Gamma \sim \model{\Gamma}) \ra
  % \\
  %  & && \qquad
  % (\wf{\Delta} : \Gamma\merge\Delta \vdash) \ra
  % (\wf\Delta \sim \model{\Delta}) \ra
  % \\
  % & && \qquad
  (\wf{A} : \Gamma \vdash A) \ra
  (\wf{A} \sim \model{A}) \ra
  \\
  &&& \qquad
  (\wf{\sigma} : \Gamma\vdash \sigma \Rightarrow \Delta) \ra
  (\wf{\sigma} \sim \model{\sigma}) \ra
  % \\
  % &&& \qquad
  (\wf{\wkO}\wf{A}\wf\sigma \sim \model{\sigma}\model{\circ}\model{\wk})
  \end{alignat*}



%   But to give a precise meaning to the statement that $\wk_n$ (which weakens at
%   the middle of context) preserve the
%   relation, we define (inductively) the notion of semantic telescopes.
%   First, we define the notion of being a prefix:
% \begin{alignat*}{5}
%   & \blank\leq\blank  && : \model\Con\ra \model\Con\ra\Set \\
%   & {\leq}\cdot  && : \model{\Gamma} \leq \model{\Gamma} \\
%   & \blank{\leq}{\ext}\blank  && : \model{\Gamma} \leq \model{\Delta} \ra (A :
%   \model{\Ty}\Delta) \ra \model{\Gamma}\leq\model{\Delta}\model{\ext}\model{A}
% \end{alignat*}
% Now, a telescope on a semantic context $\model{\Gamma}$ is a semantic context
% $\model{\Delta}$ such that $\model{\Gamma}\leq \model{\Delta}$.

% \begin{alignat*}{5}
%   & \Tel\,\model{\Gamma}  && := (\model{\Delta} : \model{\Con})\times (\model{\Gamma}\leq \model{\Delta})
% \end{alignat*}
% Similarly to the untyped case, we define the merging of a context and a telescope by recursion on the telescope:
% \begin{alignat*}{5}
%   & \blank\model{\merge}\blank  && := (\model{\Gamma} : \model{\Con}) \ra
%   (\model{\Delta} : \Tel\,\model\Gamma) \ra \model{\Con} \\
%   & \model\Gamma \merge (\model\Gamma,{\leq}\cdot) && :=  \model\Gamma \\
%   & \model\Gamma \merge (\model\Delta\model{\ext}\model{A}, \Delta_{\leq} {\leq}{\ext} \model{A}) && :=  (\model\Gamma \model\merge \model\Delta)\model\ext \model{A}\\
% \end{alignat*}
% We then define a relevant relation between syntactic telescopes and semantic telescopes.
% Untyped syntactic telescopes are modeled by untyped contexts, because they are
% list of types.
% A syntactic telescope $\Delta$ is then well formed in a context $\Gamma$ if
% $\Gamma\merge\Delta$ is well formed.
% The relation is defined by recursion on the untyped syntactic telescope:
% \begin{alignat*}{5}
%   & \blank\sim\blank  && := ( \Gamma \merge \Delta \vdash) \ra (\model\Gamma\leq\model\Delta)
%   \ra \Set \\
%   & (\wf\Gamma : \Gamma \merge \emptyu \vdash)\sim\model\Delta  && :=
%   \model\Delta =  {\leq}\cdot\\
%   & (\wf\Delta \extw \wf{A} : \Gamma \merge (\Delta \extu A) \vdash)\sim\model{E}  && :=
%   \sum_{\model\Delta} (\wf\Delta\sim\model\Delta)
%   \times
%   \sum_{\model A} (\wf{A}\sim\model{A}) \times
%   \\
%   & && \qquad
%   \model{E} = \model\Delta {\leq}{\ext} \model{A}
% \end{alignat*}

\subsubsection{Stability under substitution}

 We want to prove that given any well-typed substitution $\sigma
: \Sub\,\Gamma\,\Delta$ and semantic contexts $\model{\Gamma}$ and
$\model{\Delta}$, respectively related to $\Gamma$ and $\Delta$, there exists a
semantic substitution which is related to $\sigma$.  In the extension case
$\Gamma\vdash \sigma \consu\,t\Rightarrow \Delta\,\extu\,A$, the induction
hypothesis provides $\model{\sigma}$, $\model{\Delta}$, $\model{A}$ related to
their syntactic counterpart. However, the premises of the induction hypothesis
for getting a relevant $\model{t}$ require showing that the type
$\model{A}\model{[\model{\sigma}]}$ is related to the syntactic type
$A[\sigma]$. We first establish preservation of the relation by substitution for
variables:
\begin{alignat*}{5}
    % []V~
  & []{\sim} && :
  (\wf{\sigma} : \Gamma \vdash \sigma \Rightarrow \Delta) \ra
  (\wf\sigma \sim \model{\sigma}) \ra
  (\wf{x} : \Delta \vdash x \invar A) \ra
  (\wf{x} \sim \model{x}) \ra
  \\ & && \qquad
  % \\
  % & && \qquad
   \wf{[]}\wf{x}\wf{\sigma} \sim \model{x}\model{[\model{\sigma}]}
\end{alignat*}
Then we show it for terms and types  by mutual induction
under the common hypotheses
  $(\wf{\sigma} : \Gamma \vdash \sigma \Rightarrow \Delta)$,
  $(\wf\sigma \sim \model{\sigma})$,
  $(\wf{\Gamma} : \Gamma \vdash )$,
  $(\wf{\Gamma} \sim \model{\Gamma})$,
  % \\
  % &&&\qquad
    $(\wf{\Delta} : \Delta \vdash )$,
    and
  $(\wf{\Delta} \sim \model{\Delta})$:
\begin{alignat*}{5}
     & []{\sim} && :
    (\wf{A} : \Delta \vdash A) \ra
  (\wf{A} \sim \model{A}) \ra
  % \\
  % & && \qquad
   \wf{[]}\wf{\Gamma}\wf{A}\wf{\sigma} \sim \model{A}\model{[\model{\sigma}]}
   \\
     & []{\sim} && :
    (\wf{t} : \Delta \vdash t \in A ) \ra
  (\wf{t} \sim \model{t}) \ra
  % \\
  % & && \qquad
   \wf{[]}\wf{\Gamma}\wf{t}\wf{\sigma} \sim \model{t}\model{[\model{\sigma}]}
  \end{alignat*}
  Eventually, we show under the same hypotheses the following statement
  \begin{alignat*}{5}
     & {\circ}{\sim} && :
    (\wf{E} : E \vdash ) \ra
  (\wf{E} \sim \model{E}) \ra
  % \\
  % &&&  \qquad
    (\wf{\delta} : \Delta \vdash \delta \Rightarrow E) \ra
  (\wf{\delta} \sim \model{\delta}) \ra
  \\
  & && \qquad
   \wf{{\circ}}\wf{\Gamma}\wf{\delta}\wf{\sigma} \sim \model{\delta}\model{\circ}\model{\sigma}
  \end{alignat*}
  and the fact that identity preserves the relation:
  \begin{alignat*}{5}
     & {\id}{\sim} && :
    (\wf{\Gamma} : \Gamma \vdash ) \ra
  (\wf{\Gamma} \sim \model{\Gamma}) \ra
   \wf{{\id}}\wf{\Gamma} \sim {\id}_{\model{\Gamma}}
  \end{alignat*}
% $\model{t}\model\oldapp\model{u}$, but it has type
% $\model{B}\model{[0 := \model{u}]}$
% instead of type $\model{T}$. Fortunately, we know by hypothesis that $\model{T}$
% relates to $T=B[0:=u]$ so, by uniqueness of the related semantic counterpart,
% we can deduce that $\model{T}=\m$
% Thus, we need to show that $\model{T}=\model{B}\model{[0 := \model{u}]}$ if we
% know that this last type is related to $B[0 := u]$.


\subsection{The Recursor}
For the recursor, we build a morphism from the syntactic model to the semantic
one. The image of a syntactic context is a unique semantic context which is
related to it, and similarly for types and terms.
% We construct it as the semantic
% Induction on the typing judgments shows that any QIIT morphism from the
% syntactic model $S$  to a semantic model $M$ sends well-typed syntax on a related
% semantic counter-part.
Thus, as a first step, we use induction on well-formedness judgments to construct
semantic counterparts:
  \begin{alignat*}{5}
     & \Sigma\Con{\sim} && :
    (\wf{\Gamma} : \Gamma \vdash ) \ra
    \sum_{\model{\Gamma}} \wf{\Gamma} \sim \model{\Gamma}
    \\
     & \Sigma\Ty{\sim} && :
    (\wf{\Gamma} : \Gamma \vdash ) \ra
    (\wf{\Gamma}\sim \model\Gamma) \ra
    (\wf{A} : \Gamma \vdash A ) \ra
    ({\model{A}:\model{\Ty}\model\Gamma})\times ( \wf{A} \sim \model{A})
    \\
     & \Sigma\Tm{\sim} && :
    (\wf{\Gamma} : \Gamma \vdash ) \ra
    (\wf{\Gamma}\sim \model\Gamma) \ra
    (\wf{A} : \Gamma \vdash A ) \ra
    ( \wf{A} \sim \model{A}) \ra
    \\ & && \qquad
    (\wf{t} : \Gamma \vdash t \in A ) \ra
    ({\model{t}:\model{\Tm}\model\Gamma}\model{A})\times ( \wf{t} \sim \model{t})
    \\
     & \Sigma\Var{\sim} && :
    (\wf{\Gamma} : \Gamma \vdash ) \ra
    (\wf{\Gamma}\sim \model\Gamma) \ra
    (\wf{A} : \Gamma \vdash A ) \ra
    ( \wf{A} \sim \model{A}) \ra
    \\ & && \qquad
    (\wf{x} : \Gamma \vdash x \invar A ) \ra
    ({\model{x}:\model{\Tm}\model\Gamma}\model{A})\times ( \wf{x} \sim \model{x})
    \\
     & \Sigma\Sub{\sim} && :
    (\wf{\Gamma} : \Gamma \vdash ) \ra
    (\wf{\Gamma}\sim \model\Gamma) \ra
    (\wf{\Delta} : \Delta \vdash  ) \ra
    ( \wf{\Delta} \sim \model{\Delta}) \ra
    \\ & && \qquad
    (\wf{\sigma} : \Gamma \vdash \sigma \Rightarrow \Delta ) \ra
    ({\model{\sigma}:\model{\Sub}\model\Gamma}\model{\Delta})\times ( \wf{\sigma} \sim \model{\sigma})
  \end{alignat*}
  The right uniqueness of the relation is used in this induction. It is then
straightforward to show (without induction) that the first projection of these
constructions yield a model morphism from the syntax to the model, again using
right uniqueness.
\subsection{Uniqueness}
Our goal is to show that the syntactic model is initial. Thus, it remains to
show that the morphism constructed in the previous section is unique. We exploit
right uniqueness of the relation: it is enough to show that any such morphism
maps a syntactic context to a related semantic context, and similarly for types
and terms.

More formally, we assume an arbitrary morphism from the
syntax to the model, inducing the following maps:
\begin{alignat*}{5}
  &
  \mor{\Con}
  && :
   (\Gamma \vdash) \ra \model\Con
   \\
  &
  \mor{\Ty}
  && :
   (\wf\Gamma:\Gamma \vdash) \ra (\Gamma\vdash A)\ra\model\Ty\,(\mor\Con\wf\Gamma)
   \\
  &
  \mor{\Tm}
  && :
  (\wf\Gamma:\Gamma \vdash) \ra (\wf{A}:\Gamma\vdash A)\ra
  (\Gamma\vdash t \in A)\ra\model\Tm\,(\mor\Con\wf\Gamma)\,
  (\mor\Ty\wf\Gamma\,\wf{A})
   \\
  &
  \mor{\Var}
  && :
  (\wf\Gamma:\Gamma \vdash) \ra (\wf{A}:\Gamma\vdash A)\ra
  (\Gamma\vdash x \invar A)\ra\model\Tm\,(\mor\Con\wf\Gamma)\,
  (\mor\Ty\wf\Gamma\,\wf{A})
   \\
  &
  \mor{\Sub}
  && :
  (\wf\Gamma:\Gamma \vdash) \ra
  (\wf\Delta:\Delta \vdash) \ra
  (\Gamma\vdash \sigma \Rightarrow \Delta)\ra\model\Sub\,(\mor\Con\wf\Gamma)\,
  (\mor\Con\wf\Delta)
\end{alignat*}
Then, we show by induction on typing judgments that
the image is related:
\begin{alignat*}{5}
  &
  {\sim}\mor{\Con}
  && :
  (\wf\Gamma : \Gamma \vdash) \ra \wf\Gamma\sim \mor\Con\,\wf\Gamma
  \\
  &
  {\sim}\mor{\Ty}
  && :
  (\wf\Gamma : \Gamma \vdash) \ra
  (\wf{A} : \Gamma \vdash A) \ra
  \wf\Gamma\sim \mor\Ty\,\wf\Gamma\,\wf{A}
  \\
  &
  {\sim}\mor{\Tm}
  && :
  (\wf\Gamma : \Gamma \vdash) \ra
  (\wf{A} : \Gamma \vdash A) \ra
  (\wf{t} : \Gamma \vdash t \in A) \ra
  \wf\Gamma\sim \mor\Tm\,\wf\Gamma\,\wf{A}\,\wf{t}
  \\
  &
  {\sim}\mor{\Var}
  && :
  (\wf\Gamma : \Gamma \vdash) \ra
  (\wf{A} : \Gamma \vdash A) \ra
  (\wf{x} : \Gamma \vdash x \invar A) \ra
  \wf\Gamma\sim \mor\Var\,\wf\Gamma\,\wf{A}\,\wf{x}
  \\
  &
  {\sim}\mor{\Sub}
  && :
  (\wf\Gamma : \Gamma \vdash) \ra
  (\wf{\Delta} : \Delta \vdash) \ra
  (\wf{\sigma} : \Gamma \vdash \sigma \Rightarrow \Delta) \ra
  \wf\Gamma\sim \mor\Sub\,\wf\Gamma\,\wf{\Delta}\,\wf{\sigma}
\end{alignat*}
This justifies the uniqueness of the morphism, by right uniqueness of
$\blank\sim\blank$.


% \begin{description}
%   \item[untyped syntax and well typed judgments as an algebra]
%   \begin{enumerate}
%   \item define untyped syntax as an inductive datatype
%   \item define operations on the syntax, such as substitution, by recursion
%   \item define well-typed judgments as an inductive datatype indexed over the untyped syntax
%   \item show the dependant pair of untyped syntax with well-typed judgments
%     define an algebra
%   \end{enumerate}
%   \item[specification of the initial morphism as a functional relation]
%   \begin{enumerate}
%   \item given an algebra, define the functional relation enjoyed by the recursor
%     from the syntax to the algebra by recursion over the well-typed judgment
%   \item show right-uniqueness of the relation
%   \item show left-totality of the relation
%   \end{enumerate}
%   \item[existence and uniqueness of the morphism from the syntax to a model]
%     \begin{enumerate}
%   \item show uniqueness of such a morphism
%   \item extract an algebra morphism from the relation
%     \end{enumerate}
% \end{description}


%%%%%%%%%%%%%%%%%%%%%%%%%%%%%%%%%%%%%%%%%%%%%%%%%%%%%%%%%%%%%%%%%%%%%%%%%%%

\section{Conclusions}
\label{sec:conclusions}

TODO

The current work only concerns finitary IITs. An extension would be to also
allow infinitely branching inductive types such as W-types. This would first
require giving semantics for infinitary IITs (to our knowledge there is no
previously published semantics that we can borrow), and also giving a term model
construction analogously to finitary QIITs. These steps seem feasible. However,
it seems to be more difficult to construct the syntax of infinitary IIT
signatures without using quotients. The reason is that such syntax would not be
strictly restricted to neutral terms: we would need $\lambda$-abstraction and
$\beta\eta$-rules for infinitary $\Pi$ types, in order to allow a term model
construction for infinitary IITs. A definition of normal preterms and typing
judgments on them may still be possible, but it appears to be much more
complicated than before (the current authors have attempted this without
conclusive success).

As mentioned in Section \ref{sec:right_uniqueness}, it also remains an open
problem whether IITs are reducible to inductive types in a UIP-free setting. To
show this, we would need to construct the syntax of signatures without UIP, and
also reproduce the semantics and term model construction for IITs without UIP.

%%%%%%%%%%%%%%%%%%%%%%%%%%%%%%%%%%%%%%%%%%%%%%%%%%%%%%%%%%%%%%%%%%%%%%%%%%%

\bibliography{b}

%%%%%%%%%%%%%%%%%%%%%%%%%%%%%%%%%%%%%%%%%%%%%%%%%%%%%%%%%%%%%%%%%%%%%%%%%%%

\newpage

\appendix

\end{document}
