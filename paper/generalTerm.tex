\documentclass[a4paper,UKenglish,cleveref, autoref]{lipics-v2019}
\usepackage{amssymb}
\usepackage{amsmath}
\usepackage{amsthm}
\usepackage{hyperref}
\usepackage{todonotes}
\presetkeys{todonotes}{inline}{}
\usepackage{csquotes} %for \llbracket and \rrbracket
\usepackage{stmaryrd}
% \usepackage{pdflscape} % if we want \begin{landscape} ... \end{landscape}

%\graphicspath{{./graphics/}}%helpful if your graphic files are in another directory

\bibliographystyle{plainurl}% the mandatory bibstyle

\begin{document}

%ppp metatheory
\newcommand{\blank}{\mathord{\hspace{1pt}\text{--}\hspace{1pt}}} %from the book
\newcommand{\ra}{\rightarrow}
\newcommand{\Set}{\mathsf{Set}}
\newcommand{\Prop}{\mathsf{Prop}}

% object theory: universal QIIT
\newcommand{\Con}{\mathsf{Con}}
\newcommand{\Ty}{\mathsf{Ty}}
\newcommand{\Sub}{\mathsf{Sub}}
\newcommand{\Tm}{\mathsf{Tm}}
\newcommand{\Var}{\mathsf{Var}}
\newcommand{\id}{\mathsf{id}}
\newcommand{\ass}{\mathsf{ass}}
\newcommand{\idl}{\mathsf{idl}}
\newcommand{\idr}{\mathsf{idr}}
\newcommand{\ext}{\rhd}
\newcommand{\p}{\mathsf{p}}
\newcommand{\q}{\mathsf{q}}
\newcommand{\U}{\mathsf{U}}
\newcommand{\El}{\mathsf{El}}
\newcommand{\app}{\mathsf{app}}
\newcommand{\oldapp}{\mathop{{\scriptstyle @}}}
\newcommand{\Pim}{\hat{\Pi}}
\newcommand{\appm}{\mathop{\hat{{\scriptstyle @}}}}
\newcommand{\Pii}{\tilde{\Pi}}
\newcommand{\appi}{\mathop{\tilde{{\scriptstyle @}}}}
\newcommand{\Id}{\mathsf{Id}}
\newcommand{\reflect}{\mathsf{reflect}}

% untyped syntax
\newcommand{\untyp}[1]{{#1}^{\mathsf{p}}}
\newcommand{\Conu}{\untyp{\Con}}
\newcommand{\Tyu}{\untyp{\Ty}}
\newcommand{\Subu}{\untyp{\Sub}}
\newcommand{\Tmu}{\untyp{\Tm}}
\newcommand{\idu}{\untyp{\id}}
\newcommand{\emptyu}{\untyp{\cdot}}
\newcommand{\epsilonu}{\untyp{\epsilon}}
\newcommand{\extu}{\mathbin{\untyp{\ext}}}
\newcommand{\consu}{\mathbin{\untyp{\cons}}}
\newcommand{\varu}{\untyp{\mathsf{var}}}
\newcommand{\Uu}{\untyp{\U}}
\newcommand{\Elu}{\untyp{\El}}
\newcommand{\Piu}{\untyp{\Pi}}
\newcommand{\appu}{\mathbin{\untyp{{\scriptstyle @}}}}
\newcommand{\Pipu}{\untyp{\Pi}}
\newcommand{\Pimu}{\untyp{\Pim}}
\newcommand{\Piiu}{\untyp{\Pii}}
\newcommand{\appmu}{\mathop{\hat{\tilde{{\scriptstyle @}}}}}
\newcommand{\erru}{\untyp{\mathsf{err}}}
\newcommand{\wku}{\untyp{\wk}}
\newcommand{\idru}{\untyp{\idr}}
\newcommand{\idlu}{\untyp{\idl}}

% well-typed judgement
\newcommand{\wf}[1]{{#1}^{\mathsf{w}}}
\newcommand{\Conw}{\wf{\Con}}
\newcommand{\Tyw}{\wf{\Ty}}
\newcommand{\Subw}{\wf{\Sub}}
\newcommand{\Tmw}{\wf{\Tm}}
\newcommand{\invar}{\in_\N}
\newcommand{\idw}{\wf{\id}}
\newcommand{\emptyw}{\wf{\cdot}}
\newcommand{\extw}{\mathbin{\wf{\ext}}}
\newcommand{\consw}{\wf{\cons}}
\newcommand{\epsilonw}{\wf{\epsilon}}
\newcommand{\varw}{\wf{\mathsf{var}}}
\newcommand{\vzw}{\wf{\mathsf{0}}}
\newcommand{\vsw}{\wf{\mathsf{S}}}
\newcommand{\Uw}{\wf{\U}}
\newcommand{\Elw}{\wf{\El}}
\newcommand{\Piw}{\wf{\Pi}}
\newcommand{\appw}{\wf{\app}}
\newcommand{\Pipw}{\wf{\Pi}}
\newcommand{\Pimw}{\wf{\Pim}}
\newcommand{\appmw}{\wf{\hat{\app}}}
\newcommand{\Piiw}{\wf{\Pii}}
\newcommand{\appiw}{\wf{\tilde{\app}}}

% model
\newcommand{\model}[1]{{#1}^M}

% morphism of model
\newcommand{\mor}[1]{{#1}^f}

% syntax model
\newcommand{\syn}[1]{{#1}^{\mathsf{S}}}

% functional relation
\newcommand{\rel}[1]{{#1}^r}
\newcommand{\Conr}{\rel{\Con}}
\newcommand{\Tyr}{\rel{\Ty}}
\newcommand{\Subr}{\rel{\Sub}}
\newcommand{\Tmr}{\rel{\Tm}}
\newcommand{\emptyr}{\rel{\cdot}}
\newcommand{\extr}{\rel{\ext}}
\newcommand{\varr}{\rel{\mathsf{var}}}
\newcommand{\vzr}{\rel{0}}
\newcommand{\vsr}{\rel{S}}
\newcommand{\Ur}{\rel{\U}}
\newcommand{\Elr}{\rel{\El}}
\newcommand{\Pir}{\rel{\Pi}}
\newcommand{\appr}{\rel{\app}}
\newcommand{\Pipr}{\rel{\Pi}}
\newcommand{\Pimr}{\rel{\Pim}}
\newcommand{\appmr}{\rel{\hat{\app}}}
\newcommand{\Piir}{\rel{\Pii}}
\newcommand{\appir}{\rel{\tilde{\app}}}

\newcommand{\relT}[3]{#1 \sim_{#2} #3}
\newcommand{\relt}[4]{#1 \sim_{#2\vdash #3} #4}
\newcommand{\relV}[4]{#1 \sim_{#2\vdash #3} #4}
\newcommand{\relS}[4]{#1 \sim_{#2\Rightarrow #3} #4}




\newcommand{\A}{\mathsf{A}}
\newcommand{\F}{\mathsf{F}}
\renewcommand{\S}{\mathsf{S}}
\renewcommand{\O}{\mathsf{O}}
\newcommand{\M}{\mathsf{M}}
\newcommand{\con}{\mathsf{con}}
\newcommand{\elim}{\mathsf{elim}}
\newcommand{\nat}{\mathsf{nat}}
\newcommand{\map}{\mathsf{map}}
\newcommand{\List}{\mathsf{List}}
\newcommand{\lb}{\langle}
\newcommand{\rb}{\rangle}
\newcommand{\wk}{\mathsf{wk}}
\newcommand{\wkO}{\wk_0}
\newcommand{\keep}{\mathsf{keep}}
\newcommand{\cons}{,}
\newcommand{\nil}{\mathsf{nil}}
\renewcommand{\merge}{;}
\newcommand{\length}[1]{\| #1 \|}
\newcommand{\vz}{\mathsf{vz}}
\newcommand{\vs}{\mathsf{vs}}
\newcommand{\Ra}{\Rightarrow}
\renewcommand{\tt}{\mathsf{tt}}
\newcommand{\proj}{\mathsf{proj}}
\newcommand{\refl}{\mathsf{refl}}
\newcommand{\J}{\mathsf{J}}
\newcommand{\tr}{\mathsf{tr}}
\newcommand{\trans}{\mathbin{\raisebox{0.2ex}{$\displaystyle\centerdot$}}}
\newcommand{\ap}{\mathsf{ap}}
\newcommand{\apd}{\mathsf{apd}}
\newcommand{\rec}{\mathsf{rec}}
\newcommand{\C}{\mathsf{C}}
\newcommand{\R}{\mathsf{R}}
\newcommand{\E}{\mathsf{E}}
\newcommand{\transp}{\mathsf{transp}}
\newcommand{\Ram}{\mathbin{\hat{\Ra}}}
\newcommand{\funext}{\mathsf{funext}}
\newcommand{\UIP}{\mathsf{UIP}}
\newcommand{\coe}{\mathsf{coe}}
\newcommand{\LET}{\mathsf{let}}
\newcommand{\IN}{\mathsf{in}}
\newcommand{\N}{\mathbb{N}}
\newcommand{\D}{\mathsf{D}}
\newcommand{\K}{\mathsf{K}}
\newcommand{\Eq}{\mathsf{Eq}}
\newcommand{\mk}{\mathsf{mk}}
\newcommand{\unk}{\mathsf{unk}}
\newcommand{\0}{\mathit{0}}
\newcommand{\1}{\mathit{1}}
\newcommand{\eqreflect}{\mathsf{eqreflect}}

\newcommand{\isaprop}{\mathsf{is\text{-}prop}}
\newcommand{\isp}[1]{{#1}^{\mathsf{p}}}

\mathchardef\mhyphen="2D

\renewcommand{\ll}{\llbracket}
\newcommand{\rr}{\rrbracket}

\newcommand{\SignAlg}{\mathsf{SignAlg}}
\newcommand{\SignMor}{\mathsf{SignMor}}
\newcommand{\Sign}{\mathsf{Sign}}
\newcommand{\ADS}{\mathsf{ADS}}
\newcommand{\Bool}{\mathsf{Bool}}
\newcommand{\I}{\mathsf{I}}
\newcommand{\IE}{\mathsf{IE}}



\begin{definition}[Term signature algebra $\A\blank\C_{\blank}$]
  For an $M:\SignAlg$ and $\Omega:\Con^M$, we define ${\A M\C}_\Omega$
  which we call the term signature algebra. It is equipped with
  morphisms $\blank^\A : \SignMor\,({\A M\C}_\Omega)\,\A$ and
  $\blank^M : \SignMor\,({\A M\C}_\Omega)\,M$.

  We define ${\A M\C}_\Omega$ by listing its components $\Con$, $\Ty$,
  $\Sub$, and so on, one per row. Each component has three parts
  denoted by $^\A$, $^M$ and $^\C$. The $^\A$ and $^M$ parts are just
  reusing the corresponding components from $\A$ and $M$,
  respectively, and thus the morphisms $\blank^\A$ and $\blank^M$ are
  defined as the obvious projections. We omit the equality components,
  as they come from UIP or are trivial. We also omit the components
  for terms and substitutions as their $^\C$ parts are uninformative
  equational reasonings.
  \begin{alignat*}{10}
    & \Gamma^\A:\Set && \hspace{0.4em} && \Gamma^M:\Con^M && \hspace{0.4em} && \Gamma^\C: \Sub^M\,\Omega\,\Gamma^M\ra \Gamma^\A \\
    & A^\A:\Gamma^\A\ra\Set && && A^M:\Ty^M\,\Gamma^M && && A^\C:(\nu:\Sub^M\,\Omega\,\Gamma^M)\ra \\
    & && && && && \hspace{1em} \Tm^M\,\Omega\,(A^M[\nu])\ra A^\A\,(\Gamma^\C\,\nu) \\
    & \sigma^\A:\Gamma^\A\ra\Delta^\A && && \sigma^M:\Sub^M\,\Gamma^M\,\Delta^M && && \sigma^\C:\Delta^\C\,(\sigma^M\circ\nu)=\sigma^A\,(\Gamma^\C\,\nu) \\
    & t^\A:(\gamma:\Gamma^\A)\ra A^\A\,\gamma && && t^M:\Tm^M\,\Gamma^M\,A^M && && t^\C : A^\C\,\nu\,(t^M[\nu])= t^A\,(\Gamma^\C\,\nu) \\
    & (A[\sigma])^\A\,\gamma := A^\A\,(\sigma^\A\,\gamma) && && (A[\sigma])^M := A^M[\sigma^M]^M && && (A[\sigma])^\C\,\nu\,t := A^\C\,(\sigma^M\circ\nu)\,t \\
    & \cdot^\A := \top && && \cdot^M := \cdot^M && && \cdot^\C\,\nu := \tt \\
    & (\Gamma\rhd A)^\A := && && (\Gamma\rhd A)^M :=  && && (\Gamma\rhd A)^\C\,\nu := \\
    & \hspace{1em} (\gamma:\Gamma^\A)\times A^\A\,\gamma && && \hspace{1em} \Gamma^M\rhd^M A^M && && \hspace{1em} (\Gamma^\C\,(\pi_1\,\nu), A^\C\,(\pi_1\,\nu)\,(\pi_2\,\nu)) \\
    & \U^\A\,\gamma := \Set && && \U^M := \U^M && && \U^\C\nu\,a := \Tm^M\,\Omega\,(\El^M\,a) \\
    & (\El\,a)^\A\,\gamma := a^\A\,\gamma && && (\El\,a)^M := \El^M\,a^M && && (\El\,a)^\C\,\nu\,t := (a^\C\,\nu) \# t \\
    & (\Pi\,a\,B)^\A\,\gamma := && && (\Pi\,a\,B)^M := && && (\Pi\,a\,B)^\C\,\nu\,t :=  \\
    & \hspace{1em} (\alpha:a^\A\,\gamma)\ra B^\A\,(\gamma,\alpha) && && \hspace{1em} \Pi^M\,a^M\,B^M && && \hspace{1em} \lambda\alpha.B^\C\,\big(\nu, (a^\C\,\nu)\#\alpha\big)\,\big(t\oldapp ((a^\C\,\nu) \#\alpha)\big) \\
    & (\Pim\,T\,B)^\A\,\gamma := && && (\Pim\,T\,B)^M := && && (\Pim\,T\,B)^\C\,\nu\,t := \lambda\alpha.(B\,\alpha)^\C\,\nu\,(t\appm\alpha) \\
    & \hspace{1em} (\alpha:T)\ra(B\,\alpha)^\A\,\gamma && && \hspace{1em} \Pim^M\,T\,B^M && && 
  \end{alignat*}
\end{definition}

\begin{remark}
  Given a syntactic signature $\Omega : \Con^\I$, we can use the term
  signature algebra with choices other than $M = \I$. For example, if
  $M = \A$, we usual get the Church encoding by
  $(\ll\Omega\rr_{{\A\A\C}_{\ll\Omega\rr^\A}})^\C\,\id^\A$. In the
  case of natural numbers, this computes (up to curry-uncurry) to
  $(N:\Set)(z:N)(s:N\ra N)\ra N$. Using $M = {\A\M}$ (the signature
  algebra $\A$ extended with the binary version of $\D$ also called
  logical relations, see \cite[Section
  5]{Kaposi:2019:CQI:3302515.3290315}), we obtain the Church encoding
  of Awodey, Frey and Speight \cite{DBLP:conf/lics/AwodeyFS18}.
\end{remark}

\end{document}